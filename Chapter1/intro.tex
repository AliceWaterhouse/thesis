%!TEX root = ../thesis.tex
%*******************************************************************************
%****************************** Introduction *********************************
%*******************************************************************************


\chapter{Introduction}\label{chap:intro}

%The relationship of geometry to non--linear PDEs is close and has long been known.
%\begin{itemize}
%%\item Can define objects on a manifold satisfying certain properties expressed in terms of PDEs in the coordinates. e.g. GR. Generalises to projective and conformal geometry. (chapters \ref{chap:KK_lift} and \ref{chap:c-proj})
%\item Results from integrable systems reveal unexpected examples of PDEs which can be expressed in terms of geometrical conditions. (Ward). (chapter \ref{chap:EW_and_toda})
%\item Geometry also important in the theory of solitons (e.g. Manton and Sutcliffe.) (chapter \ref{chap:wormhole}).
%\end{itemize} 

This thesis splits naturally into two parts. In the first chapter, we discuss a solitonic solution of the equations of motion associated with $\phi^4$--field theory on a wormhole spacetime. \mynote{More detail here. Motivate it I guess, rather than just repeating what's in the abstract.} 

The remainder of the thesis is concerned with a class of Einstein manifolds which can be canonically constructed from projective structures as shown in recent work by Dunajski and Mettler \cite{DM}. A projective structure on a manifold specifies a preferred curve in every direction at every point. It can be understood as an equivalence class of affine connections which share the same geodesics up to parametrisation. Because it is defined at the level of the connection rather than a first order structure such as a metric, it is an intrinsically second order object.

Given a projective structure on a manifold $N$ of dimension $n$,
Dunajski and Mettler \cite{DM} canonically construct a neutral signature Einstein metric $g$ with non-zero
scalar curvature on a certain rank $n$ affine bundle $M\rightarrow N$. They thus convert a second order object into a first order object on a larger manifold.
The $2n$--dimensional space $M$ also carries a natural symplectic form $\Omega$, and an endomorphism $J:TM\rightarrow TM$ which is such that $J^2$ is the identity and $g(\cdot\,,\cdot)=\Omega(\cdot\,,J\cdot)$. This makes $(M,g,\Omega)$ a so--called \textit{almost para--K\"ahler} structure. It is interesting for a number of different reasons.

Firstly, $(M,g)$ is interesting by virtue of being an Einstein
space. In fact, it turns out that $g$ arises as the Kaluza--Klein reduction of an Einstein metric $\mathcal{G}$ on an $\R^*$ bundle $\sigma:\mathcal{Q}\rightarrow M$ which has curvature form $\sigma^*(\Omega)$. In chapter \ref{chap:KK_lift} we will construct $\mathcal{G}$ explicitly, and give an interpretation of the manifold $(\mathcal{Q},\mathcal{G})$ in terms of the projective geometry on $N$.

The work in chapter \ref{chap:EW_and_toda} is based on the fact that for $n=2$ (so that $M$ has dimension $4$), the conformal
curvature of $g$ is anti-self-dual. Recall that the Hodge operator
$\star$ defined by a Euclidean or neutral signature metric in four
dimensions is an involution on two--forms (i.e. squares to the identity).
It thus has eigenvalues $\pm1$, and the space of two-forms splits
into the corresponding eigenspaces, which are referred to as self--dual
(SD) or anti--self--dual (ASD) respectively. Due to its index symmetries,
the Weyl tensor can be thought of as a map from two-forms to two-forms,
and therefore has a corresponding decomposition. Since the Weyl tensor
encodes the conformal curvature, we say that a conformal or (psuedo--)Riemannian
manifold whose Weyl tensor is ASD is equipped with an \textit{ASD
conformal structure}.

The field equations corresponding to anti-self-duality of the Weyl
tensor in four dimensions can be solved by a twistor construction,
and are thus \textit{integrable} \cite{ward}. This means that any systems of differential
equations which can be obtained from them by symmetry reduction should
also be integrable (see \cite{MW} for a review). In particular, the class of dispersionless
integrable systems in 2+1 and 3 dimensions arise in this way. The
construction \cite{DM} provides some examples of
ASD conformal structures in neutral signature which, in the presence
of a (non--null) symmetry, give rise to solutions of an integrable
system called the $SU(\infty)$--Toda field equation via $2+1$--dimensional
Einstein-Weyl structures. In chapter \ref{chap:EW_and_toda} we discuss the extraction
of all possible Toda solutions obtainable in this way.

In chapter \ref{chap:c-proj} we return to projective structures of any dimension, and show that the structure $(M,g,\Omega)$ can be thought of as compactifiable in a certain sense. Recall that a (psuedo--)Riemannian manifold $(M,g)$ is said to be \textit{conformally compact} if there is a smooth positive function $T$ such that $T^2g$ smoothly extends to a manifold with boundary $\ov{M}=M\cup\p M$, and the set $\{m\in\ov{M}:T(m)=0\}$ is a hypersurface which coincides with the boundary $\p M$.  This is a useful concept because $(M,T^2g)$ has the same conformal structure, and hence the same \textit{causal} structure, as $(M,g)$. It has been used to study said causal structure in both general relativity \cite{penrose65} and quantum field theory \cite{witten}. It is also useful for formulating the boundary conditions of conformally invariant field equations such as those arising in Yang--Mills theory \cite{uhlen}.

Recent work by \v Cap and Gover \cite{CG0,CG} has generalised this idea to other geometrical structures which admit some weakening which extends to a manifold with boundary. In particular, on an almost complex manifold $(M,J)$ with a connection $\nabla$ which is correctly adapted to $J$, one can define the $c$--projective equivalence class $[\nabla]$ to which $\nabla$ belongs, and show that the $c$--projective structure $(M,J,[\nabla])$ extends to a manifold with boundary $\ov{M}$ \cite{CG}. The main goal of chapter \ref{chap:c-proj} is to adapt the work of \cite{CG} to the para--$c$--projective case, and to show that the almost complex structure $J$ on $M$ has a complex connection which admits a so--called para--$c$--projective compactification. The result of this is that the manifolds $(M,g,\Omega)$ can be thought of as para--$c$--projectively compact.

\mynote{some notation and conventions e.g. $i,j,k=1,...,n$ are indices for coordinates on $n$? and $\Gamma$ are connection components which are defined by... Curvature is defined by}
\begin{itemize}
\item We denote the bundle of anti--symmetrised covariant tensors of degree $m$ as $\Lambda^m$, and call sections of this bundle $m$--forms.
\item Note that our conventions are $(d\omega)_{ab\dots c}=\partial_{[a}\omega_{b\dots c]},$
$(\eta\wedge\omega)_{a\dots d}=\eta_{[a\dots b}\omega_{c\dots d]},$
$\omega=\omega_{a\dots b}dx^{a}\wedge\dots\wedge dx^{b},$
and $F_{ab}{d}x^{a}\wedge{d}x^{b}=F_{[ab]}{d}x^{a}\otimes{d}x^{b}$
implying ${d}x^{a}\wedge{d}x^{b}=\frac{1}{2}({d}x^{a}\otimes{d}x^{b}-{d}x^{b}\otimes{d}x^{a})$.
%\item $N$ is oriented. \mynote{Although $\RP^n$ is not oriented right?? also something about which part of $SL(n+1)$ we mean?}
\item Probably a convention for the Hodge operator
\item Probably a convention for $\varepsilon_{AB}$ and matching indices descending to the right
\item Line projectivisation except where stated otherwise.
\end{itemize}






