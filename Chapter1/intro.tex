%!TEX root = ../thesis.tex
%*******************************************************************************
%****************************** Introduction *********************************
%*******************************************************************************


\chapter{Introduction}\label{chap:intro}

%The relationship of geometry to non--linear PDEs is close and has long been known.
%\begin{itemize}
%%\item Can define objects on a manifold satisfying certain properties expressed in terms of PDEs in the coordinates. e.g. GR. Generalises to projective and conformal geometry. (chapters \ref{chap:KK_lift} and \ref{chap:c-proj})
%\item Results from integrable systems reveal unexpected examples of PDEs which can be expressed in terms of geometrical conditions. (Ward). (chapter \ref{chap:EW_and_toda})
%\item Geometry also important in the theory of solitons (e.g. Manton and Sutcliffe.) (chapter \ref{chap:wormhole}).
%\end{itemize} 

This thesis splits naturally into two parts. The first 4 chapters are concerned with a class of Einstein manifolds which can be canonically constructed from projective structures as shown in recent work by Dunajski and Mettler \cite{DM}. Chapter \ref{chap:wormhole} discusses a solitonic solution of the equations of motion associated with $\phi^4$--field theory on a wormhole spacetime. Chapter \ref{chap:wormhole} is self--contained, and possesses its own introduction. The remainder of this chapter will introduce some preliminaries for the first part of the thesis, culminating in the results of \cite{DM} on the Einstein manifolds which will form the basis of chapters 2--4.




\begin{defi} A projective structure $(N,[\nabla])$
on a manifold $N$ is an equivalence class $[\nabla]$ of torsion--free affine connections
on $N$ which have the same unparametrised geodesics. For two sets
of connection components $\Gamma_{jk}^{i}$ to represent connections
in the same projective class, they must be related by
\begin{equation}
\delta\Gamma_{jk}^{i}=\delta_{j}^{i}\Upsilon_{k}+\delta_{k}^{i}\Upsilon_{j}\label{eq:proj_change}
\end{equation}
for some 1-form $\Upsilon.$
\end{defi}

Given a projective structure on a manifold $N$ of dimension $n$,
Dunajski and Mettler \cite{DM} construct a neutral signature Einstein metric $g$ with non-zero
scalar curvature on a certain rank $n$ affine bundle $M\rightarrow N$.
The $2n$--dimensional space $M$ also carries a natural symplectic form $\Omega$, and
an endomorphism $J:TM\rightarrow TM$ which is such that $J^2$ is the identity and $g(\cdot\ ,\cdot)=\Omega(\cdot\ ,J\cdot)$. This makes $(M,g,\Omega)$ a so--called \textit{almost para--K\"ahler} structure. It is interesting for a number of different reasons.

Firstly, $(M,g)$ is interesting by virtue of being an Einstein
space. In fact, it turns out that $g$ arises as the Kaluza--Klein reduction of an Einstein metric $\mathcal{G}$ on an $\R^*$ bundle $\sigma:\mathcal{Q}\rightarrow M$ which has curvature form $\sigma^*(\Omega)$. In chapter \ref{chap:KK_lift} we will construct $\mathcal{G}$ explicitly, and give an interpretation of the manifold $(\mathcal{Q},\mathcal{G})$ in terms of the projective geometry on $N$.

It also turns out that $(M,g,\Omega)$ can be thought of as compactifiable in a certain sense. Recall that a (psuedo--)Riemannian manifold $(M,g)$ is said to be \textit{conformally compact} if there is a smooth positive function $T$ such that $T^2g$ smoothly extends to a manifold with boundary $\ov{M}=M\cup\p M$, and the set $\{p\in\ov{M}:T(p)=0\}$ is a hypersurface which coincides with the boundary $\p M$.  This is a useful concept because $(M,T^2g)$ has the same conformal structure, and hence the same \textit{causual} structure, as $(M,g)$. It has been used to study said causual structure in both general relativity \cite{penrose65} and quantum field theory \cite{witten}. It is also useful for formulating the boundary conditions of conformally invariant field equations such as those arising in Yang--Mills theory \cite{uhlen}.

A mathematical summary of conformal compactification is that some weakening of the (psuedo--)Riemannian geometry on $M$ extends to a manifold with boundary $\ov{M}$. Recent work by \v Cap and Gover \cite{CG0,CG} has generalised this idea to other geometrical structures which admit some weakening which extends to a manifold with boundary. In particular, on an almost complex manifold $(M,J)$ with complex connection $\nabla$, one can define the $c$--projective equivalence class $[\nabla]$ to which $\nabla$ belongs, and show that the $c$--projective structure $(M,J,[\nabla])$ extends to a manifold with boundary $\ov{M}$ \cite{CG}. The main goal of chapter \ref{chap:c-proj} is to adapt the work of \cite{CG} to the para--$c$--projective case, and to show that the almost complex structure $J$ on $M$ has a complex connection which admits a so--called para--$c$--projective compactification. The result of this is that the manifolds $(M,g,\Omega)$ can be thought of as para--$c$--projectively compact.

Another reason this construction is interesting is that for $n=2$ (so that $M$ has dimension $4$), the conformal
curvature of $g$ is anti-self-dual. Recall that the Hodge operator
$\star$ defined by a Euclidean or neutral signature metric in four
dimensions is an involution on two--forms (i.e. squares to the identity).
It thus has eigenvalues $\pm1$, and the space of two-forms splits
into the corresponding eigenspaces, which are referred to as self--dual
(SD) or anti--self--dual (ASD) respectively. Due to its index symmetries,
the Weyl tensor can be thought of as a map from two-forms to two-forms,
and therefore has a corresponding decomposition. Since the Weyl tensor
encodes the conformal curvature, we say that a conformal or (psuedo--)Riemannian
manifold whose Weyl tensor is ASD is equipped with an \textit{ASD
conformal structure}.

The field equations corresponding to anti-self-duality of the Weyl
tensor in four dimensions can be solved by a twistor construction,
and are thus \textit{integrable} \cite{ward}. This means that any systems of differential
equations which can be obtained from them by symmetry reduction should
also be integrable (see \cite{MW} for a review). In particular, the class of dispersionless
integrable systems in 2+1 and 3 dimensions arise in this way. The
construction \cite{DM} provides some examples of
ASD conformal structures in neutral signature which, in the presence
of a (non--null) symmetry, give rise to solutions of an integrable
system called the $SU(\infty)$--Toda field equation via $2+1$--dimensional
Einstein-Weyl structures. In chapter \ref{chap:EW_and_toda} we discuss the extraction
of all possible Toda solutions obtainable in this way.






\section{The Cartan Geometry of a Projective Structure}

One way of understanding the construction of $(M,g)$ in \cite{DM}
is via the Cartan bundle of the projective structure $(N,[\nabla])$.
By virtue of being modelled on $\mathbb{RP}^{n},$ which can be viewed
as a homogeneous space, projective structures admit a description
as Cartan geometries. These generalise Klein's Erlangen programme,
a study of homogeneous spaces $G/H$, to the curved case, in which
the total space $G$ is replaced by a principal right $H$-bundle
over a manifold $N$ such that the tangent space to $N$ is isomorphic
to the Lie algebra quotient $\mathfrak{g}/\mathfrak{h}$. In the Riemannian
case, this corresponds to generalising $\mathbb{R}^{n}\cong\mathrm{Euc}(n)/SO(n)$
to a general, curved Riemannian manifold, whose orthonormal frame
bundle is a principal $SO(n)$ bundle and whose tangent spaces are
modelled on $\mathbb{R}^{n}\cong\mathfrak{Euc}(n)/\mathfrak{so}(n)$.

The theory of Cartan geometries was developed as part of Cartan's
\textit{method of moving frames}. In Riemannian geometry, one has
an obvious subclass of frames which are ``adapted'' to the metric,
i.e. those which are orthonormal. The idea of Cartan's method is to
pick out some adapted frames for manifolds equipped with some non-metric
structure. The bundle of such frames over a manifold is then a principal
bundle $\pi:P\rightarrow S$ with structure group $H$.

In addition, $P$ is equipped with a $\mathfrak{g}$-valued one-form
$\theta$ called the Cartan connection. It satisfies a number of properties,
in particular equivariance, i.e. $R_{h}^{*}\theta=\mathrm{Ad}(h^{-1})\theta$
for all $h\in H$. It also defines an isomorphism $\theta:T_{p}P\rightarrow\mathfrak{g}$
every point $p\in P$ such that the vertical subspace $V_{p}P\subset T_{p}P$
is mapped to $\mathfrak{h}$ and the horizontal subpace $H_{p}P\subset T_{p}P$
is defined as the inverse image of $\mathfrak{g}/\mathfrak{h}$. Note
that it is not a connection in the usual sense of a principal bundle
connection, since it takes value in a Lie algebra larger than that
of the structure group. Further details can be found in \cite{Sharpe1997}.

In the case of a projective surface, the model is $\mathbb{RP}^{n}\cong SL(n+1,\mathbb{R})/H$,
where $SL(n+1,\mathbb{R})$ acts via the fundamental representation
on homogenous coordinates $[X_1,X_2,\dots,X_n]$ in $\mathbb{RP}^{n}$, and $H$
is the stabiliser subgroup of the point $[1,0,\dots,0]$, whose elements
are matrices of the general form
\[
\begin{pmatrix}\mathrm{det}a^{-1} & b\\
0 & a
\end{pmatrix}
\]
for some $a\in GL(n,\mathbb{R})$ and $b\in\mathbb{R}_{n}$. We say
that the Cartan geometry of a projective surface is of type $(SL(n+1,\mathbb{R}),H)$.
We call it $(\pi:P\rightarrow N,\theta)$, where $\theta$ is the
Cartan connection and takes values in $\mathfrak{sl}(n+1,\mathbb{R})$.
It can be written as a matrix
\[
\theta=\begin{pmatrix}-\mathrm{tr}\phi & \eta\\
\omega & \phi
\end{pmatrix},
\]
where $\omega$, $\eta$ and $\phi$ are one-forms valued in $\mathbb{R}^{n}$,
$\mathbb{R}_{n}$ and $\mathfrak{gl}(n,\mathbb{R})$ respectively.
We will refer to the components of $\omega$ and $\eta$ with respect
to the natural basis of $\mathfrak{sl}(n+1,\mathbb{R})$ as $\{\omega^{(i)}\}$
and $\{\eta_{(i)}\}$, so that $\omega^{(i)}$ and $\eta_{(i)}$ are
both one-forms. We also note for later use that the $\mathfrak{sl}(n+1,\mathbb{R})$-valued
curvature two-form $\Theta$ satisfies
\[
\Theta=d\theta+\theta\wedge\theta=\begin{pmatrix}0 & L(\omega\wedge\omega)\\
0 & W(\omega\wedge\omega)
\end{pmatrix},
\]
where $L$ and $W$ are smooth curvature functions.


\section{The Dunajski--Mettler Construction $(M,g_\Lambda,\Omega_\Lambda)$}

We can think of $M$ as a quotient of the total space $P$ of the
Cartan geometry by $GL(n,\mathbb{R})$, which is embedded in $H$
in the obvious way:
\[
GL(n,\mathbb{R})\ni a\longmapsto\begin{pmatrix}\mathrm{det}a^{-1} & 0\\
0 & a
\end{pmatrix}\in H.
\]
It is easily verified that 
\[
\begin{pmatrix}\mathrm{det}a^{-1} & 0\\
0 & a
\end{pmatrix}\begin{pmatrix}0 & \eta\\
\omega & 0
\end{pmatrix}\begin{pmatrix}\mathrm{det}a^{-1} & 0\\
0 & a
\end{pmatrix}^{-1}=\begin{pmatrix}0 & \eta a^{-1}\mathrm{det}a^{-1}\\
(\mathrm{det}a)a\omega & 0
\end{pmatrix},
\]
for any $a\in GL(n,\mathbb{R})$, meaning that due to the equivarience
property of the Cartan connection, the natural contraction $\eta\omega:=\sum_{i}\eta_{(i)}\otimes\omega^{(i)}$
defined by $\theta$ is presevered by the adjoint action of this $GL(n,\mathbb{R})$
subgroup. It thus descends to a naturally defined object on the quotient
$M=P/GL(n,\mathbb{R})$.

\begin{theo}{\cite{DM}}\label{thm:DM}
There exist a metric and two-form
$(g,\Omega)$ on $\mbox{\ensuremath{M=P/GL(n,\mathbb{R})}}$ such
that the quotient map $q:P\rightarrow M$ gives
\begin{eqnarray*}
q^{*}g & = & \mathrm{Sym}(\eta\omega)\\
q^{*}\Omega & = & \mathrm{Ant}(\eta\omega),
\end{eqnarray*}
where $\mathrm{Sym}$ and $\mathrm{Ant}$ denote the symmetric and
anti-symmetric parts of the $(0,2)$ tensor $\eta\omega$. Moreover,
$\Omega$ is closed as a consequence of the Bianchi identity, $g$
is Einstein with non-zero scalar curvature, and the two are related
by an endomorphism $J$ satisying $J^{2}=\mathrm{id}$. Hence $(g,\Omega)$
is an almost para-Kahler structure on $M$.
\end{theo}

Note that the full proof of theorem \ref{thm:DM} only appears explicitly in \cite{DM} in the case $n=2$, although it can be generalised to $n>2$. This generalisation is discussed in their appendix.

The quotient $M$ turns out to be an affine bundle over $N$ with
structure group $H$, i.e. $H$ acts affinely on the fibers of $\rho:M\rightarrow N$,
and sections of this bundle are in one-to-one correspondence with
representative connections $\nabla\in[\nabla]$. This means that given
some choice of connection $\nabla\in[\nabla]$ we have a diffeomorphism
$\varphi:T^{*}N\rightarrow M$ with which we can pull back the pair
$(g,\Omega)$. In canonical local coordinates $(x^{i},p_{i})$ on
the cotangent bundle, we find

\begin{eqnarray}
\varphi^{*}g & = &  dp_{i}\odot dx^{i}-(\Gamma_{ij}^{k}p_{k}-p_{i}p_{j}-\Rho_{ij}) dx^{i}\odot dx^{j},\label{eq:coord_form}\\
\varphi^{*}\Omega & = &  dp_{i}\wedge dx^{i}+\Rho_{ij} dx^{i}\wedge dx^{j},\qquad i,j=1,\dots ,n.\nonumber 
\end{eqnarray}
Here $\Gamma_{jk}^{i}$ are the connection components of the representative
connection $\nabla$ that we chose, and its Schouten tensor%
\footnote{Recall that the Schouten tensor is given in terms of the Ricci tensor
$R_{ij}$ by $P_{ij}=\frac{1}{n-1}R_{(ij)}+\frac{1}{n+1}R_{[ij]}$
where $n$ is the dimension of the manifold.%
} is denoted $P_{ij}$. This can be shown to be projectively invariant
in the sense that a different choice of $\nabla\in[\nabla]$ corresponds
to shifting the fiber coordinates $p_{i}$, i.e. metrics on $T^{*}N$
resulting from pulling back $g$ using different representative connections
are isometric. Explicitly, a projective transformation (\ref{eq:proj_change})
corresponds to a change
\begin{equation}
\Rho_{ij}\longrightarrow \Rho_{ij}+\Upsilon_{i}\Upsilon_{j}-\nabla_{i}\Upsilon_{j},\qquad p_{i}\longrightarrow p_{i}+\Upsilon_{i}.\label{eq:p_change}
\end{equation}


In fact, the metric and symplectic form (\ref{eq:coord_form}) turn
out to belong to a one-parameter family $\{g_{\Lambda}\}$, which
can be written in local coordinates as 
\begin{eqnarray}
g_{\Lambda} & = &  dp_{i}\odot dx^{i}-(\Gamma_{ij}^{k}p_{k}-\Lambda p_{i}p_{j}-\Lambda^{-1}\Rho_{ij}) dx^{i}\odot dx^{j}\label{eq:general_g}\\
\Omega_{\Lambda} & = &  dp_{i}\wedge dx^{i}+\frac{1}{\Lambda}\Rho_{ij} dx^{i}\wedge dx^{j},\qquad i,j=1,\dots ,n.\label{eq:general_omega}
\end{eqnarray}
%Metrics of the form (\ref{eq:general_g}) are a subclass of so-called
%Osserman metrics. More details can be found in \cite{No-louzao1991}.
They are all Einstein and, for $n=2$, all ASD, but for $\Lambda\neq1$ the relation to projective geometry is lost. For the remainder of the thesis we will write $g$ for $g_{\Lambda=1}$ unless stated otherwise. Note that $\{g_\Lambda\}$ will be the subject of chapter \ref{chap:KK_lift}, whilst in chapters \ref{chap:c-proj} and \ref{chap:EW_and_toda} we will restrict our attention to $g$ because the projective geometry is a key aspect of the content of these chapters.


\subsection{Symmetries of $(M,g_{\Lambda},\Omega_{\Lambda})$}

Recall that a projective vector field on any manifold with a connection
generates a 1-parameter family of transformations which preserve the
geodesics of that connection up to parametrisation. Projective vectors
fields thus naturally arise as the symmetries of a projective structure.
Explicitly, a vector field $K$ is projective if it satisfies
\begin{equation}
\mathcal{L}_{K}\Gamma_{ij}^{k}=\delta_{i}^{k}\Upsilon_{j}+\delta_{j}^{k}\Upsilon_{i}\label{eq:proj_transf}
\end{equation}
for some 1-form $\Upsilon$, where $\Gamma_{ij}^{k}$ are the connection
components, and their Lie derivative is defined (see \cite{Yano1955})
by
\begin{equation}
\mathcal{L}_{K}\Gamma_{ij}^{k}\equiv\frac{\partial^{2}K^{k}}{\partial x^{i}\partial x^{j}}+K^{m}\frac{\partial\Gamma_{ij}^{k}}{\partial x^{m}}-\Gamma_{ij}^{m}\frac{\partial K^{k}}{\partial x^{m}}+\Gamma_{im}^{k}\frac{\partial K^{m}}{\partial x^{j}}+\Gamma_{mj}^{k}\frac{\partial K^{m}}{\partial x^{i}}.\label{eq:liederivGamma}
\end{equation}


One consequence of the above theorem is that for every open set $\mathcal{U}\subset N$
we have an isomorphism between the Lie algebra of projective vector
fields on $\mathcal{U}$ and the Lie algebra of vector fields on $\rho^{-1}(\mathcal{U})$
which preserve both $g$ and $\Omega$. This follows from a standard
result about Cartan geometries: projective vector fields on $\mathcal{U}$
are in one-to-one correspondence with vector fields on $\pi^{-1}(\mathcal{U})$
which preserve $\theta$ and are equivariant under the principal $H$-action.
Such vector fields thus preserve the natural contraction $\eta\omega$
and must descend to vector fields on $M$ preserving $(g,\Omega)$.
In fact, it can be shown that every Killing vector field of $(M,g)$
is also symplectic with respect to $\Omega$ and is therefore the
lift of a projective vector field on $(N,[\nabla])$.

Explicitly, for every projective vector field $K$ of $(N,[\nabla])$
there is a corresponding Killing vector $\mathcal{K}$ of $(M,g_{\Lambda})$
given in local coordinates by 
\begin{equation}
\mathcal{K}=K-p_{i}\frac{\partial K^{j}}{\partial x^{i}}\frac{\partial}{\partial p_{j}}+\frac{1}{\Lambda}\Upsilon_{i}\frac{\partial}{\partial p_{i}},\label{eq:kvf_from_pvf}
\end{equation}
where $\Upsilon_{i}$ is defined by (\ref{eq:proj_transf}).

\subsection{Anti--Self--Duality for $n=2$}
For $n=2$, a local of characterisation of the sapces $M$ is provided in
\cite{DM}: they show that any 4-dimensional anti-self-dual
Einstein space with scalar curvature $-24\Lambda$ and a parallel
anti-self-dual totally null distribution can be considered as the
total space of a rank 2 affine bundle $T^{*}S$ over a projective
surface $S$ of the form (\ref{eq:coord_form}). The anti--self-duality property, in combination with the correspondence of symmetries of $(M,g)$ with symmetries of $(N,[\nabla])$, is important in the context of the applications of
the work \cite{DM} to integrability. It means that
if we start with a projective surface with at least one projective
vector field, we will find an ASD Einstein space with at least one
Killing vector field, and thus will be able to perform a symmetry
reduction to obtain an Einstein-Weyl structure in $2+1$ dimensions
and a corresponding solution to the $SU(\infty)$-Toda field equation. This is the subject of chapter \ref{chap:EW_and_toda}.
