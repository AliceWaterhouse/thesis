%!TEX root = ../thesis.tex
%*******************************************************************************
%****************************** Introduction *********************************
%*******************************************************************************


\chapter{Introduction}\label{chap:intro}

%The relationship of geometry to non--linear PDEs is close and has long been known.
%\begin{itemize}
%%\item Can define objects on a manifold satisfying certain properties expressed in terms of PDEs in the coordinates. e.g. GR. Generalises to projective and conformal geometry. (chapters \ref{chap:KK_lift} and \ref{chap:c-proj})
%\item Results from integrable systems reveal unexpected examples of PDEs which can be expressed in terms of geometrical conditions. (Ward). (chapter \ref{chap:EW_and_toda})
%\item Geometry also important in the theory of solitons (e.g. Manton and Sutcliffe.) (chapter \ref{chap:wormhole}).
%\end{itemize} 

This thesis splits naturally into two parts. The first 4 chapters are concerned with a class of Einstein manifolds which can be canonically constructed from projective structures as shown in recent work by Dunajski and Mettler \cite{DM}. Chapter \ref{chap:wormhole} discusses a solitonic solution of the equations of motion associated with $\phi^4$--field theory on a wormhole spacetime. Chapter \ref{chap:wormhole} is self--contained, and possesses its own introduction and notation. The remainder of this chapter will introduce some preliminaries for the first part of the thesis, culminating in the results of \cite{DM} on the almost para--K\"ahler Einstein manifolds which will form the basis of chapters 2--4.

\mynote{Projective geometry, based on geodesics as unparametrised curves, initiated in ... by ...}

\mynote{General point - should be more than in abstract but not repeating the actual intro below.}

Given a projective structure on a manifold $N$ of dimension $n$,
Dunajski and Mettler \cite{DM} construct a neutral signature Einstein metric $g$ with non-zero
scalar curvature on a certain rank $n$ affine bundle $M\rightarrow N$.
The $2n$--dimensional space $M$ also carries a natural symplectic form $\Omega$, and an endomorphism $J:TM\rightarrow TM$ which is such that $J^2$ is the identity and $g(J\cdot\,,\cdot)=\Omega(\cdot\,,\cdot)$. This makes $(M,g,\Omega)$ a so--called \textit{almost para--K\"ahler} structure. It is interesting for a number of different reasons.

\mynote{It expresses the second order object which is the projective structure as a first order object on a larger space. See if you can find the slides of the talk Thomas gave in damtp where he talked about this? If I put in Thomas' explicit construction of $g$ (i.e. define the second order coframe bundle) this point would be more relevant.}

Firstly, $(M,g)$ is interesting by virtue of being an Einstein
space. In fact, it turns out that $g$ arises as the Kaluza--Klein reduction of an Einstein metric $\mathcal{G}$ on an $\R^*$ bundle $\sigma:\mathcal{Q}\rightarrow M$ which has curvature form $\sigma^*(\Omega)$. In chapter \ref{chap:KK_lift} we will construct $\mathcal{G}$ explicitly, and give an interpretation of the manifold $(\mathcal{Q},\mathcal{G})$ in terms of the projective geometry on $N$.

It also turns out that $(M,g,\Omega)$ can be thought of as compactifiable in a certain sense. Recall that a (psuedo--)Riemannian manifold $(M,g)$ is said to be \textit{conformally compact} if there is a smooth positive function $T$ such that $T^2g$ smoothly extends to a manifold with boundary $\ov{M}=M\cup\p M$, and the set $\{p\in\ov{M}:T(p)=0\}$ is a hypersurface which coincides with the boundary $\p M$.  This is a useful concept because $(M,T^2g)$ has the same conformal structure, and hence the same \textit{causal} structure, as $(M,g)$. It has been used to study said causal structure in both general relativity \cite{penrose65} and quantum field theory \cite{witten}. It is also useful for formulating the boundary conditions of conformally invariant field equations such as those arising in Yang--Mills theory \cite{uhlen}.

Recent work by \v Cap and Gover \cite{CG0,CG} has generalised this idea to other geometrical structures which admit some weakening which extends to a manifold with boundary. In particular, on an almost complex manifold $(M,J)$ with complex connection $\nabla$, one can define the $c$--projective equivalence class $[\nabla]$ to which $\nabla$ belongs, and show that the $c$--projective structure $(M,J,[\nabla])$ extends to a manifold with boundary $\ov{M}$ \cite{CG}. The main goal of chapter \ref{chap:c-proj} is to adapt the work of \cite{CG} to the para--$c$--projective case, and to show that the almost complex structure $J$ on $M$ has a complex connection which admits a so--called para--$c$--projective compactification. The result of this is that the manifolds $(M,g,\Omega)$ can be thought of as para--$c$--projectively compact.

Another reason this construction is interesting is that for $n=2$ (so that $M$ has dimension $4$), the conformal
curvature of $g$ is anti-self-dual. Recall that the Hodge operator
$\star$ defined by a Euclidean or neutral signature metric in four
dimensions is an involution on two--forms (i.e. squares to the identity).
It thus has eigenvalues $\pm1$, and the space of two-forms splits
into the corresponding eigenspaces, which are referred to as self--dual
(SD) or anti--self--dual (ASD) respectively. Due to its index symmetries,
the Weyl tensor can be thought of as a map from two-forms to two-forms,
and therefore has a corresponding decomposition. Since the Weyl tensor
encodes the conformal curvature, we say that a conformal or (psuedo--)Riemannian
manifold whose Weyl tensor is ASD is equipped with an \textit{ASD
conformal structure}.

The field equations corresponding to anti-self-duality of the Weyl
tensor in four dimensions can be solved by a twistor construction,
and are thus \textit{integrable} \cite{ward}. This means that any systems of differential
equations which can be obtained from them by symmetry reduction should
also be integrable (see \cite{MW} for a review). In particular, the class of dispersionless
integrable systems in 2+1 and 3 dimensions arise in this way. The
construction \cite{DM} provides some examples of
ASD conformal structures in neutral signature which, in the presence
of a (non--null) symmetry, give rise to solutions of an integrable
system called the $SU(\infty)$--Toda field equation via $2+1$--dimensional
Einstein-Weyl structures. In chapter \ref{chap:EW_and_toda} we discuss the extraction
of all possible Toda solutions obtainable in this way.

\mynote{some notation and conventions e.g. $i,j,k=1,...,n$ are indices for coordinates on $n$? and $\Gamma$ are connection components which are defined by... Curvature is defined by}
\begin{itemize}
\item We denote the bundle of anti--symmetrised covariant tensors of degree $m$ as $\Lambda^m$, and call sections of this bundle $m$--forms.
\item Note that our conventions are $(d\omega)_{ab\dots c}=\partial_{[a}\omega_{b\dots c]},$
$(\eta\wedge\omega)_{a\dots d}=\eta_{[a\dots b}\omega_{c\dots d]},$
$\omega=\omega_{a\dots b}dx^{a}\wedge\dots\wedge dx^{b},$
and $F_{ab}{d}x^{a}\wedge{d}x^{b}=F_{[ab]}{d}x^{a}\otimes{d}x^{b}$
implying ${d}x^{a}\wedge{d}x^{b}=\frac{1}{2}({d}x^{a}\otimes{d}x^{b}-{d}x^{b}\otimes{d}x^{a})$.
\item $N$ is oriented. \mynote{something about which part of $SL(n+1)$ we mean?}
\end{itemize}

\section{Projective Geometry}\label{sec:projgeom}

Our discussion follows Eastwood \cite{Eastwood}.

\begin{defi}\label{def:projstruct} A projective structure $(N,[\nabla])$
on a manifold $N$ is an equivalence class $[\nabla]$ of torsion--free affine connections on $N$ which have the same unparametrised geodesics.
\end{defi}

The following proposition converts definition \ref{def:projstruct} to a to a more operational form.

\begin{prop} Two torsion--free connections $\nabla$ and $\ov{\nabla}$ belong to the same projective class if and only if their components $\Gamma^i_{jk}$ and $\ov{\Gamma}^i_{jk}$ are related by
\begin{equation}
\ov{\Gamma}^i_{jk} - \Gamma^i_{jk} = \delta_{j}^{i}\Upsilon_{k}+\delta_{k}^{i}\Upsilon_{j}\label{eq:proj_change}
\end{equation}
for some one--form $\Upsilon.$
\end{prop}

{\bf Proof.} We denote by $V$ the vertical sub--bundle of $T(TN)$, where $\nu:TN\rightarrow N$ is the tangent bundle to $N$. A connection defines a splitting of the exact sequence
\be \label{eq:TTMsequence}
0 \longrightarrow V \longrightarrow T(TN) \longrightarrow \nu^*TN \longrightarrow 0
\ee
so that each $X^i\in T_pM$ has a unique pull--back in the horizontal sub--bundle complementary to $V$. The integral curves of these pull--backs, when projected down to $N$, then define the geodesics of the connection.

Any two connections are related by some $\delta\Gamma^k_{ij}$, which satisfies $\delta\Gamma^k_{ij}=\delta\Gamma^k_{(ij)}$ as long as both connections are torsion--free. A change of connection is equivalent to a change in the splitting of (\ref{eq:TTMsequence}). At $X^i\in T_pN$, the change is given by the homomorphism from $T_pN$ to $T_pN=V_p$ defined by the contraction $X^i\Gamma^k_{ij}$. Thus the two connections define the same geodesics if and only if $X^iX^j\Gamma^k_{ij}$ is a multiple of $X^k$ for all $X^i$. This is true if and only if there is a one--form $\Upsilon_i$ such that (\ref{eq:proj_change}) is satisfied.\footnote{To see this, take some one--form $\omega_i$ and note that $2X^iX^j\delta^k_{(i}\Upsilon_{j)}\omega_k$ vanishes if and only if $X^k\omega_k$ does.}
\koniec

One can show that the curvature of a connection $\nabla$ in the projective class can be uniquely decomposed as
\be \label{eq:projcurvdecomp}
R_{ijk}^{\ \ \ l} = W_{ijk}^{\ \ \ l} + 2\delta^l_{[i}\Rho_{j]k} -2\Rho_{[ij]}\delta^l_k,
\ee
where the Weyl projective curvature tensor, $W_{ijk}^{\ \ \ l}$, is trace free, and the Schouten tensor, $\Rho_{ij}$, is given in terms of the Ricci tensor by
\[
\Rho_{ij}=\frac{1}{n-1}R_{(ij)}+\frac{1}{n+1}R_{[ij]}.
\]
The objects $W_{ijk}^{\ \ \ l}$ and $\Rho_{ij}$ transform as
\be \label{eq:schout_change}
\ov{W}_{ijk}^{\ \ \ l} = W_{ijk}^{\ \ \ l}, \qquad \ov{\Rho}_{ij} = \Rho_{ij} - \nabla_i\Upsilon_j + \Upsilon_i\Upsilon_j
\ee
under a change of representative connection (\ref{eq:proj_change}). Note that for $n=2$ the Weyl tensor is always vanishing.

A projective structure in dimension $n$ is said to be flat if it is diffeomorphic to the real projective space $\RP^n$ with its standard flat projective structure.
\begin{defi}
The real projective space $\RP^n$ of dimension $n$ is the space of unoriented lines through the origin in $\R^{n+1}$, thought of as $\RP^n=(\R^{n+1}\backslash\{0\})/\R^*$, with geodesics given by straight lines in $\R^{n+1}$ under the projection $\kappa:\R^{n+1}\rightarrow\RP^n$.
\end{defi}
Let $X$ denote a non--zero point in $\R^{n+1}$ with coordinates $(X^0,X^1,\dots,X^n)^T$, and let $[X]$ denote the corresponding point in $\RP^n$, labelled by homogeneous coordinates. In a patch where $X^0\neq 0$, we can write $[X]=[1,X^1/X^0,\dots,X^n/X^0]$ and define inhomogeneous coordinates on $\RP^n$ by
\[
(x^1,\dots,x^n) = (X^1/X^0,\dots,X^n/X^0).
\]
If we combine this with coordinate patches where $X^i\neq 0,\ i=1,\dots,n$, we can build an atlas for $\RP^n$.

Real projective space can be viewed as homogeneous space as follows. 
The group $SL(n+1,\mathbb{R})$ acts from the left via the fundamental representation on coordinates $(X^0,\dots,X^n)^T$ in $\R^{n+1}$, and this descends to a transitive action on $\RP^n$. By the orbit stabiliser theorem, $\RP^n=SL(n+1,\R)/H$, where $H$ is a subgroup stabilising a point. If we choose the point $[1,0,\dots,0]$, the elements of $H$ are matrices of the general form
\[
\begin{pmatrix}\mathrm{det}a^{-1} & b\\
0 & a
\end{pmatrix}
\]
for some $a\in GL(n,\mathbb{R})$ and $b\in\mathbb{R}_{n}$. 

It can be shown that a projective structure in dimension $n>2$ is flat if and only if its Weyl projective curvature tensor vanishes. The necessary and sufficient condition in dimension $n=2$ is the vanishing of the Cotton tensor $\nabla_{[i}\Rho_{j]k}$ for any choice of representative connection.

\subsection{The Cartan bundle}

One way of understanding the construction of $(M,g)$ in \cite{DM}
is via the Cartan bundle \cite{Cartan} of the projective structure $(N,[\nabla])$ (see also \cite{KobNag}). Cartan geometries generalise Klein's Erlangen programme \cite{Klein}, a study of homogeneous spaces $G/H$, to the curved case, in which the total space $G$ is replaced by a principal right $H$-bundle over a manifold $N$ such that the tangent space to $N$ at every point is isomorphic to the Lie algebra quotient $\mathfrak{g}/\mathfrak{h}$. Since projective structures are modelled on $\mathbb{RP}^{n}$, which can be viewed as a homogeneous space, they constitute a type of Cartan geometry.

In the Riemannian case, the model space is $\mathbb{R}^{n}\cong\mathrm{Euc}(n)/SO(n)$. The corresponding Cartan geometry is a general, curved Riemannian manifold. One has an obvious subclass of frames which are ``adapted'' to the metric, i.e. those which are orthonormal. We can thus think of a curved Riemannian manifold as a principal $SO(n)$ bundle whose tangent spaces are modelled on $\mathbb{R}^{n}\cong\mathfrak{Euc}(n)/\mathfrak{so}(n)$. We say that Riemannian manifolds are Cartan geometries of type $(\mathrm{Euc}(n),SO(n))$.

The theory of Cartan geometries was developed as part of Cartan's
\textit{method of moving frames}. The idea is to pick out some adapted frames for manifolds equipped with some non-metric structure. The bundle of such frames over a manifold is then a principal bundle $\pi:P\rightarrow N$ with structure group $H$. In the projective case, the notion of a \textit{second order frame} must be introduced to obtain an object which is correctly adapted to the projective structure.

The bundle $P$ is equipped with a $\mathfrak{g}$-valued one-form
$\theta$ called the Cartan connection. It defines an isomorphism $\theta:T_{p}P\rightarrow\mathfrak{g}$ at every point $p\in P$ such that the vertical subspace $V_{p}P\subset T_{p}P$ is mapped to $\mathfrak{h}$ and the horizontal subpace $H_{p}P\subset T_{p}P$ is defined as the inverse image of $\mathfrak{g}/\mathfrak{h}$. Note that it is not a connection in the usual sense of a principal bundle connection, since it takes value in a Lie algebra larger than that of the structure group. Further details can be found in \cite{Sharpe}.

%The importance of the Cartan connection is that it satisfies a number of properties, in particular equivariance, i.e. $R_{h}^{*}\theta=\mathrm{Ad}(h^{-1})\theta$ for all $h\in H$. 

In the projective case, if we choose the point which is stabilised by $H$ to be $[1,0,\dots,0]$, the Cartan connection can be written as a matrix
\be \label{eq:cartan_connection}
\theta=\begin{pmatrix}-\mathrm{tr}\phi & \eta\\
\omega & \phi
\end{pmatrix},
\ee
where $\omega$, $\eta$ and $\phi$ are one-forms valued in $\mathbb{R}^{n}$, $\mathbb{R}_{n}$ and $\mathfrak{gl}(n,\mathbb{R})$ respectively.
We will refer to the components of $\omega$ and $\eta$ with respect
to the natural basis of $\mathfrak{sl}(n+1,\mathbb{R})$ as $\{\omega^{(i)}\}$ and $\{\eta_{(i)}\}$, so that $\omega^{(i)}$ and $\eta_{(i)}$ are both one-forms.

\begin{defi}
The Cartan geometry of a projective structure $(N,[\nabla])$ consists of a principal right $H$--bundle $\pi_P:P\rightarrow N$, where the right--action of some $h\in H$ on $P$ is denoted by $R_h$, carrying a one--form $\theta$ called the Cartan connection, which takes values in $\mathfrak{sl}(n+1,\mathbb{R})$, can be written in the form (\ref{eq:cartan_connection}) and has the following properties:
\begin{enumerate}
\item $\theta_u:T_uP\rightarrow\mathfrak{sl}(n+1,\R)$ is an isomorphism for all $u\in P$;
\item $\theta(X_v)=v$ for all fundamental vector fields $X_v$ on $P$;
\item $R^*_h\theta = \mathrm{Ad}(h^{-1})\theta=h^{-1}\theta h$ for all $h\in H$.
\item If $X$ is a vector field on $P$ with the property that $\eta(X)=\phi(X)=0$ and $\omega(X)\in\R^n\backslash\{0\}$, then the integral curve of $X$ projects down to a geodesic on $N$ and conversely every geodesic of $[\nabla]$ arises in this way.
\item The $\mathfrak{sl}(n+1,\mathbb{R})$-valued
curvature two-form $\Theta$ satisfies
\be \label{eq:curvature_2-form}
\Theta=d\theta+\theta\wedge\theta=\begin{pmatrix}0 & L(\omega\wedge\omega)\\
0 & W(\omega\wedge\omega)
\end{pmatrix},
\ee
where $L$ and $W$ are smooth curvature functions valued in $\mathrm{Hom}(\R^n\wedge\R^n,R_n)$ and $\mathrm{Hom}(\R^n\wedge\R^n,R_n\otimes\R^n)$ respectively. The function $W$ represents the Weyl projective curvature tensor appearing in (\ref{eq:projcurvdecomp}).
\end{enumerate}
\end{defi}

\begin{rmk}The Cartan geometry of a projective structure is unique in the sense that for any two Cartan geometries $(\widehat{\pi}_P:\widehat{P}\rightarrow N,\widehat{\theta})$ and $(\pi_P:P\rightarrow N,\theta)$ of type $(SL(n+1,\R),H)$ satisfying the above properties there is a $H$--bundle isomorphism $\psi:P\rightarrow\widehat{P}$ such that $\psi^*\widehat{\theta}=\theta$. %This means that although we cannot choose a unique connection on the tangent bundle to $N$, we can choose a unique connection on the Cartan bundle.
\end{rmk}

\begin{rmk} \label{rmk:theta_symmetry}
For every open set $\mathcal{U}\subset N$, projective vector fields on $\mathcal{U}$ are in one-to-one correspondence with vector fields on $\pi^{-1}(\mathcal{U})$
which preserve $\theta$ and are equivariant \mynote{what do I mean by this} under the principal $H$-action.
\end{rmk}


\begin{rmk}
The Cartan connection also gives us a unique connection on any bundle associated to $P$ via some $H$--module. In particular, let $E$ be a vector space and $\rho:H\rightarrow GL(E)$ a representation of $H$ acting on $E$. We can construct an \textit{associated bundle}
\[\pi_E:P\times_\rho E\rightarrow N \]
where points in $P\times_\rho E$ are equivalence classes of pairs $[u,v]$, where $u\in P$ and $v\in E$, up to the equivalence relation
\[
(u_1,v_1)\sim (u_2,v_2) \quad \Leftrightarrow \quad \exists\ h\ \mbox{such that}\  u_2=u_1 h,\  v_2 = \rho(h^{-1}) v_1.
\]

We thus obtain a vector bundle over $N$ whose fibers are diffeomorphic to $E$. A section $s:N\rightarrow P\times_\rho E$ is represented by a map $\sigma_s:P\rightarrow E$ which is \textit{equivariant} in the sense that $ \sigma_s(uh)=\rho(h^{-1})\sigma_s(u)$ for all $h\in H$. Importantly, any such bundle inherits a connection from the Cartan connection $\theta$ on $P$.
\end{rmk}


\subsection{The cotractor bundle}
A particularly important example of a vector bundle associated to $P$ is the \textit{cotractor bundle}, which defined by the canonical right action of $H$ on $\R_{n+1}$ given by $(h,v)\mapsto vh^{-1}$. We call this bundle $\pi:\cT^*\rightarrow N$. In order to describe its connection, we consider a section represented by $\sigma:P\rightarrow\R_{n+1}$ and define the one--form
\be \label{eq:df-ftheta}
d\sigma - \sigma\theta.
\ee
This turns out to be a \textit{semi--basic}\footnote{Recall that a semi--basic form on a fiber bundle $P\rightarrow N$ is a form which is a linear combination, with coefficients parametrised by the fibers, of basic forms on $P$ (i.e. forms which are the pull-backs of forms on $N$).} one--form satisfying
\[
R_h^*(d\sigma-\sigma\theta) = (d\sigma - \sigma\theta)\rho(h^{-1}),
\]
making $\sigma\mapsto d\sigma-\sigma\theta$ an equivariant connection on $\cT^*$.

Although this construction of $\mathcal{T}^*$ relies on the Cartan bundle, it is possible to construct it independently. In order to do so we need the notion of a \textit{projective density}.%To do so, we consider the transformation of the derivative of a volume form under a projective change (\ref{eq:proj_change}). This approach can be motivated by the fact that the special linear group $SL(n,\R)$, which we have already seen to be important in projective geometry, is the group of volume preserving transformations of $\R^n$.
\subsubsection{Projective densities}
From the projective change of connection (\ref{eq:proj_change}) we can derive the the corresponding change in $\nabla\chi$ for some $m$--form $\chi$ on $N$:
\be \label{eq:p-form_change}
\ov{\nabla}_i\chi_{jk\dots l} = \nabla_i\chi_{jk\dots l} - (m+1)\Upsilon_i\chi_{jk\dots l} - (m+1)\Upsilon_{[i}\chi_{jk\dots l]}.
\ee
In particular, for a volume form ($m=n$) we find
\[
\ov{\nabla}_i\chi_{jk\dots l} = \nabla_i\chi_{jk\dots l} - (n+1)\Upsilon_i\chi_{jk\dots l},
\]
where the final term in (\ref{eq:p-form_change}) has vanished because it contains a symmetrisation over $n+1$ indices. We can write this in a more compact way as
\[
\ov{\nabla}_i\chi = \nabla_i\chi - (n+1)\Upsilon_i\chi.
\]

Note that for sections $\chi$ of the bundle $\mathcal{E}(w):=(\Lambda^n)^{-w/(n+1)}$ we have
\be \label{eq:ddensity_change}
\ov{\nabla}_i\chi = \nabla_i\chi + w\Upsilon_i\chi.
\ee
We called such sections \textit{projective densities of weight $w$}, and for any vector bundle $\mathcal{B}\rightarrow N$ we write $\mathcal{B}(w)$ for the tensor product of $\mathcal{B}$ with $\mathcal{E}(w)$. For example, $T^*N(w)$ is the bundle of one--forms with projective weight $w$, and for sections $\mu_i$ of $T^*N(w)$ we have
\be \label{eq:dweighted_form_change}
\ov{\nabla}_i\mu_j = \nabla_i\mu_j + (w-1)\Upsilon_i\mu_j - \Upsilon_j\mu_i.
\ee

We can now define the cotractor bundle $\pi:\mathcal{T}^*\rightarrow N$. For a choice of connection in the projective class we identify
\be \label{eq:T*splitting}
\mathcal{T}^* = \mathcal{E}(1)\oplus T^*N(1),
\ee
so that a section can be represented by a pair
\be \label{eq:T*coords}
\begin{pmatrix}
{\chi} \\ {\mu}_i
\end{pmatrix}.
\ee
Under a change of projective connection (\ref{eq:proj_change}), this splitting changes according to
\be \label{eq:chi_mu_change}
\ov{\begin{pmatrix}
{\chi} \\ {\mu}_i
\end{pmatrix}} =
\begin{pmatrix}
\chi \\ \mu_i + \Upsilon_i\chi
\end{pmatrix}.
\ee
Note the exact sequence
\be \label{eq:T*sequence}
0\longrightarrow T^*N(1)\longrightarrow \mathcal{T}^* \overset{X}\longrightarrow \mathcal{E}(1)\longrightarrow 0,
\ee
where we call the map $X$ the projective \textit{canonical tractor}. A choice of connection in the projective class defines a splitting (\ref{eq:T*splitting}) of (\ref{eq:T*sequence}).\footnote{In fact $X$ is a section of a bundle $\mathcal{T}(1)$, where $\mathcal{T}$ can be identified with a direct sum $\mathcal{E}(1)\oplus TN(1)$ given a choice of connection in the projective class. The natural pairing between $\mathcal{T}$ and $\mathcal{T}^*$ defines the map $X:\mathcal{T}^*\rightarrow\mathcal{E}(1)$.}

The bundle $\mathcal{T}^*$ admits a projectively invariant \textit{tractor connection} given by
\be \label{eq:tractor_connection}
\nabla^\mathcal{T}_i \begin{pmatrix}
\chi \\ \mu_j
\end{pmatrix}
= \begin{pmatrix}
\nabla_i \chi - \mu_i \\
\nabla_i\mu_j + \Rho_{ij}\chi
\end{pmatrix},
\ee
which turns out to agree with (\ref{eq:df-ftheta}). Under a change of projective connection (\ref{eq:proj_change}), we find
\begin{align*}
\ov{\nabla}^\mathcal{T}_i\ov{\begin{pmatrix}
\chi \\ \mu_j
\end{pmatrix}}
&= \ov{\nabla}^\mathcal{T}_i\begin{pmatrix}
\chi \\ \mu_j + \Upsilon_j\chi \end{pmatrix} \\
&= \begin{pmatrix}
\ov{\nabla}_i\chi - (\mu_i + \Upsilon_i\chi) \\
\ov{\nabla}_i(\mu_j+\Upsilon_j\chi) + \ov{\Rho}_{ij}\chi
\end{pmatrix} \\
&= \begin{pmatrix}
\nabla_i\chi + \Upsilon_i\chi - (\mu_i + \Upsilon_i\chi) \\
\nabla_i(\mu_j + \Upsilon_j\chi) - \Upsilon_j(\mu_i + \Upsilon_i\chi)
+ (\Rho_{ij} - \nabla_i\Upsilon_j + \Upsilon_i\Upsilon_j) \chi
\end{pmatrix},
\end{align*}
where we have used (\ref{eq:chi_mu_change}) in the first line, (\ref{eq:tractor_connection}) in the second and (\ref{eq:ddensity_change}), (\ref{eq:dweighted_form_change}) and (\ref{eq:schout_change}) in the third. After some cancellation, we identify
\begin{align*}
\ov{\nabla}^\mathcal{T}_i\ov{\begin{pmatrix}
\chi \\ \mu_j
\end{pmatrix}}
&= \begin{pmatrix}
\nabla_i\chi - \mu_i \\
\nabla_i\mu_j + \Upsilon_j\nabla_i\chi - \Upsilon_j\mu_i + \Rho_{ij}\chi
\end{pmatrix} \\
&= \ov{\begin{pmatrix}
\nabla_i\chi - \mu_i \\
\nabla_i\mu_j + \Rho_{ij}\chi
\end{pmatrix}}
= \ov{\nabla^\mathcal{T}_i \begin{pmatrix}
\chi \\ \mu_j
\end{pmatrix}}
\end{align*}
using (\ref{eq:tractor_connection}) and the change of splitting (\ref{eq:chi_mu_change}) adapted to the tensor product of $T^*N$ and $ \cT^*$ of which the derivative is a section.

\mynote{With a bit more work Eastwood calculates the curvature of the tractor connection and proves the statements about when projective structures are flat which I made above. Could add this.}

Any bundle associated to $P$ via some $H$--module is equipped with a tractor connection which is inherited from the connection on the standard cotractor bundle, or equivalently from the Cartan connection. It is this connection, with its special equivariance property, that allows us to construct an Einstein metric as in invariant of the projective structure. This construction is the subject of the following section.

\section{The projective to Einstein correspondence}
The work of \cite{DM} can be understood from two perspectives, one of which is based on the Cartan bundle of the projective structure and one of which is based on the cotractor bundle. We will make use of both, since the Cartan and tractor approaches are most natural in chapters \ref{chap:KK_lift} and \ref{chap:c-proj} respectively. We begin by describing the viewpoint based on the Cartan geometry.

\subsection{Cartan perspective}

Consider a quotient of the total space $P$ of the
Cartan bundle by $GL(n,\mathbb{R})$, which is embedded in $H$
in the obvious way:
\be \label{eq:GL(n)_embedding}
GL(n,\mathbb{R})\ni a\longmapsto\begin{pmatrix}\mathrm{det}a^{-1} & 0\\
0 & a
\end{pmatrix}\in H.
\ee
It is easily verified that 
\[
\begin{pmatrix}\mathrm{det}a^{-1} & 0\\
0 & a
\end{pmatrix}\begin{pmatrix}0 & \eta\\
\omega & 0
\end{pmatrix}\begin{pmatrix}\mathrm{det}a^{-1} & 0\\
0 & a
\end{pmatrix}^{-1}=\begin{pmatrix}0 & \eta a^{-1}\mathrm{det}a^{-1}\\
(\mathrm{det}a)a\omega & 0
\end{pmatrix},
\]
for any $a\in GL(n,\mathbb{R})$, meaning that due to the equivarience
property of the Cartan connection, the natural contraction $\eta\omega:=\sum_{i}\eta_{(i)}\otimes\omega^{(i)}$
defined by $\theta$ is presevered by the adjoint action of this $GL(n,\mathbb{R})$
subgroup. It thus descends to a naturally defined object on the quotient
$M=P/GL(n,\mathbb{R})$.

\begin{theo}{\cite{DM}}\label{thm:DM}
There exist a metric and two-form
$(g,\Omega)$ on $\mbox{\ensuremath{M=P/GL(n,\mathbb{R})}}$ such
that the quotient map $q:P\rightarrow M$ gives
\begin{eqnarray} 
q^{*}g & = & \mathrm{Sym}(\eta\omega) \label{eq:g_cartan} \\
q^{*}\Omega & = & \mathrm{Ant}(\eta\omega), \label{eq:Omega_cartan}
\end{eqnarray}
where $\mathrm{Sym}$ and $\mathrm{Ant}$ denote the symmetric and
anti-symmetric parts of the $(0,2)$ tensor $\eta\omega$. Moreover,
$\Omega$ is closed as a consequence of the Bianchi identity, $g$
is Einstein with non-zero scalar curvature, and the two are related
by an endomorphism $J$ satisying $J^{2}=\mathrm{id}$. Hence $(g,\Omega)$
is an almost para-Kahler structure on $M$.
\end{theo}

Note that the full proof of theorem \ref{thm:DM} only appears explicitly in \cite{DM} in the case $n=2$, although it can be generalised to $n>2$. This generalisation is discussed in their appendix. They show that the Ricci scalar of $g$ is $24$ in the case $n=2$. In chapter \ref{chap:KK_lift} we will need the Ricci scalar for general $n$. We will calculate this under the assumption (stated without proof in \cite{DM}) that $g$ is Einstein.

\mynote{Motivate chapter \ref{chap:KK_lift} here by mentioned the alternative quotient?}

%The quotient $M$ turns out to be an affine bundle over $N$ with
%structure group $H$, i.e. $H$ acts affinely on the fibers of $\rho:M\rightarrow N$,
%and sections of this bundle are in one-to-one correspondence with
%representative connections $\nabla\in[\nabla]$. This means that given
%some choice of connection $\nabla\in[\nabla]$ we have a diffeomorphism
%$\varphi:T^{*}N\rightarrow M$ with which we can pull back the pair
%$(g,\Omega)$. In canonical local coordinates $(x^{i},p_{i})$ on
%the cotangent bundle, we find
%\begin{eqnarray}
%\varphi^{*}g & = &  dp_{i}\odot dx^{i}-(\Gamma_{ij}^{k}p_{k}-p_{i}p_{j}-\Rho_{ij}) dx^{i}\odot dx^{j},\label{eq:coord_form}\\
%\varphi^{*}\Omega & = &  dp_{i}\wedge dx^{i}+\Rho_{ij} dx^{i}\wedge dx^{j},\qquad i,j=1,\dots ,n.\nonumber 
%\end{eqnarray}
%Here $\Gamma_{jk}^{i}$ are the connection components of the representative
%connection $\nabla$ that we chose, and its Schouten tensor is denoted $P_{ij}$. This can be shown to be projectively invariant
%in the sense that a different choice of $\nabla\in[\nabla]$ corresponds
%to shifting the fiber coordinates $p_{i}$, i.e. metrics on $T^{*}N$
%resulting from pulling back $g$ using different representative connections
%are isometric. Explicitly, a projective transformation (\ref{eq:proj_change})
%corresponds to a change
%\begin{equation}
%\Rho_{ij}\longrightarrow \Rho_{ij}+\Upsilon_{i}\Upsilon_{j}-\nabla_{i}\Upsilon_{j},\qquad p_{i}\longrightarrow p_{i}+\Upsilon_{i}.\label{eq:p_change}
%\end{equation}

A coordinate expression for the metric $g$ and two--form $\Omega$ can be obtained by writing out the Cartan connection $\theta$ explicitly for some choice of connection in the projective class. Alternatively, one can take a tractor approach using the tractor connection (\ref{eq:tractor_connection}). This approach is a product of \cite{DGW}.
\mynote{Might be good to add Thomas' construction explicitly. This would be better for justifying the comments made in chapter \ref{chap:KK_lift} about the lifted metric being a different quotient of the Cartan bundle.}

\subsection{Tractor perspective} \label{sec:trac_construction}

From the tractor perspective, the space $M$ will turn out to be the projectivised cotractor bundle of $N$ with an $\RP^{n-1}$ sub--bundle removed from each fiber. We can see this as follows.

On the total space of $\cT^*$ we pullback $\pi:\cT^*\to N$ along $\pi$ to get $\pi^*(\cT^*)\to \cT^*$ as a vector bundle over the total space $\cT^*$. By construction this bundle has a tautological section $U\in \Gamma (\pi^*(\cT^*))$.  We also have $\pi^*(\cT(w))$ for any weight $w$, and we shall write simply $X\in \Gamma(\pi^*(\cT(1)))$ for the pullback to $\cT^*$ of the canonical tractor $X$ on $N$.

Now define
\be
\label{projection_map}
\kappa: \cT^*\longrightarrow \mathcal{M}:=\mathbb{P}(\cT^*)
\ee
by the fibrewise projectivisation, and use $\pi_\cM$ for the map
$$
\pi_\cM:\mathcal{M}\to N.
$$
We denote by $\cE_{\cT^*}(w')$, for $w'\in \mathbb{R}$, the line
bundle on $\mathbb{P}(\cT^*)$ whose sections correspond to functions
$f: \pi^*\cT^* \to\mathbb{R} $ that are homogeneous of degree $w^\prime$ in
the fibres of $\pi^*\cT^*\to \mathbb{P}(\cT^*)$. For any weight $w$ we also have $\cE(w)$ on $N$ and its pull back to the bundle $\pi_\cM^*\cE(w)\to \mathbb{P}(\cT^*)$.
We define the product of these two density bundles on $\cM$ as
$$
\ce(w,w'):= \pi^*\cE(w) \otimes \cE_{\cT^*}(w').
$$

On $\cT^*$ there is  a canonical density $\tau\in \Gamma(\pi^*\cE(1))$ given by
$$
\tau:= X\hook U .
$$
%% and this is homogeneous of degree 1 up the fibres of $\pi$ in the total space $\cT^*$.
Note that $\tau$ is homogeneous of degree 1 up the fibres of the
map $\cT^*\to \mathcal{M}$. Thus $\tau$ determines, and is equivalent
to, a section (that we also denote) $\tau$ of the density bundle $\cE(1,1)$. So $\mathcal{M}$ is stratified according to
whether or not $\tau$ is vanishing, and we write $\mathcal{Z}(\tau)$
to denote, in particular, the zero locus of $\tau$. With these tools we can recover the construction in \cite{DM}.

%\rs{Note the
%  second density bundle is not necessarily orientable.}

\begin{theo}\cite{DGW}\label{metric} 
There is a metric $g$ and two--form $\Omega$ on $\mathcal{M}\setminus \mathcal{Z}(\tau)$ determined by the canonical pairing of the horizontal and vertical subspaces of $T(\cT^*)$. The pair $(g,\Omega)$ agrees with (\ref{eq:g_cartan}, \ref{eq:Omega_cartan}).
\end{theo}
{\bf Proof.}
 Considering first the total space $\cT^*$ and then its tangent
 bundle, note that there is an exact sequence
  \begin{equation}\label{TM}
0\to \pi^* \cT^*\to T(\cT^*)\to \pi^*TN\to 0,
  \end{equation}
  where we have identified $\pi^* \cT^*$ as the vertical sub-bundle of $T(\cT^*)$.
The tractor connection on the vector bundle $\cT^*\to N$ is equivalent to a splitting of this sequence, identifying $\pi^*TN$ with a distinguished  sub-bundle of horizontal subspaces in 
$ T(\cT^*)$ so that we have 
\begin{equation}\label{HV}
T(\cT^*)=  \pi^*TN\oplus \pi^* \cT^* .
\end{equation}

%From the usual Euler sequence of projective space (or see (\ref{useful}) in the last Section) it follows that
We move now to the total space of $\mathbb{P} {\cT^*}$, and we note that again the tractor (equivalently, Cartan) connection determines a splitting of the tangent bundle $T(\mathbb{P} {\cT^*})$ in which the second term of the display (\ref{HV}) is replaced by a quotient of $\pi^* \cT^*(0,1)$ \cite{CGH-duke}. Indeed, if we work at a point $p\in \mathbb{P}(\cT^*)$, observe that $\pi^*\cT^*(0,1)$ has a filtration
\begin{equation}\label{quotient}
0\to \cE (0,0)_p\stackrel{U_p}{\to}  \pi_\cM^* \cT^*(0,1)|_p \to  \pi_\cM^* \cT^*(0,1)|_p/\langle U_p \rangle \to 0
\end{equation}
where, as usual, $U$ is the canonical section. 
But away from $\mathcal{Z}(\tau )$, we have that  $U$ canonically splits 
the appropriately re-weighted pull back of  the sequence (\ref{eq:T*sequence})
$$
0\to \pi_\cM^* T^*N(1,1)   \to \pi_\cM^* \cT^*(0,1) \stackrel{X/\tau}{\to} \cE(0,0) \to 0 .
$$
This identifies the quotient in (\ref{quotient}), and thus we have canonically
$$
T(\mathbb{P}(\cT^*)\setminus \mathcal{Z}(\tau))=  \pi_\cM^*TN\oplus \pi_\cM^*T^*N(1,1).
$$
It follows that on  $\mathcal{M}$
there is canonically a metric $\boldsymbol{g}$ and symplectic form $\boldsymbol{\Omega}$ taking values in $\cE(1,1)$, given by
\begin{eqnarray*}
\boldsymbol{g}(w_1,w_2)&=& \frac{1}{2}\Big(
\pp_H(w_1)\hook \pp_V(w_2)+\pp_H(w_2)\hook\pp_V(w_1)\Big) \quad \mbox{and} \\
\boldsymbol{\Omega}(w_1,w_2)&=& \frac{1}{2}\Big(\pp_H(w_1)\hook \pp_V(w_2)-\pp_H(w_2)\hook\pp_V(w_1)\Big)
\end{eqnarray*}
where
$$
\pp_H: T(\mathcal{M}\setminus \mathcal{Z}(\tau))\to \pi_\cM^*TN \quad \mbox{and} \quad \pp_V: T(\mathcal{M}\setminus \mathcal{Z}(\tau))\to \pi_\cM^* T^*N(1,1)
$$
are the projections.
Then we obtain the metric and symplectic form by
\be
\label{almost_there}
g:=\frac{1}{\tau}\boldsymbol{g} \qquad \mbox{and} \qquad \Omega:=\frac{1}{\tau}\boldsymbol{\Omega} .
\ee
What remains to be done, is to show that (\ref{almost_there}) agrees
with the form obtained in \cite{DM} once a trivialisation of
$\cT^*\rightarrow N$ has been chosen.

Let $p\in N$ and let ${\mathcal W}\subset N$ be an open 
neighbourhood of $p$ with 
local coordinates $(x^1, \dots, x^n)$ such that
$T_pN=\mbox{span}(\p/\p x^1, \dots, \p/\p x^n)$. The connection 
(\ref{tractor_con}) gives a splitting of $T(\cT^*)$ into the horizontal and
vertical sub-bundles
\[
T(\cT^*)=H(\cT^*)\oplus V(\cT^*),
\]
as in (\ref{HV}).
To obtain the explicit form of this splitting, let $V_\alpha,\ \alpha=0, 1, \dots, n$ be components of a local section of $\cT^*$ in the trivialisation over ${\mathcal{W}}$.
Then
\[
\nabla^{\cT^*} V_\beta=d V_\beta-\gamma_{\beta}^{\alpha} V_\alpha,
\]
where $\gamma_{\alpha}^\beta= \gamma_{i\alpha}^\beta dx^i$, and the components
of the co-tractor connection 
$\gamma_{i\alpha}^\beta$  are given in terms of the connection
$\nabla$ on $N$, and its Schouten tensor, and 
can be read--off from (\ref{eq:tractor_connection}):
\[
\gamma_{i0}^0=0, \quad \gamma_{i0}^j=\delta_i^j,\quad
\gamma_{ij}^k=\Gamma_{ij}^k, \quad \gamma_{ij}^0=-\Rho_{ij}.
\]
In terms of these components we can write
\begin{align*}
H(\cT^*)&=\mbox{span}
\Big( \frac{\p}{\p x^i}+ {\gamma_{i\alpha}^\beta} V_\beta
\frac{\p}{\p V_{\alpha}}, i=1, \dots, n \Big), \\
 V(\cT^*)&=\mbox{span}\Big(\frac{\p}{\p V_{\alpha}}, \alpha=0, 1, 
\dots, n\Big).
\end{align*}
Setting $p_i=V_i/V_0$, where $\tau=V_0\neq 0$ %\footnote{Rod @ Maciej: I have added $V_0=\tau$. You agree, right!?}
on the complement of 
$\mathcal{Z}(\tau)$, 
  we can compute the push forwards
of these subspaces to $\mathbb{P}(\cT^*)\setminus {\mathcal{Z}(\tau)}$:
\[
\kappa_* H(\cT^*)=\mbox{span}\Big(h_i\equiv
\frac{\p}{\p x^i}-
(\Rho_{ij}+p_ip_j  -\Gamma_{ij}^kp_k)\frac{\p}{\partial p_j}
\Big), \quad \kappa_* V(\cT^*)=\mbox{span}\Big(v^i\equiv\frac{\p}{\p p_i}
\Big).
\]
The non--zero components of the  metric (\ref{almost_there}) are given by
\[
g(v^i, h_j)={\delta^i}_j,
\]
or in local coordinates $(x^1,\dots,x^n,p_1,\dots,p_n)$ on $\cM\setminus {\mathcal{Z}(\tau)}$,
\begin{eqnarray}
g & = &  dp_{i}\odot dx^{i}-(\Gamma_{ij}^{k}p_{k}-p_{i}p_{j}-\Rho_{ij}) dx^{i}\odot dx^{j},\label{eq:coord_form}\\
\Omega & = &  dp_{i}\wedge dx^{i}+\Rho_{ij} dx^{i}\wedge dx^{j},\qquad i,j=1,\dots ,n.\nonumber 
\end{eqnarray}
This is identical to the form appearing in \cite{DM}.
\koniec

\begin{rmk}
The expressions (\ref{eq:coord_form}) are projectively invariant in the sense that a different choice of $\nabla\in[\nabla]$ corresponds to shifting the fiber coordinates $p_{i}$, i.e. metrics corresponding to different representative connections are isometric. Explicitly, a projective transformation (\ref{eq:proj_change}) corresponds to a change \begin{equation}
p_{i}\longrightarrow p_{i}+\Upsilon_{i},\label{eq:p_change}
\end{equation}
as can be seen from (\ref{eq:chi_mu_change}) and the definitions of $V_i$ and $p_i$.
\end{rmk}

\begin{rmk}
\mynote{If I add an explicit construction from Thomas, this might be easier to motivate from his perspective.}
Next we observe that $\mathbb{P}(\cT^*)\setminus \mathcal{Z}(\tau)$ is an affine bundle modelled on $T^* N$. Given a connection in the projective class and hence a decomposition (\ref{eq:T*splitting}), there is a smooth fibre bundle isomorphism
  \begin{equation}\label{key-id}
\iota : T^* N\to \mathbb{P}(\cT^*)\setminus \mathcal{Z}(\tau).
    \end{equation}
%First, given  $\nabla$, we can represent an element $U\in
%  \cT^*_p$ ($p\in N$) by the pair $(\tau , \mu) \in \cE (1)_p\oplus
%  T_p^*N(1)$,  or, if we choose coordinates on $N$, by collection
%\be
%\label{tractor_U}
%U=(\tau, \mu_i), \quad i=1, \dots, n .
%\ee
%  Then, dropping the choice $\nabla \in [\nabla]$, $U\in
%  \cT^*_p$ is an equivalence class of such pairs by the equivalence
%  relation (\ref{ttrans}) that covers the
%  equivalence relation between elements of $[\nabla]$. 
%
 % Thus, given $\nabla$, and  from the naturality of all maps, it
 % follows that the total space of $T^*N$ can be identified with $\mathbb{P}(\cT^*)\setminus \mathcal{Z}(\tau)$
 % by (for each $p\in N$)
given by
\begin{equation}
T_p^*N\ni p_i  \mapsto [(1,p_i)]=[(\tau,\tau p_i)]\in
\mathbb{P}(\cT_p^*)\setminus \mathcal{Z}(\tau) .
\end{equation}
\end{rmk}

\begin{rmk}
A feature of this construction is that in each dimension $n$ (of $N$)
either the hypersurface $\mathcal{Z}(\tau)$ (if $n$ odd) is not
orientable, or $\mathcal{M}$ (if $n$ even) is not orientable.
\end{rmk} 

\mynote{Make a comment about the model case? Motivated by chapter \ref{chap:c-proj} and maybe \ref{chap:EW_and_toda}?}

\begin{rmk}
In fact, the metric and symplectic form (\ref{eq:coord_form}) turn
out to belong to a one-parameter family $\{g_{\Lambda}\}$, which
can be written in local coordinates as 
\begin{eqnarray}
g_{\Lambda} & = &  dp_{i}\odot dx^{i}-(\Gamma_{ij}^{k}p_{k}-\Lambda p_{i}p_{j}-\Lambda^{-1}\Rho_{ij}) dx^{i}\odot dx^{j}\label{eq:general_g}\\
\Omega_{\Lambda} & = &  dp_{i}\wedge dx^{i}+\frac{1}{\Lambda}\Rho_{ij} dx^{i}\wedge dx^{j},\qquad i,j=1,\dots ,n.\label{eq:general_omega}
\end{eqnarray}
%Metrics of the form (\ref{eq:general_g}) are a subclass of so-called
%Osserman metrics. More details can be found in \cite{No-louzao1991}.
They are all Einstein with non--zero scalar curvature $24\Lambda$, but for $\Lambda\neq1$ the relation to projective geometry is lost. For the remainder of the thesis we will write $g$ for $g_{\Lambda=1}$ unless stated otherwise. Note that $\{g_\Lambda\}$ will be the subject of chapter \ref{chap:KK_lift}, whilst in chapters \ref{chap:c-proj} and \ref{chap:EW_and_toda} we will restrict our attention to $g$ because the projective geometry is a key aspect of the content of these chapters.
\end{rmk}


 \subsection{Symmetries of $(M,g_{\Lambda},\Omega_{\Lambda})$}

Recall that a projective vector field on any manifold with a connection
generates a 1-parameter family of transformations which preserve the
geodesics of that connection up to parametrisation. Projective vectors
fields thus naturally arise as the symmetries of a projective structure.
Explicitly, a vector field $K$ is projective if it satisfies
\begin{equation}
\mathcal{L}_{K}\Gamma_{ij}^{k}=\delta_{i}^{k}\Upsilon_{j}+\delta_{j}^{k}\Upsilon_{i}\label{eq:proj_transf}
\end{equation}
for some 1-form $\Upsilon$, where $\Gamma_{ij}^{k}$ are the connection
components, and their Lie derivative is defined (see \cite{yano})
by
\begin{equation}
\mathcal{L}_{K}\Gamma_{ij}^{k}\equiv\frac{\partial^{2}K^{k}}{\partial x^{i}\partial x^{j}}+K^{m}\frac{\partial\Gamma_{ij}^{k}}{\partial x^{m}}-\Gamma_{ij}^{m}\frac{\partial K^{k}}{\partial x^{m}}+\Gamma_{im}^{k}\frac{\partial K^{m}}{\partial x^{j}}+\Gamma_{mj}^{k}\frac{\partial K^{m}}{\partial x^{i}}.\label{eq:liederivGamma}
\end{equation}


One consequence of the symmetry property of the Cartan connection discussed in remark \ref{rmk:theta_symmetry} is that for every open set $\mathcal{U}\subset N$ we have an isomorphism between the Lie algebra of projective vector fields on $\mathcal{U}$ and the Lie algebra of vector fields on $\pi_P^{-1}(\mathcal{U})$ preserving the natural contraction $\eta\omega$. Such vector fields must descend to vector fields on $\pi_\cM^{-1}(\mathcal{U})$ preserving $(g,\Omega)$. In fact, it can be shown that every Killing vector field of $(M,g_\Lambda)$ is also symplectic with respect to $\Omega_\Lambda$ and is therefore the lift of a projective vector field on $(N,[\nabla])$.

Explicitly, for every projective vector field $K$ of $(N,[\nabla])$
there is a corresponding Killing vector $\mathcal{K}$ of $(M,g_{\Lambda})$
given in local coordinates by 
\begin{equation}
\mathcal{K}=K-p_{i}\frac{\partial K^{j}}{\partial x^{i}}\frac{\partial}{\partial p_{j}}+\frac{1}{\Lambda}\Upsilon_{i}\frac{\partial}{\partial p_{i}},\label{eq:kvf_from_pvf}
\end{equation}
where $\Upsilon_{i}$ is defined by (\ref{eq:proj_transf}).

\subsection{Anti--Self--Duality for $n=2$}
\mynote{Probs be good to properly define anti--self--duality. And expand this section. Not sure what the right balance is between putting stuff here and putting it in chapter \ref{chap:EW_and_toda}. Probs want to motivate that chapter but not say things that the reader will have forgotten by the time they get there.}

For $n=2$, a local of characterisation of the sapces $M$ is provided in
\cite{DM}: they show that any 4-dimensional anti-self-dual
Einstein space with scalar curvature $-24\Lambda$ and a parallel
anti-self-dual totally null distribution can be considered as the
total space of a rank 2 affine bundle $T^{*}N$ over a projective
surface $N$ of the form (\ref{eq:coord_form}). The anti--self-duality property, in combination with the correspondence of symmetries of $(M,g)$ with symmetries of $(N,[\nabla])$, is important in the context of the applications of
the work \cite{DM} to integrability. It means that
if we start with a projective surface with at least one projective
vector field, we will find an ASD Einstein space with at least one
Killing vector field, and thus will be able to perform a symmetry
reduction to obtain an Einstein-Weyl structure in $2+1$ dimensions
and a corresponding solution to the $SU(\infty)$-Toda field equation. This is the subject of chapter \ref{chap:EW_and_toda}.
