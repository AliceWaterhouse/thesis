%!TEX root = ../thesis.tex
%*******************************************************************************
%****************************** Chapter 2: KK_lift *********************************
%*******************************************************************************

\chapter{An Einstein metric on an $S^1$--bundle over $(M,g_\Lambda,\Omega_\Lambda)$}
\label{chap:KK_lift}
In this chapter, we show that there is a canonical Einstein of metric on an $\R^*$--bundle over $M$, with a connection whose curvature is the pull--back
of the symplectic structure from $M$. This metric is interesting in the context of Kaluza-Klein theory. The material covered here is based on material appearing in \cite{DW}.

\mynote{I would like to understand this section from the tractor perspective. Try comparing my lifted metric to what can be obtained by slightly modifying the tractor construction of $g$.}

\section{Flat case} \label{sec:quadric}

We first note that taking the flat projective structure on $\mathbb{RP}^{n}$
results in a one--parameter family of $2n$-dimensional Einstein spaces $M$ which are a Kaluza-Klein
reductions of quadrics in $\mathbb{R}^{n+1,n+1}$. For $N=\RP^n$
the metric and symplectic form on $M$ reduce to 
\begin{eqnarray*}
g_{\Lambda} & = &  dx^{i} \odot d\zeta_{i}+\Lambda(\zeta_{i} dx^{i})^{2}\\
\Omega_{\Lambda} & = &  d\zeta_{i}\wedge dx^{i}.
\end{eqnarray*}

\begin{prop}
The Einstein spaces $M$ corresponding to $\mathbb{RP}^{n}$
are projections from the $2n+1$-dimensional quadrics $\mathcal{Q}\subset\mathbb{R}^{n+1,n+1}$
given by $X^{\alpha}Y_{\alpha}=\frac{1}{\Lambda}$, where $X,Y\in\mathbb{R}^{n+1}$
are coordinates on $\mathbb{R}^{n+1,n+1}$ such that the metric is
given by 
\[
\hat{g}= dX^{\alpha} dY_{\alpha},
\]
 under the embedding
\begin{eqnarray}
X^{\alpha} & = & \begin{cases}
x^{i}\mathrm{e}^{t}, & \alpha=i=1,\dots,n\\
\mathrm{e}^{t}, & \alpha=n+1
\end{cases}\nonumber \\
Y_{\alpha} & = & \begin{cases}
\zeta_{i}\mathrm{e}^{-t}, & \alpha=i=1,\dots,n\\
\mathrm{e}^{-t}\Bigl(\frac{1}{\Lambda}-x^{k}\zeta_{k}\Bigr), & \alpha=n+1
\end{cases}\label{eq:embedding-1}
\end{eqnarray}
following Kaluza-Klein reduction by the vector $\frac{\partial}{\partial t}.$
\end{prop}
\textbf{Proof. }We find the basis of coordinate 1-forms $\{ dX^{\alpha}, dY_{\alpha}\}$
to be
\begin{eqnarray*}
 dX^{\alpha} & = & \begin{cases}
\mathrm{e}^{t}( dx^{i}+x^{i} dt), & \alpha=i=1,\dots,n\\
\mathrm{e}^{t} dt, & \alpha=n+1
\end{cases}\\
 dY_{\alpha} & = & \begin{cases}
\mathrm{e}^{-t}( d\zeta_{i}-\zeta_{i} dt), & \alpha=i=1,\dots,n\\
-\mathrm{e}^{-t}\biggl[\Bigl(\frac{1}{\Lambda}-x^{k}\zeta_{k}\Bigr) dt+x^{k} d\zeta_{k}+\zeta_{k} dx^{k}\biggr], & \alpha=n+1.
\end{cases}
\end{eqnarray*}
The metric is then given by
\begin{eqnarray*}
\hat{g} & = & \mathrm{e}^{t}( dx^{i}+x^{i} dt)\mathrm{e}^{-t}( d\zeta_{i}-\zeta_{i} dt)\ -\ \mathrm{e}^{t} dt\mathrm{e}^{-t}\biggl[\Bigl(\frac{1}{\Lambda}-x^{k}\zeta_{k}\Bigr) dt+x^{k} d\zeta_{k}+\zeta_{k} dx^{k}\biggr]\\
 & = &  dx^{i} d\zeta_{i}\ +\ (x^{i} d\zeta_{i}-\zeta_{i} dx^{i}) dt\ -\ (x^{i}\zeta_{i}) dt^{2}\ -\  dt\biggl[\Bigl(\frac{1}{\Lambda}-x^{k}\zeta_{k}\Bigr) dt+x^{k} d\zeta_{k}+\zeta_{k} dx^{k}\biggr]\\
 & = &  dx^{i} d\zeta_{i}\ -\ \frac{1}{\Lambda} dt^{2}\ -\ 2\zeta_{i} dx^{i} dt\\
 & = &  dx^{i} d\zeta_{i}\ +\ \Lambda(\zeta_{i} dx^{i})^{2}\ -\ \Lambda\Bigl(\frac{ dt}{\Lambda}+\zeta_{i} dx^{i}\Bigr)^{2},
\end{eqnarray*}
which is clearly going to give $g_{\Lambda}$ under Kaluza-Klein reduction
by $\frac{\partial}{\partial t}$.
\koniec

Note that the symplectic form $\Omega$ is the exterior derivative
of the potential term $\zeta_{i} dx^{i}$, implying a possible
generalisation to the curved case.


\section{Curved case}

We now return to a general projective structure $(N,[\nabla])$. Since
symplectic form picks out the antisymmetric part of the Schouten tensor,
it has the fairly simple form
\[
\Omega_{\Lambda}= d\zeta_{i}\wedge dx^{i}-\frac{\partial_{[i}\Gamma_{j]k}^{k}}{\Lambda(n+1)} dx^{i}\wedge dx^{j}.
\]
By inspection, this can be written $\Omega_{\Lambda}=d\mathcal{A}$,
where
\[
\mathcal{A}=\zeta_{i} dx^{i}-\frac{\Gamma_{ik}^{k}}{\Lambda(n+1)} dx^{i}.
\]
This is a trivialisation of the Kaluza-Klein bundle which we are about
to construct. Note that for $\Lambda=1$, under a change of projective
connection (\ref{eq:proj_change}) the corresponding change in the
fiber coordinates (\ref{eq:p_change}) ensures that $\Omega$ and
$\mathcal{A}$ are unchanged.

Motivated by the Kaluza-Klein reduction in the flat case, we consider
the following metric.
\begin{theo}
The metric
\begin{equation}
\hat{g}_{\Lambda}=g_{\Lambda}-\Lambda\Bigl(\frac{ dt}{\Lambda}+\mathcal{A}\Bigr)^{2}\label{eq:G}
\end{equation}
on an $\R^*$--bundle $\kappa_\mathcal{Q}:\mathcal{Q}\rightarrow M$ is Einstein,
with Ricci scalar $2n(2n+1)\Lambda$.
\end{theo}
\textbf{Proof.} We prove this using the Cartan formalism. Our treatment
parallels a calculation by Kobayashi \cite{Kob}, who considered
principal circle bundles over K\"ahler manifolds in order to study the
topology of the base. For the remainder of this chapter we will suppress the constant $\Lambda$, writing $\hat{g}\equiv\hat{g}_{\Lambda}$, $g\equiv g_{\Lambda}$ and $\Omega\equiv\Omega_\Lambda$ since the proof applies to any choice $\Lambda\neq0$ within this family.

Consider a frame
\begin{equation}
e^{a}=\begin{cases}
 dx^{i}, & a=i=1,\dots,n\\
 d\zeta_{i}-(\Gamma_{ij}^{k}\zeta_{k}-\Lambda \zeta_{i}\zeta_{j}-\Lambda^{-1}\Rho_{ij}) dx^{j}, & a=i+n=n+1,\dots2n.
\end{cases}\label{eq:basis}
\end{equation}
In this basis the metric takes the form
\begin{equation}
g=e^{1}\odot e^{n+1}+\dots+e^{n}\odot e^{2n}.\label{eq:g_cov_const}
\end{equation}
We are interested in the metric
\[
\hat{g}=-e^{0}\odot e^{0}+g,
\]
where
\[
e^{0}=\sqrt{\Lambda}\biggl(\frac{ d\lambda}{\Lambda}+\mathcal{A}\biggr).
\]
\mynote{Can we have $\Lambda<0$? I think so. Then what do we mean by $\sqrt{\Lambda}$?}
 We reserve Roman indices $a,b,\dots$ for the $2n$-metric components
$1,\dots,2n$ and allow greek indices $\mu,\nu,\dots$ to take values
$0,1,\dots,2n$. The dual basis to $\{e^{\mu}\}$ will be denoted
$\{E_{\mu}\}$ and will act on functions as vector fields in the usual
way. We wish to find the new connection 1-forms $\hat{\psi}_{\ \nu}^{\mu}$
(defined by $ de^{\mu}=-\hat{\psi}_{\ \nu}^{\mu}\wedge e^{\nu}$)
in terms of the old ones $\psi_{\ b}^{a}$ (defined by $ de^{a}=-\psi_{\ b}^{a}\wedge e^{b}$).
Hence we examine ${d}e^{0}$ to find $\hat{\psi}_{\ a}^{0}.$
\begin{align*}
{d}e^{0}&=\sqrt{\Lambda}{d}\mathcal{A}=\sqrt{\Lambda}\Omega_{ab}e^{a}\wedge e^{b}=-\hat{\psi}_{\ a}^{0}\wedge e^{a}\quad\implies \\
\hat{\psi}_{\ a}^{0}&=\sqrt{\Lambda}\Omega_{[ab]}e^{b}=\sqrt{\Lambda}\Omega_{ab}e^{b},\qquad\hat{\psi}_{\ 0}^{a}=\sqrt{\Lambda}\Omega_{\ b}^{a}e^{b}.
\end{align*}
Since ${d}e^{a}$ is unchanged, we have that
\[
\hat{\psi}_{\ 0}^{a}\wedge e^{0}+\hat{\psi}_{\ b}^{a}\wedge e^{b}=\psi_{\ b}^{a}\wedge e^{b},
\]
thus
\[
\hat{\psi}_{\ b}^{a}\wedge e^{b}=\psi_{\ b}^{a}\wedge e^{b}-\sqrt{\Lambda}\Omega_{\ b}^{a}e^{b}\wedge e^{0}\qquad\implies\qquad\hat{\psi}_{\ b}^{a}=\psi_{\ b}^{a}+\sqrt{\Lambda}\Omega_{\ b}^{a}e^{0}.
\]


We now calculate the curvature 2-forms $\hat{\Psi}_{\ \nu}^{\mu}={d}\hat{\psi}_{\ \nu}^{\mu}+\hat{\psi}_{\ \rho}^{\mu}\wedge\hat{\psi}_{\ \nu}^{\rho}=\frac{1}{2}\mathcal{R}_{\rho\sigma\nu}^{\ \ \ \ \mu}e^{\rho}\wedge e^{\sigma}$
in terms of $\Psi_{\ b}^{a}={d}\psi_{\ b}^{a}+\psi_{\ c}^{a}\wedge\psi_{\ b}^{c}$,
where $\mathcal{R}_{\rho\sigma\nu}^{\ \ \ \ \mu}$ is the Riemann
tensor of $\mathcal{Q}$. Note that we use the notation $\psi_{\ b}^{a}=\psi_{\ bc}^{a}e^{c}$. 
\begin{eqnarray*}
\hat{\Psi}_{\ b}^{a} & = & {d}\hat{\psi}_{\ b}^{a}+\hat{\psi}_{\ c}^{a}\wedge\hat{\psi}_{\ b}^{c}+\hat{\psi}_{\ 0}^{a}\wedge\hat{\psi}_{\ b}^{0}\\
 & = & {d}\psi_{\ b}^{a}+{\sqrt{\Lambda}{d}}(\Omega_{\ b}^{a}e^{0})+\psi_{\ c}^{a}\wedge\psi_{\ b}^{c}+\sqrt{\Lambda}\Omega_{\ c}^{a}e^{0}\wedge\psi_{\ b}^{c}+\sqrt{\Lambda}\Omega_{\ b}^{c}\psi_{\ c}^{a}\wedge e^{0}+\Lambda\Omega_{\ [c}^{a}\Omega_{|b|d]}e^{c}\wedge e^{d}\\
 & = & \Psi_{\ b}^{a}+\sqrt{\Lambda}E_{c}(\Omega_{\ b}^{a})e^{c}\wedge e^{0}+\Lambda(\Omega_{\ b}^{a}\Omega_{cd}+\Omega_{\ [c}^{a}\Omega_{|b|d]})e^{c}\wedge e^{d}+\sqrt{\Lambda}(\Omega_{\ b}^{c}\psi_{\ cd}^{a}-\Omega_{\ c}^{a}\psi_{\ bd}^{c})e^{d}\wedge e^{0}\\
 & = & \Psi_{\ b}^{a}+\sqrt{\Lambda}\nabla_{c}\Omega_{\ b}^{a}e^{c}\wedge e^{0}+\Lambda(\Omega_{\ b}^{a}\Omega_{cd}+\Omega_{\ [c}^{a}\Omega_{|b|d]})e^{c}\wedge e^{d}.\\
\hat{\Psi}_{\ a}^{0} & = & {d}\hat{\psi}_{\ a}^{0}+\hat{\psi}_{\ b}^{0}\wedge\hat{\psi}_{\ a}^{b}\\
 & = & \sqrt{\Lambda}E_{[c}(\Omega_{|a|b]})\theta^{c}\wedge\theta^{b}-\sqrt{\Lambda}\Omega_{ab}\psi_{\ c}^{b}\wedge e^{c}+\sqrt{\Lambda}\Omega_{bc}e^{c}\wedge(\psi_{\ a}^{b}+\sqrt{\Lambda}\Omega_{\ a}^{b}e^{0})\\
 & = & \sqrt{\Lambda}(E_{[d}(\Omega_{|a|b]})-\Omega_{ac}\psi_{\ [bd]}^{c}+\Omega_{c[d}\psi_{\ |a|b]}^{c})e^{d}\wedge e^{b}+\Lambda\Omega_{bc}\Omega_{\ a}^{b}e^{c}\wedge e^{0}\\
 & = & \sqrt{\Lambda}\nabla_{[c}\Omega_{|a|d]}e^{c}\wedge e^{d}+\Lambda\Omega_{bc}\Omega_{\ a}^{b}e^{c}\wedge e^{0}.
\end{eqnarray*}
Hence we have that
\begin{eqnarray*}
\mathcal{R}_{cdb}^{\ \ \ a} & = & R_{cdb}^{\ \ \ a}+2\Lambda(\Omega_{\ b}^{a}\Omega_{cd}+\Omega_{\ [c}^{a}\Omega_{|b|d]})\\
\mathcal{R}_{c0b}^{\ \ \ a} & = & \sqrt{\Lambda}\nabla_{c}\Omega_{\ b}^{a}\\
\mathcal{R}_{cda}^{\ \ \ 0} & = & 2\sqrt{\Lambda}\nabla_{[c}\Omega_{|a|d]}\\
\mathcal{R}_{c0a}^{\ \ \ 0} & = & \Lambda\Omega_{bc}\Omega_{\ a}^{b},
\end{eqnarray*}
and thus, using $\mathcal{R}_{\mu\nu}=\mathcal{R}_{\rho\mu\nu}^{\ \ \ \ \rho}$,
\begin{eqnarray*}
\mathcal{R}_{00} & = & \Lambda\Omega_{bc}\Omega^{bc}=-2n\Lambda=2n\Lambda\hat{g}_{00}\\
\mathcal{R}_{b0} & = & \sqrt{\Lambda}\nabla_{c}\Omega_{\ b}^{c}=0\\
\mathcal{R}_{db} & = & R_{db}+2\Lambda(\Omega_{\ b}^{c}\Omega_{cd}+\frac{1}{2}\Omega_{\ c}^{c}\Omega_{bd}-\frac{1}{2}\Omega_{\ d}^{c}\Omega_{bc})-\Lambda\Omega_{cd}\Omega_{\ b}^{c}\\
 & = & R_{db}+2\Lambda\Omega_{b}^{\ c}\Omega_{dc}\\
 & = & 2(n+1)\Lambda g_{db}-2\Lambda g_{db}=2n\Lambda g_{db}=2n\Lambda\hat{g}_{db}.
\end{eqnarray*}


Note that we have used the facts that $g$ is Einstein with Ricci
scalar $4n(n+1)\Lambda$ and the symplectic form $\Omega$ is divergence--free;
these are justified in lemmas \ref{lem:Ricci_scalar} and \ref{lem:div_free} below. Since $\hat{g}_{a0}=0$,
we conclude that
\[
\mathcal{R}_{\mu\nu}=2n\Lambda\hat{g}_{\mu\nu}=\frac{\mathcal{R}}{2n+1}\hat{g}_{\mu\nu},
\]
i.e. $\hat{g}$ is Einstein with Ricci scalar $2n(2n+1)\Lambda$.
\koniec

Physically, this is a Kaluza-Klein reduction with constant dilation
field and where the Maxwell two-form is related to the reduced metric
by $\Omega_{a}^{\ c}\Omega_{cb}=g_{ab}$. This is what allows both
the reduced and lifted metric to be Einstein. A more general discussion
can be found in \cite{Pope}.

From the Cartan perspective, $\hat{g}_{\Lambda=1}$ can be thought
of as a metric on the $2n+1$-dimensional space obtained by taking
a quotient $\hat{q}:P\mapsto P/SL(n,\mathbb{R})=\mathcal{Q}$ of the Cartan
bundle, where we embed $SL(n,\mathbb{R})\subset GL(n,\mathbb{R})$
in $H$ as in (\ref{eq:GL(n)_embedding}) but with $a$ now denoting
an element of $SL(n,\mathbb{R})$ (so that $\mathrm{det}a^{-1}=1$).
This new subgroup acts adjointly on $\theta$ as
\[
\begin{pmatrix}1 & 0\\
0 & a
\end{pmatrix}\begin{pmatrix}-\mathrm{tr}\phi & \eta\\
\omega & \phi
\end{pmatrix}\begin{pmatrix}1 & 0\\
0 & a^{-1}
\end{pmatrix}=\begin{pmatrix}-\mathrm{tr}\phi & \eta a^{-1}\\
a\omega & \phi
\end{pmatrix},
\]
so not only is the inner product $\eta\omega$ invariant but also
the $(0,0)$-component $\theta_{\ 0}^{0}=\mathrm{-tr}\phi$, which
is a scalar one-form whose exterior derivative is constrained by (\ref{eq:curvature_2-form}) to be $ d\theta_{\ 0}^{0}=-\theta_{\ i}^{0}\wedge\theta_{\ 0}^{i}=-\mathrm{Ant}(\eta\wedge\omega)$.
Thus, denoting by $\mathcal{A}$ the object on $\mathcal{Q}=P/SL(n,\mathbb{R})$
which is such that $\hat{q}^{*}\mathcal{A}=\mathrm{tr}\phi$, we have that
$ d\mathcal{A}=\Omega$ (where we are now taking $\Omega$ and $g$
to be defined on $\mathcal{Q}$ by $\hat{q}^{*}\Omega=\mathrm{Ant}(\eta\wedge\omega)$
and $\hat{q}^{*}g=\mathrm{Sym}(\eta\wedge\omega)$ respectively).
%, or equivalently redefining $\hat{\Omega}=\kappa_\mathcal{Q}^{*}\Omega$ and $\hat{g}=\kappa_\mathcal{Q}^{*}g$).

We then have a natural way of constructing a metric $\hat{g}$
on $\mathcal{Q}$ as a linear combination of $g$ and $e^{0}\odot e^{0}$,
where $e^{0}$ is $A$ up to addition of some exact one-form. It turns
out that there is choice of linear combination such that $\hat{g}$
is Einstein:
\[
\hat{g}=-e^{0}\odot e^{0}+g.
\]
The fact that this metric is exactly (\ref{eq:G}) can be verified
by constructing the Cartan connection of $(N,[\nabla])$ explicitly
in terms of a representative connection $\nabla\in[\nabla]$.
\mynote{Would be nice to verify this explicitly, if I had written out Thomas' explicit construction of $g$ in the intro.}

\subsection{Ricci scalar of $g_\Lambda$ and divergence of $\Omega_\Lambda$}

\begin{lemma} \label{lem:div_free} The symplectic form $\Omega\equiv\Omega_\Lambda$ on $M$ is divergence--free. In index notation,
\[
\nabla^c\Omega_{cb}=0.
\]
\end{lemma}

{\bf Proof.} In the basis
(\ref{eq:basis}) we have $g$ as above (\ref{eq:g_cov_const})
and
\[
\Omega=\sum_{i=1}^{n}e^{i}\wedge e^{i+n}\quad\implies\quad\Omega_{ab}=\sum_{i=1}^{n}\delta_{[a}^{i}\delta_{b]}^{i+n}.
\]
Note that from now on we will omit the summation sign and use the
summation convention regardless of whether $i,j$-indices are up or
down. As in section \ref{sec:quadric}, we look for $\psi_{\ b}^{a}$
by considering $ de^{a}$ (recall that $i,j=1,\dots,n$ and
$a,b=1,\dots,2n$):
\begin{eqnarray*}
 de^{i} & = & 0\\
 de^{i+n} & = & -(E_{l}(\Gamma_{ij}^{k})\zeta_{k}-\Lambda^{-1}E_{l}(\Rho_{ij})) dx^{l}\wedge dx^{j}-(\Gamma_{ij}^{k}-2\Lambda \zeta_{(i}\delta_{j)}^{k}) d\zeta_{k}\wedge dx^{j}\\
 & = & -(E_{l}(\Gamma_{ij}^{k})\zeta_{k}-\Lambda^{-1}E_{l}(\Rho_{ij}))e^{l}\wedge e^{j}\\
 &  & -(\Gamma_{ij}^{k}-2\Lambda \zeta_{(i}\delta_{j)}^{k})(e^{k+n}+(\Gamma_{km}^{l}\zeta_{l}-\Lambda \zeta_{k}\zeta_{m}-\Lambda^{-1}\Rho_{km})e^{m})\wedge e^{j}\\
 & = & \bigl[\Lambda^{-1}E_{m}(\Rho_{ij})-E_{m}(\Gamma_{ij}^{k})\zeta_{k}+\Lambda^{-1}\Gamma_{ij}^{k}\Rho_{km}-\Gamma_{ij}^{k}\Gamma_{km}^{l}\zeta_{l}+\Lambda\Gamma_{ij}^{k}\zeta_{m}\zeta_{k}\\
 &  & +2\Lambda \zeta_{(i}(\Gamma_{j)m}^{l}\zeta_{l}-\Lambda \zeta_{j)}\zeta_{m}-\Lambda^{-1}\Rho_{j)m})\bigr]e^{m}\wedge e^{j}+(2\Lambda \zeta_{(i}\delta_{j)}^{k}-\Gamma_{ij}^{k})e^{k+n}\wedge e^{j}\\
 & = & \bigl[\Lambda^{-1}D_{m}\Rho_{ij}-(D_{m}\Gamma_{ij}^{k})\zeta_{k}-2\zeta_{(i}\Rho_{j)m}\bigr]e^{m}\wedge e^{j}+(2\Lambda \zeta_{(i}\delta_{j)}^{k}-\Gamma_{ij}^{k})e^{k+n}\wedge e^{j}\\
\end{eqnarray*}
Note that we have used $D$ to denote the chosen connection on $N$
with components $\Gamma_{jk}^{i}$. Next we wish to read off the spin
connection $\psi_{\ b}^{a}$ such that ${d}e^{a}=-\psi_{\ b}^{a}\wedge e^{b}$
and the following index symmetries are satisfied:
\begin{eqnarray*}
\psi_{\ j}^{i} & = & \frac{1}{2}\psi_{i+n\, j}=-\frac{1}{2}\psi_{j\, i+n}=-\psi_{\ i+n}^{j+n}\\
\psi_{\ j+n}^{i} & = & \frac{1}{2}\psi_{i+n\, j+n}=-\frac{1}{2}\psi_{j+n\, i+n}=-\psi_{\ i+n}^{j}\\
\psi_{\ j}^{i+n} & = & \frac{1}{2}\psi_{i\, j}=-\frac{1}{2}\psi_{j\, i}=-\psi_{\ i}^{j+n}
\end{eqnarray*}
We find that 
\begin{eqnarray*}
\psi_{\ k+n}^{i+n} & = & (2\Lambda \zeta_{(i}\delta_{j)}^{k}-\Gamma_{ij}^{k})e^{j}=-\psi_{\ i}^{k}\\
\psi_{\ j}^{i+n} & = & \bigl[2(D_{[i}\Gamma_{j]k}^{l})\zeta_{l}-2\Lambda^{-1}D_{[i}\Rho_{j]k}^{\mathrm{S}}-\Lambda^{-1}D_{k}\Rho_{ij}^{\mathrm{A}}+2\zeta_{(j}\Rho_{k)i}-2\zeta_{(i}\Rho_{k)j}\bigr]e^{k}=:A_{ijk}e^{k}\\
\psi_{\ j+n}^{i} & = & 0.
\end{eqnarray*}
One can check that these satisfy both the index symmetries above and
are such that $ de^{a}=-\psi_{\ b}^{a}\wedge e^{b}$, and
we know from theory that there is a unique set of $\psi_{\ b}^{a}$
that have both of these properties. Note that we have used $P^{\mathrm{S}}$
and $P^{\mathrm{A}}$ to denote the symmetric and antisymmetric parts
of $P$ in order to avoid too much confusion from having multiple
symmetrisation brackets in the indices.

We are now ready to calculate the divergence of $\Omega$. Since it
is covariantly constant in this basis, we obtain
\[
\nabla_{c}\Omega_{ab}=-\psi_{\ ac}^{d}\Omega_{db}-\psi_{\ bc}^{d}\Omega_{ad}=-\psi_{\ ac}^{d}\Omega_{db}+\psi_{\ bc}^{d}\Omega_{da}=2\Omega_{d[a}\psi_{\ b]c}^{d}.
\]
We can split the right hand side into
\begin{eqnarray*}
\Omega_{da}\psi_{\ bc}^{d} & = & \Omega_{ka}\psi_{\ bc}^{k}+\Omega_{k+n\, a}\psi_{\ bc}^{k+n}\\
 & = & \delta_{[k}^{i}\delta_{a]}^{i+n}\psi_{\ bc}^{k}+\delta_{[k+n}^{i}\delta_{a]}^{i+n}\psi_{\ bc}^{k+n}\\
 & = & \frac{1}{2}\Bigl(-\delta_{a}^{k+n}\delta_{b}^{i}\delta_{c}^{j}(2\Lambda \zeta_{(i}\delta_{j)}^{k}-\Gamma_{ij}^{k})-\delta_{a}^{k}\delta_{b}^{l+n}\delta_{c}^{j}(2\Lambda \zeta_{(k}\delta_{j)}^{l}-\Gamma_{kj}^{l})-\delta_{a}^{k}\delta_{b}^{l}\delta_{c}^{m}A_{klm}\Bigr).\\
\end{eqnarray*}
The first two terms are the same but with $a\leftrightarrow b$, so
are lost in the antisymmetrisation. Thus
\[
\nabla_{c}\Omega_{ab}=-\delta_{[a}^{k}\delta_{b]}^{l}\delta_{c}^{m}A_{klm}.
\]
Tracing amounts to contracting this with $g^{ac}$:
\[
\nabla^{c}\Omega_{cb}=-\delta_{[a}^{k}\delta_{b]}^{l}g^{ac}\delta_{c}^{m}A_{klm}=-\delta_{[a}^{k}\delta_{b]}^{l}g^{am}A_{klm},
\]
but $g^{am}$ is non-zero only when $a=m+n>n$ and $\delta_{[a}^{k}\delta_{b]}^{l}$
is non-zero only when $a=k\leq n$ or $a=l\leq n$. We can therefore
conclude that the right hand side is zero and $\Omega$ is divergence-free.
\koniec

\begin{lemma} \label{lem:Ricci_scalar} The metric $g\equiv g_\Lambda$ corresponding to a projective structure $(N,[\nabla])$ in dimension $n$ has Ricci scalar
\[
R=4n(n+1)\Lambda.
\]
\end{lemma}
{\bf Proof.} We calculate the Ricci scalar of $g$ (given that it's Einstein, as stated in the appendix of \cite{DM}) via the curvature two-forms $\Psi_{\ b}^{a}={d}\psi_{\ b}^{a}+\psi_{\ c}^{a}\wedge\psi_{\ b}^{c}=\frac{1}{2}R_{cdb}^{\ \ \ a}e^{c}\wedge e^{d}$. We are only interested in non-zero components of the Ricci tensor
such as $R_{i\, j+n}=R_{ci\, j+n}^{\ \ \ \ c}$. In fact, we will
calculate only $R_{m+n\, j}$, for which we need to consider $R_{l\, m+n\, j}^{\ \ \ \ i}$
and $R_{k+n\, m+n\, j}^{\ \ \ \ \ l+n}$, i.e. we need only calculate
$\Psi_{\ j}^{i}$ and $\Psi_{\ \ j}^{l+n}$. 
\[
\Psi_{\ j}^{i}= d\Bigl((\Gamma_{jk}^{i}-2\Lambda \zeta_{(j}\delta_{k)}^{i})e^{k}\Bigr)+\psi_{\ k}^{i}\wedge\psi_{\ j}^{k}+\psi_{\ k+n}^{i}\wedge\psi_{\ j}^{k+n}.
\]
The last term vanishes since $\psi_{\ j}^{k+n}=0$, and the middle
term only has components that look like $\frac{1}{2}R_{lmj}^{\ \ \ i}e^{l}\wedge e^{m}$,
so the only term we are interested in is 
\[
-2\Lambda d\zeta_{(j}\delta_{k)}^{i}e^{k}=-2\Lambda\delta_{(k}^{i}(e^{j)+n}+(\Gamma_{j)m}^{l}\zeta_{l}-\Lambda \zeta_{j)}\zeta_{m}-\Lambda^{-1}\Rho_{j)m})e^{m})\wedge e^{k}.
\]
Again, discarding the $e^{m}\wedge e^{k}$ term gives
\[
-\Lambda(e^{j+n}\wedge e^{k}+\delta_{j}^{i}e^{k+n}\wedge e^{k})=\frac{1}{2}R_{l\, m+n\, j}^{\ \ \ \ i}e^{l}\wedge e^{m+n}+\frac{1}{2}R_{m+n\, l\, j}^{\ \ \ \ i}e^{m+n}\wedge e^{l},
\]
so we conclude
\[
R_{l\, m+n\, j}^{\ \ \ \ i}=\Lambda(\delta_{j}^{i}\delta_{l}^{m}+\delta_{l}^{i}\delta_{j}^{m}).
\]
The other Riemann tensor component we need to know to calculate $R_{m+n\, j}=R_{c\, m+n\, j}^{\ \ \ \ c}$
is $R_{l+n\, m+n\, j}^{\ \ \ \ \ i+n}$, so we examine
\[
\Psi_{\ \ j}^{i+n}={d}\psi_{\ \ j}^{i+n}+\psi_{\ \ k}^{i+n}\wedge\psi_{\ j}^{k}+\psi_{\ \ k+n}^{i+n}\wedge\psi_{\ \ j}^{k+n},
\]
but none of these terms have $e^{l+n}\wedge e^{m+n}$ components,
so $R_{l+n\, m+n\, j}^{\ \ \ \ \ i+n}=0$. Hence 
\[
R_{m+n\, j}=\delta_{l}^{i}R_{l\, m+n\, j}^{\ \ \ \ i}=\Lambda(\delta_{j}^{m}+n\delta_{j}^{m})=\Lambda(n+1)\delta_{j}^{m}.
\]
Setting this equal to $\frac{R}{2n}g_{m+n\, j}=\frac{R}{4n}\delta_{j}^{m}$
we find 
\[
R=4n(n+1)\Lambda,
\]
as required.
\koniec