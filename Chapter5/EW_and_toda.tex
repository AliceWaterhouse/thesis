%!TEX root = ../thesis.tex
%*******************************************************************************
%****************************** Chapter 4: EW_and_toda *********************************
%*******************************************************************************

\chapter{Einstein--Weyl structures and $SU(\infty)$--Toda fields} \label{chap:EW_and_toda}

In this chapter, we focus on the four--dimensional Einstein manifolds that arise from the projective to Einstein correspondence in the case $n=2$. As discussed in chapter \ref{chap:intro}, it is shown in \cite{DM} that the Einstein manifolds in this subclass have anti--self--dual Weyl tensor, and are therefore associated with a twistor space \cite{penrose}. If they also carry a Killing vector field arising from a symmetry of the underlying projective surface, one can extract solutions of the $SU(\infty)$--Toda equation via symmetry reduction to Lorentzian Einstein--Weyl structures in $2+1$ dimensions \cite{JT,Tod_note}.

The aim of this chapter is to exhibit the Einstein--Weyl structures obtainable in this way, resulting in several examples of new, explicit solutions of the Toda equation. We also give an explicit criterion for a vector field that generates a symmetry of a Weyl structure, and prove some results about the Einstein manifold and corresponding twistor space arising from the flat projective surface $\RP^2$. The content of this chapter is based on some of the work in \cite{DW}.

In the case $n=2$ we will write the metric and symplectic form as
\begin{eqnarray}
g =& dz_{A'} \odot dx^{A'} - (\Gamma^{C'}_{A'B'}z_{C'}-z_{A'}z_{B'}-\Rho_{A'B'})dx^{A'} \odot dx^{B'}, 
 \label{eq:g4} \\
{\Omega} =& dz_{A'}\wedge dx^{A'} + \Rho_{A'B'}dx^{A'} \wedge dx^{B'}, \qquad A', B', C'=0, 1. \label{eq:Omega4}
\end{eqnarray}
where we have replaced $\{\zeta_i\}$ with $\{z_{A'}\}$ and shifted the indices from $i,j=1,2$ to $A',B'=0,1$. This is helpful for the twistorial calculations because it agrees with the usual notation for two--component spinor indices. Note that a change of projective connection is now given by
\be
\label{proj_change}
\Gamma^{C'}_{A'B'} \rightarrow \Gamma^{C'}_{A'B'} + \delta^{C'}_{A'}\Upsilon_{B'} + \delta^{C'}_{B'}\Upsilon_{A'}, \qquad z_{A'}\rightarrow z_{A'} + \Upsilon_{A'}, \qquad A',B',C'=0,1.
\ee

\section{Background}
\label{sec:background}



\subsection{Anti--self--duality, spinors and totally null distributions} \label{sec:ASD'ty}
Let $M$ be an oriented four--dimensional manifold with a metric
$g$ of signature $(2, 2)$. The Hodge operator $\ast$ on the space of two forms is an involution, and induces a decomposition \cite{AHS}
\be 
\label{split_ad}
\Lambda^{2}(T^*M) = \Lambda_{-}^{2}(T^*M) \oplus \Lambda_{+}^{2}(T^*M)
\ee
of two-forms
into anti-self-dual (ASD)
and self-dual (SD)  components, which
only depends on the conformal class of $g$. 
The Riemann tensor of $g$ 
can be thought of
as a map $\mathcal{R}: \Lambda^{2}(T^*M) \rightarrow \Lambda^{2}(T^*M)$
\mynote{Be more explicit about the index symmetries once you have put the convention in the intro.}
which admits a decomposition   under (\ref{split_ad}):
\be \label{decomp}
{\mathcal R}=
\left(
\mbox{
\begin{tabular}{c|c}
&\\
$C_++\frac{R}{12}$&$\phi$\\ &\\
\cline{1-2}&\\
$\phi$ & $C_-+\frac{R}{12}$\\&\\
\end{tabular}
} \right).
\ee
Here $C_{\pm}$ are the SD and ASD parts 
of the Weyl conformal curvature tensor\footnote{Note this is a different object to the projective Weyl tensor discussed in chapter \ref{chap:intro2}.}, $\phi$ is  the
trace-free Ricci curvature, and $R$ is the scalar curvature which acts
by scalar multiplication. The metric $g$ is Einstein if $\phi=0$, and the corresponding conformal structure $[g]$ is ASD if $C_+=0$. We will call $g$ \textit{conformally ASD} if it has ASD Weyl tensor. If both of these conditions are satisfied then the Riemann tensor is also anti-self-dual.

\subsubsection{Two--component spinors}
The symmetry group of a metric in signature $(+,+,-,-)$ decomposes under the Lie group isomorphism
\[
SO(2,2)\cong SL(2,\R) \times SL(2,\R)/\mathbb{Z}_2.
\]
Locally there exist real rank two vector bundles $\spp, \spp'$  (spin bundles) over $M$ equipped with parallel symplectic structures
$\ep, \ep'$ such that \cite{penroserindler}
\be
\label{can_bun_iso}
T M\cong {\spp}\otimes {\spp'}
\ee
is a  canonical bundle isomorphism. We will use the usual notation $\iota^A\in\Gamma(\spp),\ \pi^{A'}\in\Gamma(\spp')$, where $A,A'=0,1$. There are unique skew--symmetric sections $\varepsilon\in\Gamma(\spp\otimes\spp)$ and $\varepsilon'\in\Gamma(\spp'\otimes\spp')$, and one can argue that
\be \label{eq:g(UAA'VBB'}
g_{ab}\xi^a\tilde{\xi}^b=\varepsilon_{AB}\varepsilon_{A'B'} \xi^{AA'}\tilde{\xi}^{BB'}
\ee
for vector fields $\xi,\tilde{\xi}$ on $M$.

We identify $\spp$ with its dual according to
\[
\iota_A=\iota^B\varepsilon_{BA},\qquad \iota^A=\varepsilon^{AB}\iota_B,
\]
and similarly for $\spp'$. Note that the contraction is always over adjacent indices descending to the right, and $\varepsilon^{AB}\varepsilon_{CB}=\delta_C^{\ A}$. In higher valence spinors, the relative order of primed and unprimed indices is unimportant.

A vector $\xi\in \Gamma(TM)$ is called null if $g(\xi, \xi)=0$. For $\tilde{\xi}=\xi$ the right hand side of (\ref{eq:g(UAA'VBB'}) is just the determinant of $\xi^{AA'} $ viewed as a matrix, so any null vector is of the form
$\xi=\iota \otimes \pi$ where $\iota$, and $\pi$ are sections of
$\spp$ and $\spp'$ respectively. The antisymmetry of $\varepsilon$ means that $\varepsilon(\iota,\iota)=0$ for any section $\iota\in\Gamma(\spp)$ (and similarly for any $\pi\in\spp'$), so the converse is also true (i.e. any vector that can be written $V=\iota\otimes \pi$ is null).

Since $\spp$ and $\spp'$ have dimension two, the symplectic structures $\varepsilon_{AB},\varepsilon_{A'B'}$ are the unique skew--symmetric two--index spinors up to scale. Any spinor of valence $n$ which is skew on a pair of indices can thus be factorised as the tensor product of a spinor of valence $n-2$ and either $\varepsilon$ or $\varepsilon'$. This leads to the decomposition of a two--form $F_{ab}=F_{AA'BB'}=F_{ABA'B'}$ as
\[
F_{ABA'B'} = \varepsilon_{AB}\Phi_{A'B'} + \varepsilon_{A'B'}\Psi_{AB},
\]
where $\Phi_{A'B'}=\Phi_{(A'B')}$ and $\Psi_{AB}=\Psi_{(AB)}$. One can show using an analogous decomposition of the volume form that $\Phi_{A'B'}$ and $\Psi_{AB}$ are the SD and ASD parts of $F_{ab}$ respectively.

\subsubsection{The nonlinear graviton}
Any two--dimensional distribution on a four--manifold $M$ can be expressed as the kernel of a two--form, and we define a distribution to be (A)SD if the corresponding two--form is (A)SD. Taking any $\iota^A\in\Gamma(\spp)$, the two--form $\iota_A\iota_B\varepsilon_{A'B'}$ defines an ASD distribution $D_\beta=\{\iota^Aw^{A'},w^{A'}\in\Gamma(\spp')\}$ which is totally null in the sense that $g(\xi,\tilde{\xi})=0$ for all $\xi,\tilde{\xi}\in\Gamma(D)$. We call this a $\beta$--distribution. Note that it is only defined up to scale, which means that there is a $\CP^1$ worth of $\beta$--planes at every point in $M$. Given any $\pi^{A'}\in\Gamma(\spp')$, one can similarly define a totally null SD distribution called an $\alpha$--distribution by
\be \label{eq:alpha_dist}
D_\alpha = \{\iota^A\pi^{A'},\iota^{A}\in\Gamma(\spp)\}=\mathrm{span}\{\pi^{A'}\mathsf{e}_{AA'}\},
\ee
where $\{\mathsf{e}_{AA'}\}$ is a null tetrad of vector fields.

An $\alpha$--surface (respectively $\beta$--surface) is a two--dimensional surface in $M$ which is tangent to an $\alpha$--($\beta$--)distribution at every point. A foliation by $\alpha$--surfaces exists if and only if the corresponding distribution (\ref{eq:alpha_dist}) is Frobenius integrable. Penrose's Nonlinear Graviton Theorem \cite{penrose} states that a 
maximal, three dimensional family of $\alpha$--surfaces exists in $M$ if and only if its conformal curvature is ASD, i.e. $C_+=0$. Existence of such a family is equivalent to the distribution (\ref{eq:alpha_dist}) being integrable for any $\pi^{A'}$. We can state this condition more as integrability of the lift of the distribution (\ref{eq:alpha_dist}) to $\spp'$. This is given by
\be \label{eq:twistor_dist}
\mathcal{D}=\mathrm{span}\{L_A:=\pi^{A'}\tilde{\mathsf{e}}_{AA'}\},
\ee
where the vectors
\[
\tilde{\mathsf{e}}_{AA'}=\mathsf{e}_{AA'} - \Gamma^{C'}_{AA'B'}\pi^{B'}\frac{\p}{\p\pi^{C'}}
\]
are the lifts of the null tetrad $\{\mathsf{e}_{AA'}\}$ to $\spp'$, and $\Gamma^{C'}_{AA'B'}$ are the components of the \textit{spin connection} on $\spp'$, which is inherited from the Levi--Civita connection of $g$ on $TM$. We call $\mathcal{D}$ the \textit{twistor distribution}.

In fact, Penrose considers four dimensional \textit{complex} manifolds\footnote{Note that familiar facts from real geometry such as a unique Levi--Civita connection and the Frobenius theorem carry over to holomorphic geometry. See \cite{LeBrun83} for details.} $M$ carrying a metric which is \textit{holomorphic} in the sense that the metric components depend on the coordinates on $M$ and not on their complex conjugates. Then $\spp,\spp'$ are complex vector bundles over $M$ and $\varepsilon,\varepsilon'$ are holomorphic symplectic forms. A real conformally ASD metric in a given signature can then be obtained by choosing the correct reality conditions. In neutral signature, complex conjugation is a map from $\spp$ to itself (or from $\spp'$ to itself) which simply replaces each component of a spinor with its complex conjugate. Thus the reality conditions in neutral signature amount to identifying spinors with their complex conjugates.

The \textit{twistor space} $\mathscr{T}$ of $M$ is then defined as the three--dimensional complex manifold comprising the set of all $\alpha$--surfaces in $M$. Each point $m\in M$ corresponds to a subset $\mathscr{L}_m\subset\mathscr{T}$ of $\alpha$--surfaces which pass through $m$. Since an $\alpha$--surface at $m$ is defined by a $\pi^{A'}\in\spp'|_m$ up to scale, $\mathscr{L}_m$ is an embedding $\CP^1\subset\mathscr{T}$. The \textit{correspondence space} $\mathcal{F}=M\times\CP^1$ has local coordinates $(x^a,\lambda):=(x^a,\pi_{0'}/\pi_{1'})$, where $\pi^{A'}$ parametrises the set of $\alpha$--surfaces through the point in $M$ with coordinates $x^a$. Note that $\mathcal{F}$ can be obtained from the primed spin bundle $\spp'\rightarrow M$ by projectivising each fibre, and carries a distribution $\tilde{\mathcal{D}}=\mathrm{span}\{\tilde{L}_A\}$ given by the push forward of the twistor distribution $\mathcal{D}$ to $\mathcal{F}=\PP(\spp')$. We also call $\tilde{\mathcal{D}}$ the twistor distribution. Note that $\mathcal{F}$ has the alternative definition $\mathcal{F}=\{(Z,m)\in \mathscr{T}\times M \ :\  Z\in L_m\}$, leading to the double fibration
\[
M\leftarrow\mathcal{F}\rightarrow\mathscr{T},
\]
where the map $\mathcal{F}\rightarrow\mathscr{T}$ is the quotient of $\mathcal{F}$ by the leaves of the distribution $\tilde{\mathcal{D}}$. A twistor function is a function on $\mathcal{F}$ which is constant along $\tilde{\mathcal{D}}$. %We can also define the \textit{non--projective twistor space} of $M$ as the quotient of $\spp'$ by the leaves of $\mathcal{D}$.

The Nonlinear Graviton allows us to express an ASD conformal structure in terms of the algebraic geometry of $\mathscr{T}$. First note that if two points $m_1,m_2\in M$ are null--separated, then the corresponding curves $\mathscr{L}_{m_1},\mathscr{L}_{m_2}$ intersect at a single point. This is because any null geodesic must have a tangent vector field of the form $\iota^A\pi^{A'}$ for some sections $\iota^A\in\Gamma(\spp)$ and $\pi^{A'}\in\Gamma(\spp')$, and thus the geodesic is contained within the unique $\alpha$--surface spanned by $\pi^{A'}\mathsf{e}_{AA'}$. This unique $\alpha$--surface corresponds to the point in $\mathscr{T}$ where the curves $\mathscr{L}_{m_1},\mathscr{L}_{m_2}$ meet.

In order to understand this correspondence at an infinitesimal level and thereby explicitly recover an ASD conformal structure from $\mathscr{T}$, we need to understand the \textit{normal bundle} $\mathbb{N}(\mathscr{L}_m):=\cup_{Z\in \mathscr{L}_m}\{T_Z\mathscr{T}/T_Z\mathscr{L}_m\}$ over a $\CP^1$ embedding $\mathscr{L}_m$. This is evidently a complex vector bundle, and in fact it is a \textit{holomorphic} vector bundle, meaning that the total space is a complex manifold and the projection $\mathbb{N}(\mathscr{L}_m)\rightarrow\mathscr{L}_m$ is holomorphic. It is thus subject to the following theorem due to Birkhoff and Grothendieck (see for example \cite{complex_mfds} for a proof).
\begin{theo}[Birkhoff--Grothendieck]
Any rank $k$ holomorphic vector bundle over $\CP^1$ is isomorphic to a direct sum of $k$ complex line bundles $\mathcal{O}(n_i),1\leq i\leq k$, each with first Chern class $n_i$. 
\end{theo}

The first Chern class completely classifies complex line bundles topologically. For us, $\mathcal{O}(n),n\in\mathbb{Z}$ will mean a line bundle over $\CP^1=\mathcal{U}_0\cup\mathcal{U}_1$, where $\mathcal{U}_i=\{Z_i\neq 0\}\subset\CP^1$ for homogeneous coordinates $[Z_0,Z_1]^T$ on $\CP^1$, with transition functions such that local trivialisations $(\lambda_i,v_i)\in\mathcal{U}_i\times\mathbb{C}$ are related on the overlap by $v_1=\lambda_0^{-n}v_0$, where $\lambda_0=\lambda_1^{-1}=Z_1/Z_0$. Note that $\mathcal{O}(-n)$ is dual to $\mathcal{O}(n)$, and $\mathcal{O}(n)$ is the tensor product of $n$ copies of $\mathcal{O}(1)$. A section of $\mathcal{O}(n)$ is represented by functions $\sigma_i(\lambda_i)$ such that $(\lambda_0,\sigma_0(\lambda_0))$ and $(\lambda_1,\sigma_1(\lambda_1))$ correspond to the same point, i.e.
\[
\sigma_1(\lambda_1)=\lambda_0^{-n}\sigma_0(\lambda_0).
\]
If we expand these as power series in the local coordinates and use the fact that ${\lambda}_1=\lambda_0^{-1}$, we find by equating coefficients that they are polynomials of degree at most $n$, making the space of global holomorphic sections $n+1$--dimensional for $n\geq 0$.

%We now define the \textit{correspondence space} $\mathcal{F}=M\times\CP^1$ with local coordinates $(x^a,\lambda):=(x^a,\pi_{0'}/\pi_{1'})$, where $\pi^{A'}$ parametrises the set of $\alpha$--surfaces through the point in $M$ with coordinates $x^a$. Note that $\mathcal{F}$ can be obtained from the primed spin bundle $\spp'\rightarrow M$ by projectivising each fibre. Now consider a holomorphic function on $\spp'$ which is homogeneous of degree $n$ in the fibres. This corresponds to a section of $\mathcal{O}(n)$ over the $\CP^1$ factor of $\mathcal{F}$. 

%The correspondence space has the alternative definition
%\[
%\mathcal{F}=\{(Z,p)\in \mathscr{T}\times M \ :\  Z\in L_p\},
%\]
%leading to the double fibration
%\[
%M\xleftarrow{r}\mathcal{F}\xrightarrow{q}\mathscr{T}.
%\]
%We are now ready to state the following lemma.

%\begin{lemma}[\cite{penrose}]
%The normal bundle of the holomorphic curves $L_p=q(r^{-1}(p))$ corresponding to points $p\in M$ can be identified with $\mathcal{O}(1)\oplus\mathcal{O}(1)$.
%\end{lemma}
%\noindent
%{\bf Proof.} The double fibration picture allows us to identify the normal bundle with the quotient $r^*(T_pM)/\mathrm{span}\{L_A\}$. In their homogeneous form the operators $L_A$ have weight one, and the distribution spanned by them is isomorphic to the bundle $\mathbb{C}^2\otimes\mathcal{O}(-1)$. The definition of the normal bundle as a quotient gives the exact sequence
%\[
%0\rightarrow \mathbb{C}^2\otimes\mathcal{O}(-1)\rightarrow\mathbb{C}^4\rightarrow \mathbb{N}\rightarrow 0
%\]
%and thus $\mathbb{N}=\mathcal{O}(1)\oplus\mathcal{O}(1)$, since the last map is given explicitly by $V^{AA'}\mapsto V^{AA'}\pi_{A'}$ in spinor notation.
%\mynote{I don't understand this proof. I basically copied it out of Maciej's book. It might be better to just explain it rather than stating it formally as a lemma.}
%\koniec

The Nonlinear Graviton Theorem can now be stated as follows.
\begin{theo}[\cite{penrose}]
There is a one--to--one correspondence between holomorphic ASD conformal structures and three--dimensional complex manifolds containing a four--parameter family of $\CP^1$ embeddings with normal bundle $\mathcal{O}(1)\oplus\mathcal{O}(1)$.
\end{theo}
\noindent
From the results of Kodaira \cite{Kodaira} we have that a vector at a point $m\in M$ corresponds to a holomorphic section of the normal bundle $\mathcal{O}(1)\oplus\mathcal{O}(1)$ of the curve $\mathscr{L}_m$ in $\mathscr{T}$ (which we know from above belongs to a four--dimensional space). Penrose shows that we obtain an ASD conformal structure from $\mathscr{T}$ by defining a vector to be null if the corresponding holomorphic section of $\mathcal{O}(1)\oplus\mathcal{O}(1)$ has a zero. This is the infinitesimal version of the intersection condition on $\mathscr{L}_{m_1}$ and $\mathscr{L}_{m_2}$ above. %Note that the vanishing of such a section is a quadratic condition, since $V^{AA'}\pi_{A'}$ can be solved for $\pi_{A'}$ if $\mathrm{det}(V^{AA'})=0$.


\subsubsection{(Anti--)self--duality in the sense of Calderbank}
Although a $\beta$--distribution is intrinsically ASD, there are two subclasses of $\beta$--distribution which we shall call SD or ASD \textit{in the sense of Calderbank} \cite{Cal1} (see also \cite{West}). Let $D_\beta$ be a $\beta$--distribution defined by an ASD two--form $\Sigma_{ab}=\iota_A\iota_B\epsilon_{A'B'}$, and  such that the spinor $\iota_A$ satisfies
\be
\label{dm3}
\nabla_{A'(A}\iota_{B)}=\mathcal{A}_{A'(A}\iota_{B)}
\ee
where $d\mathcal{A}$ is an (A)SD Maxwell field. Then $D_\beta$ is (A)SD in the sense of Calderbank.

 
\subsubsection{Local characterisation of the Einstein manifolds $(M,g)$}

A general ASD metric depends, in the real--analytic category, on six arbitrary functions of three variables \cite{DFK}. Theorem \ref{thm:DM} gives an explicit subclass of such metrics which are additionally constrained by the following local characterisation.

\begin{theo}[\cite{DM}] \label{thm:DMcharacterisation}
Let $(M,g)$ be an real ASD Einstein manifold with scalar curvature $24$ admitting a totally null distribution $D_\beta$ which is ASD in the sense of Calderbank and parallel in the sense that $^{\bf g}\nabla_{\xi} \tilde{\xi}\in\Gamma(D_\beta)$ for all $\xi\in \Gamma(TM),\ \tilde{\xi}\in\Gamma(D_\beta)$. Then $(M,g)$ is conformally flat, or it is locally isometric to (\ref{eq:g4}).
\end{theo}
\noindent

In our coordinates $D_\beta$ is the kernel of the two--form $\Sigma=dx^{0'}\wedge dx^{1'}$, and can be written as ${D_\beta}=\mathrm{span}\{\p/\p z_{0'},\p/\p z_{1'}\}$. We find that
\be
\label{beta_eq}
^{\bf g}\nabla\Sigma=6\mathcal{A}\otimes \Sigma,
\ee
where $d\mathcal{A}=\Omega$, and $\Omega$ is the symplectic form on $M$. Writing $\Sigma_{ab}=\iota_A\iota_B\epsilon_{A'B'}$, (\ref{beta_eq}) implies (\ref{dm3}) for a rescaling of $\mathcal{A}$, so it is the anti--self--duality of $\Omega$ which makes $D$ ASD in the sense of Calderbank.

In section \ref{sec:model} we will consider the model case where $M$ is constructed from the flat projective structure on $\RP^2$. In this case, we can explicitly describe the ASD Maxwell two--form $\Omega$ in terms of the twistor space of $M$,
and we will find that $M$ carries a so--called \textit{pseudo--hyper--Hermitian} structure, in which $\mathcal{A}$ plays an important role.












\subsection{Einstein--Weyl structures}
\begin{defi}
A Weyl Structure $(\mathcal{W},\mathscr{D},[h])$ is a conformal equivalence class of metrics $[h]$ on a manifold $\mathcal{W}$ along with a fixed torsion--free affine connection $\mathscr{D}$ which preserves any representative $h\in[h]$ up to conformal class. That is, for some one-form $\varphi$,
\[
\mathscr{D}h=\varphi\otimes h.
\]
\end{defi}
A pair $(h,\varphi)$ uniquely defines the connection and hence the Weyl structure, so we can alternatively specify a Weyl structure as a triple $(\mathcal{W},h,\varphi)$. However, there is an equivalence class of such pairs which define the same Weyl structure. These are related by transformations
\be
\label{weyl_tr}
h\rightarrow \rho^2h,\quad\varphi\rightarrow\varphi+2d\mathrm{ln}(\rho),
\ee
where $\rho$ is a smooth, non-zero function on $\mathcal{W}$. 
Physically, the Weyl condition in Lorentzian signature corresponds to the statement that null geodesics of the conformal structure $[h]$ are also geodesics of the connection $\mathscr{D}$.

If additionally the symmetric part of the Ricci tensor of $\mathscr{D}$ is a scalar multiple of $h$, then $\mathcal{W}$ is said to carry an Einstein-Weyl structure.
This condition is invariant under (\ref{weyl_tr}). A trivial Einstein--Weyl structure is one whose one--form $\varphi$ is closed, so that it is locally exact and thus may be set to zero by a change of scale (\ref{weyl_tr}). Then $\mathscr{D}$ is the Levi--Civita connection of some representative $h\in[h]$, and this representative is Einstein.

In three dimensions, the Einstein--Weyl equations give a set of five non--linear PDEs on the pair $(h, \varphi)$ which are integrable by the twistor transform of Hitchin \cite{hitchin}.
\begin{theo}[\cite{hitchin}]
There is a one--to--one correspondence between 3--dimensional Einstein--Weyl structures and two--dimensional complex manifolds containing a three--parameter family of $\CP^1$ embeddings with normal bundle $\mathcal{O}(2)$.
\end{theo}
\noindent The conformal structure $[h]$ is obtained by demanding that a vector on $\mathcal{W}$ is null if and only if the corresponding section of $\mathcal{O}(2)$ has a single zero. This condition is equivalent to the quadratic function $s(\lambda)$ which represents the section having vanishing discriminant.

Hitchin's results can be regarded as a reduction of Penrose's twistor transform for ASD conformal structures by the following theorem of Tod.
\begin{theo}\cite{JT} \label{theo_tod1} \begin{enumerate} \item Let $(M, g)$ be a neutral signature, conformally ASD four--manifold with a conformal Killing vector $K$. Let
\be 
\label{EWgen}
h=|K|^{-2}g-|K|^{-4}{\bf{K}}\odot{\bf{K}},\qquad \varphi=\frac{2}{|K|^2}\star({\bf{K}}\wedge d{\bf{K}}),
\ee
where $|K|^2=g(K,K)$, ${\bf{K}}=g(K, \cdot)$ and $\star$ is the Hodge operator defined by $g$. Then $(h, \varphi)$ is a solution of the Einstein--Weyl equations  on the space of orbits $\mathcal{W}$ of $K$ in $M$.
\item Given an Einstein--Weyl structure $(\mathcal{W},h,\varphi)$ there is a one--to--one correspondence between solutions $(\mathscr{V},\alpha)$ to the Abelian monopole equation
\be \label{eq:monopole_eq}
d\mathscr{V}+\frac{1}{2}\varphi \mathscr{V}=\star_h d\alpha.
\ee
on $\mathcal{W}$, where $\mathscr{V}$ is a function and $\alpha$ is a one--form, and conformally ASD four--metrics
\be \label{eq:monopole_correspondence}
g=\mathscr{V}h-\mathscr{V}^{-1}(dx +\alpha)^2
\ee
over $\mathcal{W}$ with an isometry $K=\p/\p x$.
\end{enumerate}
\end{theo}
\noindent If a metric with ASD Weyl tensor has more than one conformal symmetry, then distinct Einstein--Weyl structures are obtained on the space of orbits of conformal Killing vectors which are not conjugate with respect to an isometry \cite{PT}.




\subsection{The $SU(\infty)$--Toda equation}

The $SU(\infty)$--Toda equation is given by
\be
\label{md_toda}
U_{XX}+U_{YY}=\epsilon(e^U)_{ZZ}, \quad\mbox{where}\quad U=U(X, Y, Z), \quad
\mbox{and}\;\;\epsilon=\pm 1
\ee
Equation (\ref{md_toda}) has originally arisen in  the context of complex general relativity \cite{FP, BF82, Prz}, and then
in Einstein--Weyl \cite{ward_toda} and (in Riemannian context, with
$\epsilon=-1$) scalar--flat K\"ahler geometry \cite{LeBrun}. It belongs to a class
of dispersionless systems integrable by the twistor transform 
\cite{MW, MDbook, ADM}, 
the method of  hydrodynamic reduction \cite{F},  and  the Manakov--Santini approach \cite{MS}. 
The equation
is nevertheless not linearisable and most known explicit solutions admit Lie point or other symmetries (there are exceptions - see 
\cite{c_toda, CT,martina, Sheftel}).


The $SU(\infty)$--Toda equation is related to a subclass of Einstein--Weyl structures by the following result
of Tod which improved the earlier result of Przanowski \cite{Prz}.
\begin{theo}
\label{th3int}\cite{Tod_note}
Let $(\mathcal{W},h, \varphi)$ be an Einstein--Weyl structure arising from the first part of Theorem \ref{theo_tod1}, under the additional assumption that
the ASD conformal structure $(M, [g])$ has a representative $g\in[g]$ which is Einstein with non--zero Ricci scalar. Then 
%\begin{enumerate}
%\item The Einstein--Weyl structure admits a shear--free, twist--free geodesic congruence.
%\item
there exists $h\in [h]$, and
coordinates $(X, Y, Z)$ on an open set in $\mathcal{W}$ such that
(assuming the signature of $h$ is $(2, 1)$ and the one--form $dZ$ corresponds to a time--like vector)
\be
\label{metric_toda}
h=e^U(dX^2+dY^2)-dZ^2, \quad \varphi =2U_ZdZ
\ee
and the function $U=U(X, Y, Z)$ satisfies the $SU(\infty)$--Toda equation
(\ref{md_toda}) 
with $\epsilon=1$.
%\end{enumerate}
\end{theo}

Note that the assumptions about the signature of $h$ and the timelike character of $dZ$ in the above theorem are satisfied for all the Einstein--Weyl structures that can be obtained from the projective to Einstein correspondence.

%\mynote{Need to define a congruence and its shear and twist, or modify the statement of Tod's theorem to be independent of the congruence.}

\subsection{Projective structures on a surface}
Recall (see, for example, \cite{BDE}) that a projective structure on a surface can be locally specified by a single second order ODE: taking coordinates $(x,y)$ on the surface we find that geodesics on which $\dot{x}\neq 0$ can be written as unparametrised curves $y(x)$ such that
\be
\label{odealice}
y^{\prime \prime} + a_0(x,y)+3a_1(x,y)y^{\prime}+3a_2(x,y)(y^{\prime})^2 + a_3(x,y)(y^\prime)^3=0,
\ee
where the coefficients $\{a_i\}$ are given by the projectively invariant formulae
\[
a_0=\Gamma^1_{00},\quad
3a_1=-\Gamma^0_{00}+2\Gamma^1_{01},\quad
3a_2=-2\Gamma^0_{01}+\Gamma^1_{11},\quad
a_3=-\Gamma^0_{11}.
\]

Hitchin \cite{hitchin} solves the complexified version of (\ref{odealice}) by the following twistor transform theorem.
\begin{theo}
There is a one--to--one correspondence between
\begin{itemize}
\item equivalence classes under coordinate transformations of complex ODEs of the form (\ref{odealice}), where the coefficients $a_i$ are holomorphic functions of $x$ and $y$, and
\item complex surfaces containing a two--parameter family of $\CP^1$ embeddings with normal bundle $\mathcal{O}(1)$.
\end{itemize}
\end{theo}
\noindent In the case of the ODE resulting from the flat projective structure on $\CP^2$, the corresponding twistor space, whose points correspond to projective lines in $\CP^2$, is the dual projective surface $\CP^{2*}$, and its $\CP^1$ embeddings are given by projective lines in $\CP^{2*}$. In analogy with the Nonlinear Graviton, we can define the correspondence space $\mathcal{F}$ such that a point in $\mathcal{F}$ is given by a point $p\in\CP^2$ and a projective line (or equivalently a direction) through $p$. This makes $\mathcal{F}$ the projectivised tangent bundle $\PP(T\CP^2)$, or equivalently $\PP(T\CP^{2*})$.

As we saw in chapter \ref{chap:intro}, the maximally symmetric projective surface $\RP^2$ has symmetry group $SL(3,\R)$. In fact, the possible symmetry groups of projective surfaces are $SL(3, \R)$, $SL(2,\R)$, the two--dimensional affine group, and $\R$. A partial classification is given in \cite{Bryant}.

\begin{enumerate}
\item On {\bf the flat projective surface} $\RP^2$ described in section \ref{def:RPn}, geodesics $y(x)$ are described in inhomogeneous coordinates $(x,y)=({P^0}/{P^2},{P^1}/{P^2})$ by the ODE
\[
y^{\prime\prime}=0.
\]
\item {\bf The punctured plane $\R^2\backslash\{0\}$} has symmetry group $SL(2,\R)$ acting via its fundamental representation. In this case there is a one--parameter family of projective structures falling into three distinct equivalence classes. For simplicity we will consider only one of the classes, with geodesics $y(x)$ described by the differential equation
\be \label{eq:submaxODE}
y^{\prime\prime} = -(y-xy^\prime)^3,
\ee
where $(x,y)$ are standard Euclidean coordinates on $\R^2$.
\item {\bf The two--dimensional Lie group of affine transformations on $\R$}, which we denote $\mathrm{Aff}(1)$, is generated by the unique non--abelian two--dimensional Lie algebra $\{v_1,v_2\}$, where we choose a basis such that $[v_1,v_2]=v_1$. We can choose coordinates on $\mathrm{Aff}$(1) such that these correspond to vector fields
\[
\frac{\p}{\p y},\quad \frac{\p}{\p x} + y\frac{\p}{\p y},
\]
and using invariance under these vector fields, the geodesic equation can be cast in the form \cite{FLL}
\[
y^{\prime\prime} = e^{-2x}(y^\prime)^3 + A_1y^\prime + A_2e^x,
\]
where $A_1$ and $A_2$ are constants.
\item {\bf The general projective surface with a symmetry}, after a choice of coordinates such that the symmetry is $\frac{\p}{\p x}$, corresponds to a set of geodesics $y(x)$ which satisfy an ODE that can be written uniquely in the form \cite{FLL}
\be \label{eq:1symode}
y^{\prime\prime} = A(y)(y^\prime)^3 + B(y)(y^\prime)^2 + 1.
\ee
\end{enumerate}

Note that each of these classes of projective structures forms a subset of the next, and this can be seen explicitly by some changes of coordinates. For example, the general projective surface with a symmetry is flat when $A(y)=B(y)=0$.

\subsection{From projective surfaces to $SU(\infty)$--Toda fields}
The whole construction can now be summarised in the following diagram
\begin{eqnarray}
\label{diagram}
\text{Projective structure with symmetry} &\overset{\text{thm}\;\ref{thm:DM}}
{\longrightarrow}& \text{ASD Einstein with symmetry}\nonumber\\
\downarrow & & \downarrow \scriptstyle{\text{thm}\;\ref{theo_tod1}}\\
\text{Solution to}\;SU(\infty)\; \text{Toda} &\overset{\text{thm}\;\ref{th3int}}\longleftarrow&\text{Einstein--Weyl.}\nonumber
\end{eqnarray}
%\be
%\label{diagram}
%A \xrightarrow{\text{Theorem}} \text{ASD Einstein with symmetry}\xrightarrow{\t%ext{Theorem}} \text{Einstein--Weyl} \xrightarrow{\text{Theorem}} SU(\infty)\; \%text{Toda}
%\ee
We now consider each of the above projective structures in turn, constructing the corresponding ASD Einstein manifold $(M,g)$ and discussing some examples of Einstein--Weyl structures and $SU(\infty)$--Toda fields that can be obtained from them.


\section{The most general case}
\label{general}

Consider the most general Einstein--Weyl structure arising from the combination
of Theorem \ref{thm:DM} and Theorem \ref{theo_tod1}. Because of the correspondence (\ref{eq:kvf_from_pvf}) between symmetries of $(M, g)$ and symmetries of the projective surface $(N, [\nabla])$, the construction must begin with the general projective surface with at least one symmetry. 
 
By trial and error, we chose a representative connection for (\ref{eq:1symode}) such that the metric (\ref{eq:g4}) had the simplest possible form. The choice of connection we took was
\[
\Gamma^{0}_{11}=A(y),\quad \Gamma^{1}_{00}=-1, \quad \Gamma^{1}_{11}=-B(y)
\]
with all other components vanishing. Note that this choice of connection has a symmetric Ricci tensor, so the Schouten tensor is also symmetric and the symplectic form (\ref{eq:Omega4}) pulls back to just $dz_{A'}\wedge dx^{A'}$. Thus we can write the Maxwell potential $\mathcal{A}$ which is such that $d\mathcal{A}=\Omega$ as $\mathcal{A}=z_{A'}dx^{A'}$. Writing $x^{A'}=(x,y)$, $z_{A'}=(p,q)$, the resulting metric (\ref{eq:g4}) is
\be
\label{einstein_1}
g=(B(y) + p^2 +q)dx^2+2(pq+A(y))dxdy + (-A(y)p+B(y)q+q^2)dy^2 + dxdp +dydq.
\ee
Factoring by $K=\frac{\p}{\p x}$ following the algorithm of Theorem
\ref{theo_tod1}, equation (\ref{EWgen}) gives the following form for the Einstein--Weyl structure.
\begin{prop}
\label{prop1}
The most general  Einstein--Weyl structure arising
from the procedure (\ref{diagram}) is locally equivalent to
\begin{eqnarray}
\label{ew_final}
h&=&\frac{1}{\mathscr{V}}\big((Bq -Ap+ q^2)dy+dq\big)dy
-\Big({pq+A}dy+\frac{1}{2}dp\Big)^2, \label{genh} \\
\varphi&=&\mathscr{V}(4dq+2 pdp), \quad\mbox{where}\quad \mathscr{V}=
({B}+ p^2+q)^{-1}.\nonumber
\end{eqnarray}
Here $(p, q, y)$ are local coordinates on $\mathcal{W}$, $A(y), B(y)$ are arbitrary functions of $y$, and the solution to the monopole equation (\ref{eq:monopole_eq}) arising from the second part of Theorem \ref{theo_tod1} is the pair $(\mathscr{V},\alpha)$, where
\[
\alpha=\mathscr{V}( pq+A)dy+\frac{\mathscr{V}}{2}dp.
\]
\end{prop}


\subsection{Solution to the $SU(\infty)$--Toda equation}
\label{steps_sec}
The procedure for extracting the corresponding solution to the $SU(\infty)$--Toda equation is given in \cite{Tod_note} 
(see also \cite{LeBrun} and \cite{DT}). It involves finding the coordinates $(X,Y,Z)$ that put the metric (\ref{genh}) in the 
form (\ref{metric_toda}). Given an ASD Einstein metric $(M, g)$ with a
Killing vector $K$
\begin{enumerate}
\item The conformal factor $c:M\rightarrow \R^+$  given by
\[
c={|d{\bf K}+*_g d{\bf K}|_{g}}^{-1/2}
\]
has a property that
the rescaled self--dual derivative of $K$
\[
\vartheta\equiv c^3\Big(\frac{1}{2}(d{\bf K}+*_g d{\bf K})\Big)
\]
is parallel with respect to $\hat{g}=c^2 g$.
The metric $\hat{g}$ is para--K\"ahler with  self--dual 
para--K\"ahler form $\vartheta$, and admits a Killing vector $K$, as
${\mathcal L}_K(c)=0$.
\item
Define a function $Z:M\rightarrow \R$ to be the moment map:
\be
\label{ztilde}
dZ=K\hook \vartheta.
\ee
It is well defined, as the K\"ahler form is Lie--derived along $K$.
\item
Construct the Einstein--Weyl structure of Theorem \ref{theo_tod1}
by factoring $(M, \hat{g})$ by $K$. Restrict the metric $h$
to a surface $Z=Z_0=\mbox{const}$, and construct isothermal coordinates 
$(X, Y)$ on this surface:
\[
\gamma\equiv h|_{Z=Z_0}=e^{U}(dX^2+dY^2), \quad U=U(X, Y, Z_0).
\]
To implement this step chose an orthonormal basis of one--forms
such that $\gamma= {e_1}^2+{e_2}^2$. Now $(X, Y)$ are solutions to the linear
system of 1st order PDEs
\[
(e_1+ie_2)\wedge (dX+idY)=0.
\]
\item Extend the coordinates $(X, Y)$ from the surface $Z=Z_0$ to $\mathcal{W}$. This may
involve a $Z$--dependent affine transformation of $(X, Y)$.
\end{enumerate}
Implementing the steps 1--4 on MAPLE we find that if $A=0$, and $B=B(y)$ 
is arbitrary, then the $SU(\infty)$--Toda solution is given implicitly by
\begin{eqnarray}
\label{toda_implicit1}
 X&=&-\frac{8\mathrm{e}^{-2\int{B(y)dy}}Z^3p}{(Z^2p^2+4)^2},\quad
Y=\int{\mathrm{e}^{-2\int{B(y)dy}}dy}+\frac{\mathrm{e}^{-2\int{B(y)dy}}(-2Z^4p^2+8Z^2)}{(Z^2p^2+4)^2}.\nonumber\\
U&=&\mathrm{ln}\bigg(\frac{(Z^2p^2+4)^3}{64Z^2}\bigg)+4\int{B(y)dy}.
\end{eqnarray}
We can check that this is indeed a solution using the fact that the $SU(\infty)$--Toda equation is equivalent to 
$d\star_h dU=0$. We have also checked by performing a coordinate transformation of (\ref{md_toda}) to the coordinates $(y,p,Z)$.

To simplify the form of (\ref{toda_implicit1}) set
\[
G=\int\exp{\Big(-2\int B(y)dy\Big)}, \quad T=\frac{2Z^2}{Z^2p^2+4}.
\]
Then (\ref{toda_implicit1}) becomes
\[
e^U=\frac{Z^4}{8T^3 (G')^2}, \quad Y=G+G'T\Big(\frac{4T}{Z^2}-1\Big), \quad
X^2=\frac{4T^4(G')^2}{Z^2}\Big(\frac{2}{T}-\frac{4}{Z^2} \Big).
\]
Eliminating $(T, y)$ between these three equations gives one relation between $(X, Y, Z)$ and  $U$ which is our implicit solution.
The elimination can be carried over explicitly if $G=y^k$ for any integer $k$, or if $G=\exp{y}$. In the latter case the solution is given by
\[
4Y^2e^U(e^UX^2-Z^2)^3+(2e^{2U}X^4-3e^UX^2Z^2+Z^4+2Z^2)^2=0.
\]

We can also consider the flat projective structure with $A=B=0$, in which case
the coordinate $p$ can be eliminated between
\[
e^U=\bigg(\frac{(Z^2p^2+4)^3}{64Z^2}\bigg), \quad
X=-\frac{8Z^3p}{(Z^2p^2+4)^2}
\]
by taking a resultant. This yields
%\[
%\mathrm{e}^{4U}X^6+3\mathrm{e}^{3U}X^4Z^2+3\mathrm{e}^{2U}X^2Z^4+\mathrm{e}^UZ^%6+Z^4=0.
%\]
\[
e^U(e^UX^2-Z^2)^3+Z^4=0.
\]
Note that even the flat projective surface can yield a non--trivial solution to the Toda equation; further discussion can be found in section \ref{neat2}.


\subsection{Two monopoles}
The Einstein--Weyl structures $(\mathcal{W},h,\varphi)$ in (\ref{genh}) that we have constructed in Proposition \ref{prop1}
are special, as they belong to the $SU(\infty)$--Toda class. %, and so (as shown by Tod \cite{Tod_toda}) admit a non--null geodesic congruence which has vanishing shear and twist.
The general solution to the $SU(\infty)$--Toda equation depends (in the real analytic category) on 
two arbitrary functions of two variables, but the solutions of the form (\ref{genh}) depend on two functions of one variable. The additional constraints on the solutions can be traced back to the four dimensional ASD conformal structures
which give rise (by the Jones--Tod construction) to (\ref{genh}).
%As discussed above, in addition to their being ASD and Einstein they are  characterised \cite{DM} by a $\beta$--distribution which is parallel with respect to the Levi--Civita connection and ASD in the sense of Calderbank \cite{Cal1}. The corresponding $\beta$--surfaces do not generically intersect with a given $\alpha$--surface, however if they do intersect then they will intersect in curves (null geodesics) which descend to the Einstein--Weyl structures, and give rise to another (in addition to the Tod shear--free, twist--free) geodesic congruence.
In what follows we shall point out how some of the additional structure on $\mathcal{W}$ arises as a couple of solutions to the Abelian monopole equation.

Let us call the solution $(\mathscr{V},\alpha)$ arising in Proposition \ref{prop1} the Einstein monopole, as the resulting conformal class contains an Einstein metric
(\ref{einstein_1}). The second solution $(\mathscr{V}_M, \alpha_M)$ (which we shall call the Maxwell monopole)
arises
as a symmetry reduction of the ASD Maxwell potential
\[
{\mathcal A}=pdx+qdy=-\mathscr{V}_M {\bf K}+\alpha_M,
\]
where ${\bf K}$ is the Killing one--form, and
we find
\[
\mathscr{V}_M=-p\mathscr{V}, \quad \alpha_M=qdy-p\alpha.
\]
\mynote{I want to understand and explain this better.}
\section{The submaximally symmetric case}
\label{neat}
%\mynote{Say something about $\mathrm{Aff}(1)$?}
Choosing a representative connection from the projective class defined by (\ref{eq:submaxODE}), we obtain from (\ref{eq:g4}) an
Einstein metric
\be \label{eq:submax_einstein}
\begin{split}
g=( p^2- xy^2p-y^3q + 4 y^2 )dx^2 + 2(pq +  x^2yp +  xy^2q - 4 xy)dxdy \\
+ (q^2 - x^3p -  x^2yq + 4 x^2)dy^2 + dxdp + dydq
\end{split}
\ee
on $M$, again with $z_0=:p,\,z_1=:q$, having Killing vectors
\[
K_1=x\frac{\p}{\p x} - p\frac{\p}{\p p} - y\frac{\p}{\p y} + q\frac{\p}{\p q},\quad
K_2=x\frac{\p}{\p y} - q\frac{\p}{\p p}, \quad
K_3=y\frac{\p}{\p x} - p\frac{\p}{\p q}.
\]
These are lifts of the projective vector fields corresponding to the $\mathfrak{sl}(2,\R)$ elements
\[
T_1=\begin{pmatrix}\epsilon & 0\\
0 & -\epsilon
\end{pmatrix}
\quad
T_2 = \begin{pmatrix}0 & 0\\
\epsilon & 0
\end{pmatrix}
\quad
T_3 = \begin{pmatrix} 0 & \epsilon\\
0 & 0
\end{pmatrix}.
\]
%\mynote{There is a comment about conjugacy classes to be made here.}

To obtain an example of a Jones--Tod reduction of (\ref{eq:submax_einstein}), we factor by $K_3$. Choosing coordinates
\[
r=\frac{p^2}{ y^2}, \quad z=2\ln( y^2), \quad w=xp+yq,
\]
gives an Einstein-Weyl structure
\begin{eqnarray}
\label{ew_neat}
h&=&-dr^2-2drdw-w(w^2+r-5w+4)dz^2+2(r-w+4)dzdw,\\
\varphi&=&\frac{1}{r-w+4}dr-\frac{3w}{r-w+4}dz-\frac{4}{r-w+4}dw.\nonumber
\end{eqnarray}
The solution to the $SU(\infty)$--Toda equation (\ref{md_toda}) which determines the Einstein-Weyl structure (\ref{ew_neat}) is described by an algebraic curve $f(\mathrm{e}^U,X,Y,Z)=0$ of degree six in $\mathrm{e}^U$ and degree twelve in the other coordinates. 
This solution has been found following the Steps 1-4 in \S\ref{steps_sec}, 
and is given by
\be
\begin{split}
64\mathrm{e}^{6U}X^6(X+Y)^3(X-Y)^3
-92\mathrm{e}^{5U}X^4Z^2(X+Y)^3(X-Y)^3 \\
+48\mathrm{e}^{4U}X^2Z^2(5X^6Z^2-14X^4Y^2Z^2+13X^2Y^2Z^2-4Y^4Z^2+9X^4+27X^2) \\
+8\mathrm{e}^{3U} Z^4(-20X^6Z^2+48X^4Y^2Z^2-36X^2Y^4Z^2+8Y^6Z^2-81X^4-243X^2Y^2) \\
+3\mathrm{e}^{2U}Z^4
(20X^4Z^4-36X^2Y^2Z^4+16Y^4Z^4+108X^2Z^2+216Y^2Z^2+243) \\
+6\mathrm{e}^UZ^8(-2X^2Z^2+2Y^2Z^2-9) +Z^{12}\\ =0.
\end{split}
\nonumber
\ee


Note that the  formulae (\ref{ew_neat}) are independent of the coordinate $z$, and therefore have a symmetry. This was unexpected because there is no other symmetry of $(M,g)$ that commutes with $K_3$. However, it is possible for symmetries to appear in the Einstein-Weyl structure without a corresponding symmetry of the ASD conformal structure. This can be seen from the general formula (\ref{eq:monopole_correspondence}); the function $\mathscr{V}$ may depend on the coordinate $z$ so that $g$ depends on $z$ even though $h$ does not. For example, the Gibbons-Hawking metrics \cite{GH} give a trivial Einstein--Weyl structure with the maximal symmetry group, but the four-metric is in general not so symmetric. Our discovery of this unexpected symmetry motivated a more concrete description of a symmetry of a Weyl structure.

\begin{defi}
An infinitesimal symmetry of a Weyl structure $(\mathcal{W},\mathscr{D},[h])$ is a vector field $\mathcal{K}$ which is both an affine vector field with respect to the connection\footnote{Recall that an affine vector field of a connection $\mathscr{D}$ is one which preserves its components, i.e. $\mathcal{L}_\mathcal{K}\Gamma^i_{jk}=0$.} $\mathscr{D}$ and a conformal Killing vector with respect to the conformal structure $[h]$.
\end{defi}

\begin{prop}
\label{ewsymprop}
Given an infinitesimal symmetry $\mathcal{K}$ of a Weyl structure $(\mathcal{W},\mathscr{D},[h])$ in dimension $N$, and a representative $h\in[h]$ such that $\mathscr{D}h=\varphi\otimes h$, there exists a smooth function $f:W\rightarrow \R$ such that
\be
\label{EWsym}
\mathcal{L}_\mathcal{K}h=fh,\qquad\mathcal{L}_\mathcal{K}\varphi=\frac{1}{N}d[\mathcal{K}\hook d(\mathrm{ln}(\mathrm{det}(h)))].
\ee
\end{prop}
\noindent\textbf{Proof.} The first equation follows immediately from the fact that $\mathcal{K}$ is a conformal Killing vector of $h$. It remains to evaluate the Lie derivative of the one--form $\varphi$ along the flow of $\mathcal{K}$ given that $\mathcal{L}_\mathcal{K}h=fh$ and $\mathcal{L}_\mathcal{K}\Gamma^i_{jk}=0$, where $\Gamma^i_{jk}$ are the components of the connection $\mathscr{D}$. We do this by considering the Lie derivative of $\mathscr{D}h$:
\begin{align*}
\mathcal{L}_\mathcal{K}(\mathscr{D}_ih_{jk}) &= \mathcal{L}_\mathcal{K}(\p_ih_{jk})-\mathcal{L}_\mathcal{K}(\Gamma^l_{ji}h_{lk}+\Gamma^l_{ki}h_{jl}) \\
&= \mathcal{L}_\mathcal{K}(\p_ih_{jk}) - f(\Gamma^l_{ji}h_{lk}+\Gamma^l_{ki}h_{jl}).
\end{align*}
Now
\begin{align*}
\mathcal{L}_\mathcal{K}(\p_ih_{jk}) &= \mathcal{K}^l\p_l\p_ih_{jk}+(\p_i\mathcal{K}^l)\p_lh_{jk}+(\p_j\mathcal{K}^l)\p_ih_{lk} + (\p_k\mathcal{K}^l)\p_ih_{jl} \\
&= \p_i[\mathcal{K}^l\p_lh_{jk}+(\p_j\mathcal{K}^l)h_{lk}+(\p_k\mathcal{K}^l)h_{jl}] - (\p_i\p_j\mathcal{K}^l)h_{lk} - (\p_i\p_k\mathcal{K}^l)h_{jl}.
\end{align*}
The term with square brackets is just
\[
\p_i(\mathcal{L}_\mathcal{K}h_{jk})=\p_i(fh_{jk})=f\p_ih_{jk}+\p_ifh_{jk},
\]
so we have
\[
\mathcal{L}_\mathcal{K}(\mathscr{D}_ih_{jk})=f\mathscr{D}_ih_{jk}+\p_ifh_{jk}- (\p_i\p_j\mathcal{K}^l)h_{lk} - (\p_i\p_k\mathcal{K}^l)h_{jl}.
\]
Setting this equal to $\mathcal{L}_\mathcal{K}(\varphi_ih_{jk})=(\mathcal{L}_\mathcal{K}\varphi_i)h_{jk}+f\varphi_ih_{jk}$ and cancelling $f\varphi_ih_{jk}$ with $f\mathscr{D}_ih_{jk}$, we find
\begin{eqnarray}
(\mathcal{L}_\mathcal{K}\varphi_i)g_{jk} &=& \p_ifh_{jk} - (\p_i\p_j\mathcal{K}^l)h_{lk} - (\p_i\p_k\mathcal{K}^l)h_{jl}\nonumber \\
\label{liederivom}
\implies\ \mathcal{L}_\mathcal{K}\varphi_i &=& \p_if - \frac{2}{N}\p_i\p_j\mathcal{K}^j.
\end{eqnarray}
Finally, we note that
\[
\p_i\p_j\mathcal{K}^j=\frac{N}{2}\p_if-\frac{1}{2}\p_i[\mathcal{K}\hook d(\mathrm{ln}(\mathrm{det}(h))].
\]
This follows from tracing the expression $\mathcal{L}_\mathcal{K}h_{ij}=fh_{ij}$:
\be
\begin{gathered}
\nonumber
\mathcal{L}_\mathcal{K}h_{ij} = \mathcal{K}^k\p_kh_{ij} + (\p_i\mathcal{K}^k)h_{kj} + (\p_j\mathcal{K}^k)h_{ik} = fh_{ij} \\
\implies \quad \mathcal{K}^kh^{ij}\p_kh_{ij} + 2\p_k\mathcal{K}^k = Nf \\
\implies \quad 2\p_i\p_k\mathcal{K}^k = N\p_if -  \p_i(\mathcal{K}^kh^{jl}\p_kh_{jl})
\end{gathered}
\ee
and recalling that $h^{jl}\p_kh_{jl}=\p_k\mathrm{ln}(\mathrm{det}(h))$.
Substituting into (\ref{liederivom}) then yields the result.
\begin{flushright}
$\square$
\par\end{flushright}

We can easily verify the invariance of (\ref{EWsym}) under Weyl transformations. Let $(\hat{h},\hat{\varphi})$ be a new metric and one--form related to the old ones by (\ref{weyl_tr}). Then
\[
\mathcal{L}_\mathcal{K}\hat{\varphi}=\mathcal{L}_\mathcal{K}\varphi + 2\mathcal{K}\hook d\mathrm{ln}(\rho)
\]
from (\ref{weyl_tr}), and from (\ref{EWsym}) we have
\begin{align*}
\mathcal{L}_\mathcal{K}\hat{\varphi} & = \frac{1}{N}d[\mathcal{K}\hook d(\mathrm{ln}(\rho^{2N}\mathrm{det}(h)))] \\
& = \frac{1}{N}d[\mathcal{K}\hook d(\mathrm{ln}(\mathrm{det}(h)))] + \frac{2N}{N}\mathcal{K}\hook d\mathrm{ln}(\rho) \\
& = \mathcal{L}_\mathcal{K}\varphi + 2\mathcal{K}\hook d\mathrm{ln}(\rho),
\end{align*}
as above. Note that the function $f$ in (\ref{EWsym}) will change according to
\[
\hat{f}=f+2\mathcal{K}\hook d\mathrm{ln}\rho.
\]

In the case of the Weyl structure (\ref{ew_neat}), the infinitesimal symmetry is
\[
\mathcal{K}=\frac{\p}{\p v}.
\]
Since we have chosen a scale such that $\mathcal{K}$ is in fact a Killing vector of $h$, we have that $\mathcal{K}\hook d(\mathrm{ln}(\mathrm{det}(h))=0$, so the one--form $\varphi$ is also preserved by $\mathcal{K}$. This is consistent with the fact that it has no explicit $v$--dependence.
\section{The model case}
\label{sec:model}
In the following section we discuss the four--manifold $(M,g)$ obtained from the maximally symmetric flat 
projective surface $N=\RP^2$. In this case, $g$ is not only almost para--K\"ahler but in fact para--K\"ahler, since the symplectic form $\Omega$ is parallel with respect to the Levi--Civita connection of $g$. Chosing a representative connection with
$\Gamma_{A'B'}^{C'}=0$ gives $g$ as
\be
\label{special_ein}
g=dz_{A'}\odot dx^{A'}+ z_{A'}z_{B'} dx^{A'}\odot dx^{B'}.
\ee

We begin by discussing the  conformal structure
of (\ref{special_ein}), both explicitly and in terms of its twistor space. We then note a pseudo--hyper--Hermiticity property which is unique to the model case, and find some special structure on the twistor space. Finally, we present a classification of the Einstein--Weyl structures which can be obtained by Jones--Tod factorisation, and exhibit an explicit example of such a factorisation from the twistor perspective, reconstructing the conformal structure on $\mathcal{W}$ from minitwistor curves.
\subsection{Conformal Structure on $M$}
\label{model_conf}
Recall that points $P\in\R^3$ and $L\in\R_3$ respectively define lines and planes in $\R^3$ which are preserved by multiplication of $P,L$ by members of $\R^*$, and that these lines and planes respectively descend to points $[P]$ and lines $[L]$ in $\RP^2$. In what follows we will drop the square brackets and understand points $[P]\in\RP^2$ and lines $[L]\in\RP^{2*}$ to be represented by vectors $P\in\R^3$ and $L\in\R_3$. Let $M\subset \RP^2\times {\RP^2}^*$ be the set of non--incident pairs 
$(P, L)$.
\begin{prop}
\label{prop_cone}
Two pairs $(P, L)$ and $(\tP, \tL)$ are null--separated
with respect to the conformal structure (\ref{special_ein})
if there exists
a line which contains the three points $(P, \tP, L\cap \tL)$. 
\end{prop}
{\bf Proof.}
First not that the  null condition of Proposition \ref{prop_cone}
defines a co--dimension one cone in $TN$: 
generically there is no line through three given points. To make explicit the condition for such a line to exist, consider two pairs  $(P, L)$ and $(\tP, \tL)$ 
of non--incident points and lines. By thinking of $L,\tL$ as normal vectors to planes in $\R^3$, we see that $L+t\tL$ is a plane which intersects $L$ and $tL$ at their intersection, thus defining a line in $\RP^2$ which intersects the lines $L,\tL\subset\RP^2$ at their intersection.

If $P,\tP,L\cap\tL$ are co--linear then there exists $t$ such that both $P$ and $\tP$ lie on $L+t\tL$, i.e.
\be
\label{dm1}
P\cdot (L+t\tL)=0,  \quad
\tP\cdot (L+t\tL)=0.
\ee 
Eliminating $t$ from 
(\ref{dm1}) gives
\[
(P\cdot L)(\tP\cdot \tL)-(\tP\cdot L)(P\cdot \tL)=0.
\]
Setting $\tP=P+dP, \tL=L+dL$ yields a metric 
$g$  representing the conformal structure
\[
g=\frac{dP\cdot dL}{P\cdot L}-\frac{1}{(P\cdot L)^2}(L\cdot dP)(P\cdot dL).
\]
We can use the normalisation $P\cdot L=1$, so that $P\cdot dL=-L\cdot dP$,
and
\be
\label{dm_metric}
g={dP\cdot dL}+(L\cdot dP)^2.
\ee
We take affine coordinates 
\be
\label{DM_parameter}
P=[x^{A'}, 1],\quad L=[z_{A'}, 1-x^{A'}z_{A'}]
\ee 
with a normalisation $P\cdot L=1$ to recover the metric (\ref{special_ein}).
\koniec
\subsection{Twistor space of $M$}
\label{twist_SSS}
To understand $(M,[g])$ from the twistor perspective, we need to move to the complex picture. In what follows, we will view $M$ as the set of non--incident pairs in $\CP^2\times\CP^{2*}$. Let $F_{12}(\mathbb{C}^3)\subset \CP^2\times {\CP^2}^*$ be set of incident pairs 
$(p, l)$, so that $p\cdot l=0$. Note that, since $l$ and $p$ correspond to planes and lines in $\mathbb{C}^3$ respectively, and since $p\cdot l=0$ is the condition for the line $p$ lying in the plane $l$, $F_{12}(\mathbb{C}^3)$ coincides with the flag manifold of type $(1,2)$ in $\mathbb{C}^3$, i.e. the collection of one-- and two--dimensional vector subspaces $(p,l)$ in $\mathbb{C}^3$ such that $p\subset l$. This is the twistor space of $(M, g)$.
A $\CP^1$ embedding corresponding to a point $(P, L)\in M$
consists of all lines $l$ thorough $P$, and all points
$p=l\cap L$:
\be
\label{dm2}
P\cdot l=0, \quad p\cdot L=0, \quad p\cdot l=0.
\ee

Let $(P, L)$ and $(\tP, \tL)$ be points in $M$. These uniquely define a point $p=L\cap\tL\in\CP^2$ and line $l\subset\CP^2$ such that $P,\tP\in l$ given by
\[
p=L\wedge \tL, \quad l=P\wedge \tP,
\]
where $[L\wedge\tL]^{\alpha}=\epsilon^{\alpha\beta\gamma}L_{\alpha}\tL_{\beta}$ etc.
The pair $(p,l)$ lies in $F_{12}$ if $p$ lies on $l$, i.e. if $p\cdot l=0$, so that $(P, L)$ and $(\tP, \tL)$ are null--separated with respect to the conformal structure
(\ref{dm1}). Then $(p,l)$ is the intersection of the $\CP^1$ embeddings corresponding to $(P, L)$ and $(\tP, \tL)$. %The contact structure on $F_{12}$ is $(l\cdot dp-p\cdot dl)/2=p\cdot dl$. \mynote{Contact structure on $F_{12}$?}

We shall now give an explicit parametrisation of twistor lines, and show how 
the metric (\ref{dm_metric}) arises from the Penrose condition 
\cite{penrose, ward}.
Let $P\in \CP^2$. The corresponding $l\in {\CP^2}^*$ is represented by some normal vector which is perpendicular to $P$ in $\mathbb{C}^3$, i.e.
\be \label{eq:l=Pwedgepi}
l=P\wedge \pi, \quad \mbox{where}\quad  \pi\sim a\pi+b P,
\ee
where $a\in \mathbb{C}^*, b\in \mathbb{C}$. Thus $\pi$ parametrises a projective line $\CP^1$,
and by making a choice of $b$ we can take
$
\pi=[\pi^{0'}, \pi^{1'}, 0], 
$ where $\pi^{A'}=[\pi^{0'}, \pi^{1'}]\in \CP^1$. The constraint $P\cdot l=0$ now holds.
To satisfy the remaining constraints in (\ref{dm2}) we take
\be \label{eq:p=Lwedgel}
p=L\wedge l=(L\cdot\pi)P-(L\cdot P)\pi.
\ee
Substituting (\ref{DM_parameter}) gives 
the corresponding twistor line parametrised by $[\pi]\in\CP^1$ 
\be
\label{sl3curves}
p^{\alpha}=[(z\cdot \pi)x^{A'}-\pi^{A'}, z\cdot \pi], \quad l_\alpha=[\pi_{A'}, -\pi\cdot x],
\ee
where the spinor indices are raised and lowered with $\epsilon^{AB}$ and its inverse, and  $z\cdot x\equiv z_{A'}x^{A'}$. 

We shall now derive the expression for the conformal structure using the Nonlinear Graviton prescription described in section \ref{sec:ASD'ty}. To compute the normal bundle, let $([l(\pi, P, L)], 
[p(\pi, P, L)])$
be the twistor line corresponding to a point $m=(P, L)$ in $M$. The vector in the direction of a nearby point $(P+\delta P,L+\delta L)$ corresponds to the neighbouring line $([l+\delta l], [p+\delta p])$, where from (\ref{eq:l=Pwedgepi}) and (\ref{eq:p=Lwedgel}) we have
\[
\delta l=\delta P\wedge \pi, \quad
\delta p= (\delta L\cdot \pi)P+(L\cdot\pi) \delta P-\delta (L\cdot P)\pi.
\]

The lines  $(l+\delta l, p+\delta p)$  and $(l, p)$ define a section of the normal bundle to $(l,p)$, which has a zero if and only if this vector is null. Vanishing of the section is equivalent to intersection of the two lines, and this happens if there exists some $[\pi]$ such that $l+\delta l\sim l$ and $p+\delta p\sim p$ (note that the intersection point, if it exists, is unique, since projective lines in $\RP^2$ cannot meet more than once). We thus find
\[
l+\delta l\sim l \quad \iff \quad\pi\sim\delta P=[\delta x^1, \delta x^2, 0].
\]
And $p+\delta p\sim p \quad \iff$
\[
0=p\wedge \delta p=(L\cdot \pi)^2P\wedge \delta P-(L\cdot P)
(\delta L\cdot \pi)\pi\wedge P-(L\cdot \pi)\delta (L\cdot P)P\wedge \pi-
(L\cdot P)(L\cdot \pi) \pi\wedge \delta P.
\]
Substituting $\pi\sim\delta P$, we find that all terms on the RHS are proportional to $P\wedge \delta P=[0, 0, x\cdot dx]$, with
\[
(L\cdot \delta P)^2-(L\cdot \delta P)\delta(L\cdot P)+(L\cdot P)(\delta L\cdot \delta P)=0.
\]
Setting $L\cdot P=1$ this gives the conformal structure 
(\ref{dm_metric}).
%\mynote{Is it possible to write this in the language of vectors, like we do in the mini--twistor section? To make it about a section of the normal bundle rather than the neighbouring line? I guess you'd divide (\ref{sl3curves}) by $\pi^{0'}$ like we do in the minitwistor section.}
%%%%%%%%%%%%%%%%%%%%%%%%%%%%%%%%%%%%%%%%%%%%%%%%%%%%%%%%%%%%%%
\subsection{Pseudo--hyper--Hermitian structure on $M$}
A pseudo--hyper--complex structure on a four manifold $M$ is a triple of endomorphisms
$I_1, I_2, I_3$ of $TM$ which satisfy
\[
I_1^2=-Id, \quad I_2^2=I_3^2=Id, \quad I_1I_2I_3=Id,
\]
and such that $c_1I_1+c_2I_2+c_3I_3$ is an integrable complex structure for any point 
on the hyperboloid $c_1^2-c_2^2-c_3^2=1$.
A neutral signature metric $g$ on a pseudo--hyper--complex four--manifold is pseudo--hyper--Hermitian
if
\[
g(\xi, \xi)=g(I_1\xi, I_1\xi)=-g(I_2\xi, I_2\xi)=-g(I_3\xi, I_3\xi)
\]
for any vector field $\xi$ on $M$.

Given a pseudo--hyper--complex structure $(M,\{I_1,I_2,I_3\})$ and any vector field $\xi$ on $M$, the frame $(\xi,I_1\xi,I_2\xi,I_3\xi)$ defines a conformal structure on $M$. With a natural choice of orientation
which makes the fundamental two--forms of $I_1, I_2, I_3$  self--dual, 
this conformal structure is anti--self--dual.

Let $\Sigma^{A'B'}$ be a basis of SD two--forms on $M$. The following result is proved in \cite{D99} (see also \cite{boyer}) in the Riemannian (i.e. hyper--complex) case.
\begin{theo}[\cite{D99}]\label{thm:D99}
A four--manifold $M$ equipped with a neutral signature metric $g$ is pseudo--hyper--Hermitian if there exists a one--form $A$ depending only on $g$ such that
\[
d\Sigma^{A'B}+A\wedge \Sigma^{A'B'} = 0.
\]
\end{theo}
\noindent In fact, this condition is necessary and sufficient for hyper--Hermiticity \cite{D99,boyer}. Given some $(M,g)$ which is conformally ASD, it can also be shown (see Lemma 2 in \cite{D99} and Theorem  7.1 in \cite{Cal2}) that a lack of vertical $\p/\p \pi$ terms in the twistor distribution (\ref{eq:twistor_dist}) implies hyper--Hermiticity of $(M,g)$. %\mynote{This is not clear to me from a brief look at \cite{D99,Cal2}...}
\begin{prop}
\label{propHH}
The Einstein metric (\ref{special_ein}) is pseudo--hyper--Hermitian.
\end{prop}
\noindent
{\bf Proof.}
The null frame for the 4-metric is
\be \label{eq:null_frame}
e^{0A'}=dx^{A'}, \quad e^{1A'}=dz^{A'}+z^{A'}(z\cdot dx), \quad\mbox{so that}\quad
g=\varepsilon_{A'B'}\varepsilon_{AB}e^{AA'}e^{BB'}.
\ee
Thus the forms $\Sigma=dx^{0'}\wedge dx^{1'}$ and $\Omega=dz_{A'}\wedge dx^{A'}$ are ASD. The basis of SD two forms is spanned by
\[
dx\wedge dq+ q^2 dx\wedge dy,\quad
dx\wedge dp-dy\wedge dq+2 pq dx\wedge dy,
\quad
-dy\wedge dp+ p^2 dx\wedge dy
\]
or, in a more compact notation, by
$\Sigma^{A'B'}=dx^{(A'}\wedge dz^{B')}+z^{A'}z^{B'} \Sigma$.
We can verify that
\be
\label{lie_form}
d\Sigma^{A'B'}+2{{\mathcal{A}}}\wedge\Sigma^{A'B'}=0,
\ee
where ${{\mathcal A}}= z_{A'}dx^{A'}$ is such that $d{{\mathcal{A}}}= \Omega$,
so from Theorem \ref{thm:D99} we have that $M$ carries a hyper--Hermitian structure, and in fact the corresponding ASD Maxwell field $d\mathcal{A}=\Omega$ coincides with the one arising from the para--K\"ahler structure on $M$ via (\ref{beta_eq}).

Alternatively, note that the twistor distribution (\ref{eq:twistor_dist}), having chosen the basis (\ref{eq:null_frame}), is given by
\be
\label{tdistribution}
L_{0}=\pi\cdot\frac{\p}{\p x}+(z\cdot\pi) z\cdot\frac{\p}{\p z}, \quad
L_{1}=\pi\cdot\frac{\p}{\p z},
\ee
which have no vertical terms. We can easily verify that it is Frobenius integrable, as $[L_{0}, L_{1}]=-(\pi\cdot z)L_{1}$. The SD part of the
spin connection is given in terms of ${\mathcal A}$ as
$\Gamma_{AA'B'C'}=-2{\mathcal A}_{A(B'}\varepsilon_{C')A'}$.
\koniec
In the next section we shall show how to encode
${\mathcal A}$ in the twisted--photon Ward bundle over the twistor space
of $(M, g)$.

\subsection{A line bundle over the twistor space of $M$}
%\mynote{Still not 100 per cent sure I have got this right. Does it matter what $\mathbb{C}^*$ bundle over $\mathscr{T}$ we are talking about? What would happen if we instead regarded $\mathscr{T}$ as $\PP(T\CP^2)$, and used our parametrisation (\ref{sl3curves}) but projectivised the $l$ part instead of the $p$ part?}
Ward \cite{wardtf} shows that there is a correspondence between ASD Maxwell potentials on $M$ and $\mathbb{C}^*$ bundles over $\mathscr{T}$ which are trivial on twistor lines\footnote{Recall that a principal bundle $\PP\rightarrow\mathscr{T}$ with structure group $G$ is trivial if there exists a map $\chi:\PP\rightarrow\mathscr{T}\times G$. Let $\{\chi_\alpha:\mathcal{U}_\alpha\rightarrow\mathscr{T}\times G\}$ be local trivialisations related by transition functions $F_{\alpha\beta}=\chi_\beta\circ\chi^{-1}_\alpha\in G$. If $F_{\alpha\beta}=f_\beta f_\alpha^{-1}$ for some splitting elements $f_\alpha,f_\beta\in G$ on $\mathcal{U}_\alpha,\mathcal{U}_\beta$ respectively, then there exist $\tilde{\chi}_\alpha=f_\alpha^{-1}\circ\chi_\alpha$ and $\tilde{\chi}_\beta=f_\beta^{-1}\circ\chi_\beta$ such that $\tilde{F}_{\alpha\beta}=f_\beta^{-1}f_\beta f_\alpha^{-1}f_\alpha=Id$, so that the bundle is trivial.}. The purpose of this section is to uncover the $\mathbb{C}^*$ bundle over $\mathscr{T}$ corresponding to the ASD Maxwell potential $\mathcal{A}=z_{A'}dx^{A'}$ on $M$.

The twistor space $F_{12}$ described in \S\ref{twist_SSS} can be identified with the projectivised tangent bundle ${\PP(T\CP^{2*})}$ of the minitwistor space of the flat projective structure, since a point $(p,l)$ in $F_{12}\subset\CP^2\times\CP^{2*}$ consists of a point $l\in\CP^{2*}$, and a line $p\subset\CP^{2*}$ through $l$ which we can identify with a direction in the tangent space $T_l\CP^{2*}$. Thus the twistor space of $M$ is the correspondence space (in a twistorial sense) of $\CP^2$ and its twistor space ${\CP^2}^*$. An obvious $\mathbb{C}^*$ bundle over ${\PP(T\CP^{2*})}$ is $T\CP^{2*}$.

\begin{prop}
The $\mathbb{C}^*$ bundle $T\CP^{2*}\rightarrow\PP(T\CP^{2*})=F_{12}$ is trivial on twistor lines, and corresponds via Ward's twisted photon construction to the ASD Maxwell potential $\mathcal{A}$ on $M$.
\end{prop}
 
\noindent {\bf Proof.} There are many open sets needed to cover
$\PP(T\CP^{2*})$, but it is sufficient to consider two:
$\mathcal{U}$, where $(l_1, \neq 0, p^2\neq 0)$, and $(l_2/l_1, l_3/l_1, p^3/p^2)$ are coordinates, and $\widetilde{\mathcal{U}}$ where
$(l_1\neq 0, p^3\neq 0)$, and  $(l_2/l_1, l_3/l_1, p^2/p^3)$
are coordinates. Now consider the total
space of $T\CP^{2*}$% (or perhaps it is $T\CP^{2*}$ tensored with some power of the canonical bundle to make it trivial on twistor lines) \mynote{by the canonical bundle we mean the bundle of volume forms.}
, and restrict it to the intersection of (pre--images in
$T\CP^{2*}$
of) $\mathcal{U}$ and $\widetilde{\mathcal{U}}$. The coordinates on $T\CP^{2*}$ in these
region are $(l_2/l_1, l_3/l_1, p^2/p^1, p^3/p^1)$, and the fibre
coordinates over $\tau$ over $\mathcal{U}$ and $\tilde{\tau}$ over 
$\widetilde{\mathcal{U}}$ are related by\footnote{Here we are following Ward \cite{wardtf},
and thinking of a $\mathbb{C}^*$ bundle.}
\[
\tilde \tau=\exp(F)\tau, \quad\mbox{where}\quad  
F=\ln{(p_2/p_3)}.
\]

Now we follow the procedure of \cite{wardtf}: restrict $F$ to a twistor line,
and split it.
The holomorphic splitting is $F=f-\widetilde{f}$, where
$f=\ln{(p_2)}$ is holomorphic in the pre--image of $\mathcal{U}$ in the correspondence space, and 
$\widetilde{f}=\ln{(p_3)}$ is holomorphic in the pre--image of
$\widetilde{\mathcal{U}}$. Note that $F$ is a twistor  function, but 
$f, \widetilde{f}$ are not. Therefore
$L_{A}F=0$, where the twistor distribution $L_{A}$
is given by (\ref{tdistribution}). This implies that $L_{A}f=L_{A}\widetilde{f}$. Since each side of this equation is holomorphic on an open subset of $\CP^1$, and since $\CP^1$ can be covered with such subsets, both sides are globally holomorphic and therefore linear in $\pi^{A'}$ by the Liouville theorem. Hence
\[
L_{A}f=L_{A}\widetilde{f}=\pi^{A'}\mathcal{A}_{AA'}
\]
for some one--form $\mathcal{A}$ on $M$.

To construct this one--form recall the parametrisation
of twistor curves (\ref{sl3curves}). This gives
\[
f=\ln{(z\cdot\pi)}, \quad\widetilde{f}=\ln{((z\cdot\pi)x^{1'}-\pi^{1'})}
\]
and
\[
L_{1}(f)=L_{1}(\widetilde{f})=0, \quad
L_{0}(f)=L_{0}(\widetilde{f})=\pi\cdot z.
\]
Therefore ${\mathcal A}_{1A'}=0, {\mathcal A}_{0A'}=z_{A'}$
which gives ${\mathcal A}=z_{A'}dx^{A'}$, and $d{\mathcal A}$
is indeed the ASD para--K\"ahler structure $\Omega$.
\koniec
%which I THINK is 
%\be
%\label{ptp}
%p=[p^1, p^2, p^3], \quad Z=-\frac{p^3}{p^2}=\frac{z^0+X %z^1}{-1+(z^0+X z^1)x^1}.
%\ee
\subsection{Factoring the model to Einstein-Weyl}
\label{neat2}
As stated above, we expect distinct Einstein--Weyl structures if we factor $M$ by conformal killing vectors which are not conjugate with respect to an isometry \cite{PT}. We can thus classify the Einstein--Weyl structures obtainable from the model by first classifying its symmetries up to conjugation.
\begin{prop}
The non--trivial Einstein--Weyl structures obtainable from the ASD Einstein metric (\ref{special_ein}) by the Jones--Tod correspondence consist of a two--parameter family, and two additional cases which do not belong to this family.
\end{prop}
\noindent
{\bf Proof. }Since we have an isomorphism between the Lie algebra of projective vector fields on $(N,[\nabla])$ and the Lie algebra of Killing vectors on $(M,g)$, the problem of classifying the symmetries of (\ref{special_ein}) is reduced to a classification of the infinitesimal  projective symmetries of $\RP^2$, i.e. the near--identity elements of $SL(3,\R)$, up to conjugation.

Non--singular complex matrices are determined up to similarity by their Jordan normal form (JNF). While real matrices do not have such a canonical form, all of the information they contain is determined (up to similarity) by the JNF that they would have if they were considered as complex matrices. Thus we can still discuss the JNF of a real matrix, even if it cannot always be obtained from the real matrix by a real similarity transformation. The possible non--trivial Jordan normal forms of matrices in $SL(3,\R)$ are shown below.
\[
\begin{pmatrix}l_1 & 0 & 0\\
0 & l_2 & 0\\
0 & 0 & 1/l_1l_2
\end{pmatrix}
\quad
\begin{pmatrix} l & 0 & 0\\
0 &  l & 0\\
0 & 0 & 1/ l^2
\end{pmatrix}
\quad
\begin{pmatrix} l & 1 & 0\\
0 &  l & 0\\
0 & 0 & 1/ l^2
\end{pmatrix}
\quad
\begin{pmatrix}1 & 1 & 0\\
0 & 1 & 0\\
0 & 0 & 1
\end{pmatrix}
\quad
\begin{pmatrix}1 & 1 & 0\\
0 & 1 & 1\\
0 & 0 & 1
\end{pmatrix}
\]

It is possible that two matrices in $SL(3,\R)$ with the same JNF may be related by a complex similarity transformation, and thus not conjugate in $SL(3,\R)$. However, if the JNF is a real matrix, then the required similarity transformation just consists of the eigenvectors and generalised eigenvectors of the matrix, which must also be real since they are defined by real linear simultaneous equations. This means we only have to worry about matrices with complex eigenvalues, and since these occur in complex conjugate pairs, they will only be a problem when we have three distinct eigenvalues.

In this case, we can always make a real similarity transformation such that the matrix is block diagonal, with the real eigenvalue in the bottom right. Then we have limited choice from the $2\times 2$ matrix in the top left. Let us parametrise such a $2\times 2$ matrix by $a,\,b,\,c,\,d\in\mathbb{R}$ as follows:
\[
\quad
\begin{pmatrix}1+a\epsilon & b\epsilon \\
c\epsilon & 1+d\epsilon 
\end{pmatrix}.
\]
This has characteristic polynomial
\[
\chi( l)= l^2-(2+\epsilon(a+d)) l+1+(a+d)\epsilon+(ad-bc)\epsilon^2.
\]
Evidently the important degrees of freedom are $a+d$ and $ad-bc$, so we can use these to encode every near--identity element of the class with three distinct eigenvalues. The bottom--right entry will be determined by our choice of $a+d$ and $ad-bc$.

Taking a projective vector field on $\mathbb{RP}^2$, we can find the corresponding Killing vector of (\ref{special_ein}) using (\ref{eq:kvf_from_pvf}), and factor to Einstein--Weyl using (\ref{EWgen}). We find by explicit calculation that vector fields arising from the second and fourth JNFs above give trivial Einstein--Weyl structures, so restricting to the non--trivial cases we have a two--parameter family of Einstein--Weyl structures coming from the first class, and two additional Einstein--Weyl structures coming from the third and fifth, as claimed.
\koniec


%\mynote{Give the example corresponding to the mini--twistor factorisation below.}
\subsection{An example of the mini--twistor correspondence}
\label{mini_twistor}
Below we investigate a one--parameter subfamily of the two--parameter family. We use the holomorphic vector field on 
the twistor space
$F_{12}$  (see \S\ref{twist_SSS})
corresponding to the chosen symmetry, and reconstruct the conformal structure $[h]$ on $\mathcal{W}$ using minitwistor curves 
(in the sense of \cite{hitchin})
on the space of orbits. Take $a\in \mathbb{R}$ and
\be
\label{modelK}
K=P^1\frac{\partial}{\partial P^1} - L_1\frac{\partial}{\partial L_1}+aP^2\frac{\partial}{\partial P^2} - aL_2\frac{\partial}{\partial L_2},
\ee
%corresponding to the matrix
%\[
%M=
%\begin{pmatrix}1+\epsilon & 0 & 0\\
%0 & 1+a\epsilon & 0\\
%0 & 0 & 1
%\end{pmatrix}.
%\]
%Note that we have chosen inhomogeneous coordinates, so $M$ need not be in $SL(3)$.
In order to preserve the relations
\[
p\cdot L=0,\quad P\cdot l=0,\quad p\cdot l=0, 
\]
the corresponding holomorphic action on $(p,l)$ must be $p\mapsto gp$, $l\mapsto g^{-1}l$, thus the holomorphic vector field $ K_\mathscr{T}$ on $F_{12}$ is
\[
 K_\mathscr{T}=p^1\frac{\partial}{\partial p^1} - l_1\frac{\partial}{\partial l_1}+ap^2\frac{\partial}{\partial p^2} - al_2\frac{\partial}{\partial l_2}.
\]

In order to factor $F_{12}$ by this vector field, we must find invariant minitwistor coordinates $(Q,R)$. In addition to satisfying $ K_\mathscr{T}(Q)= K_\mathscr{T}(R)=0$, they must be homogeneous of degree zero in $(P,L)$. We choose
\[
Q=\frac{p^1l_1}{p^2l_2},\quad R=\frac{(l_1)^a}{l_2(l_3)^{a-1}}.
\]
Substituting in our parametrisation (\ref{sl3curves}) and using the freedom to perform a Mobius transformation on $\pi$, we obtain
\begin{align}
\label{QR}
Q &= \frac{(\lambda t-r-1)\lambda}{w\lambda+\lambda-\frac{rw}{t}}\\
R &= \lambda^a\Big(-\lambda-\frac{w}{t}\Big)^{1-a},\nonumber
\end{align}
where we have defined $\lambda=\pi_{0'}/\pi_{1'}$, and the Einstein--Weyl coordinates
\[
r=xp,\quad w=yq, \quad t=x^aq.
\]
Note these are invariants of the Killing vector (\ref{modelK}).

Next we wish to use these minitwistor curves to reconstruct the conformal structure of the Einstein-Weyl space. In doing so we follow \cite{PT}. The tangent vector field to a fixed curve is given by
\[
T=\frac{\p Q}{\p \lambda} \frac{\p}{\p Q} + \frac{\p R}{\p \lambda} \frac{\p}{\p R},
\]
Hence we can write the normal vector field as
\begin{align*}
\xi_\mathbb{N} &=dQ\frac{\p}{\p Q} + dR\frac{\p}{\p R} \enskip \mathrm{mod}\, T\\
&= \bigg(\frac{\p R}{\p \lambda}\bigg)^{-1}\bigg(dQ\frac{\p R}{\p \lambda}-dR\frac{\p Q}{\p \lambda}\bigg)\frac{\p}{\p Q},
\end{align*}
where
\[
dQ=\frac{\p Q}{\p r}dr + \frac{\p Q}{\p w}dw + \frac{\p Q}{\p t}dt
\]
and similarly for $dR$. Calculating $\xi_\mathbb{N}$ using (\ref{QR}), we find
\[
\xi_\mathbb{N}\propto(\eta_1\lambda^2+\eta_2\lambda+\eta_3)\frac{\p}{\p Q},
\]
where
\begin{align*}
\eta_1 &= t^2(w+1)dt-t^3dw, \\
\eta_2 &= -2trwdt + t^2(a+2r)dw -t^2dr, \\
\eta_3 &= rw(1+r)dt - tr(1+r)dw - atwdr.
\end{align*}
The discriminant of this quadratic in $\lambda$ then gives a representative $h\in[h]$ of our conformal structure:
\be
\begin{split}
h=4(r^2w+rw^2+rw)dt^2 - 4tw(a(w+1)+r)dtdr + 4tr(r-aw+2w+1)dtdw \\
-t^2dr^2 + 2t^2(2aw+a+2r)dwdr - t^2(a^2+4r(a-1))dw^2.
\end{split}
\ee
This is the same conformal structure that we obtain by Jones-Tod factorisation of the metric (\ref{special_ein}) by (\ref{modelK}) using the formula (\ref{EWgen}).