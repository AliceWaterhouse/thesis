%!TEX root = ../thesis.tex
%*******************************************************************************
%****************************** Chapter 3: intro2 *********************************
%*******************************************************************************

\chapter{The projective to Einstein correspondence $N\rightarrow M$}
\label{chap:intro2}

The purpose of this chapter is to introduce the preliminaries that are required to understand the remainder of the thesis. We will first review projective geometry, including the Cartan and tractor bundles associated with a projective structure, before moving on to the projective to Einstein correspondence of \cite{DM}. We begin with some notation and conventions.

\subsubsection{Notation and conventions}
\begin{itemize}
\item We use $\R^n$ to mean the real vector space of dimension $n$, and $\R_n$ to mean its dual. We think of vectors in $\R^n$ as column vectors, and vectors in $\R_n$ as row vectors.
\item We will projectivise vector spaces by \textit{line projectivisation}, that is, we will take the space of \textit{unoriented} lines through the origin, unless stated otherwise.
\item When we refer to the projective to Einstein correspondence, the projective manifold will be called $N$ and will have dimension $n$, whilst the Einstein manifold will be called $M$ and will have dimension $2n$.
\item We use the letter $\pi$ for maps to $N$, and the letter $\kappa$ for maps to $M$. We will attach subscripts to $\pi$ and $\kappa$ to give information about the preimage or the significance of the map.
\item We use lower case Latin indices $i,j,k,\dots=1,\dots,n$ for tensorial objects on $N$ and $a,b,c,\dots=1,\dots,2n$ for tensorial objects on $M$.
\item We use $\odot$ and $\wedge$ to denote the symmetrised and antisymmetrised tensor product respectively. That is,
\[
A\odot B = \frac{1}{2}(A\otimes B + B\otimes A),\qquad A\wedge B = \frac{1}{2}(A\otimes B - B\otimes A).
\]
We also occasionally write $A^2$ for $A\odot A$, and in Chapter \ref{chap:EW_and_toda} we will sometimes omit the $\odot$ altogether.
\item Where indices have been symmetrised or antisymmetrised over, we will enclose them in round or square brackets respectively.
\item Our conventions for differential forms are
\[
(d\omega)_{ab\dots c}=\partial_{[a}\omega_{b\dots c]},\qquad (\eta\wedge\omega)_{a\dots d}=\eta_{[a\dots b}\omega_{c\dots d]},\]
\[
\omega=\omega_{a\dots b}\,dx^{a}\wedge\dots\wedge dx^{b}, \qquad F_{ab}\,{d}x^{a}\wedge{d}x^{b}=F_{[ab]}\,{d}x^{a}\otimes{d}x^{b}.
\]
\item The Riemann curvature tensor $R_{abc}^{\quad d}$ of a connection $\nabla_a$ is defined by
\[
(\nabla_a\nabla_b - \nabla_b\nabla_a)X^d = R_{abc}^{\quad d}X^c,
\]
where $X$ is any vector field.
\end{itemize}


\section{Projective Geometry}\label{sec:projgeom}

Our discussion follows Eastwood \cite{Eastwood}.

\begin{defi}\label{def:projstruct} A projective structure $(N,[\nabla])$
on a manifold $N$ is an equivalence class $[\nabla]$ of torsion--free affine connections on $N$ which have the same geodesics as unparametrised curves.
\end{defi}

The following proposition converts definition \ref{def:projstruct} to a more operational form.

\begin{prop} Two torsion--free connections $\nabla$ and $\ov{\nabla}$ belong to the same projective class if and only if their components $\Gamma^i_{jk}$ and $\ov{\Gamma}^i_{jk}$ are related by
\begin{equation}
\ov{\Gamma}^i_{jk} - \Gamma^i_{jk} = \delta_{j}^{i}\Upsilon_{k}+\delta_{k}^{i}\Upsilon_{j}\label{eq:proj_change}
\end{equation}
for some one--form $\Upsilon.$
\end{prop}

{\bf Proof.} We denote by $\mathcal{V}$ the vertical sub--bundle of $T(TN)$, where $\pi_T:TN\rightarrow N$ is the tangent bundle to $N$. A connection defines a splitting of the exact sequence
\be \label{eq:TTMsequence}
0 \longrightarrow \mathcal{V} \longrightarrow T(TN) \longrightarrow \pi_T^*TN \longrightarrow 0
\ee
so that each $\xi\in T_xN$ has a unique pull--back in the horizontal sub--bundle complementary to $\mathcal{V}_x$. The integral curves of these pull--backs, when projected down to $N$, then define the geodesics of the connection.

Any two connections are related by some $\delta\Gamma^k_{ij}$, which satisfies $\delta\Gamma^k_{ij}=\delta\Gamma^k_{(ij)}$ as long as both connections are torsion--free. A change of connection is equivalent to a change in the splitting of (\ref{eq:TTMsequence}). At $\xi\in T_xN$, the change is given by the homomorphism from $T_xN$ to $T_xN=\mathcal{V}_x$ defined by the contraction $\xi^i\Gamma^k_{ij}$. Thus the two connections define the same geodesics if and only if $\xi^i\xi^j\Gamma^k_{ij}$ is a multiple of $\xi^k$ for all $\xi^i$. This is true if and only if there is a one--form $\Upsilon_i$ such that (\ref{eq:proj_change}) is satisfied.\footnote{To see this, take some one--form $\omega_i$ and note that $2\xi^i\xi^j\delta^k_{(i}\Upsilon_{j)}\omega_k$ vanishes if and only if $\xi^k\omega_k$ does.}
\koniec

One can show that the curvature of a connection $\nabla$ in the projective class can be uniquely decomposed as
\be \label{eq:projcurvdecomp}
R_{ijk}^{\ \ \ l} = W_{ijk}^{\ \ \ l} + 2\delta^l_{[i}\Rho_{j]k} -2\Rho_{[ij]}\delta^l_k,
\ee
where the Weyl projective curvature tensor, $W_{ijk}^{\ \ \ l}$, is trace free, and the Schouten tensor, $\Rho_{ij}$, is given in terms of the Ricci tensor by
\[
\Rho_{ij}=\frac{1}{n-1}R_{(ij)}+\frac{1}{n+1}R_{[ij]}.
\]
The objects $W_{ijk}^{\ \ \ l}$ and $\Rho_{ij}$ transform as
\be \label{eq:schout_change}
\ov{W}_{ijk}^{\ \ \ l} = W_{ijk}^{\ \ \ l}, \qquad \ov{\Rho}_{ij} = \Rho_{ij} - \nabla_i\Upsilon_j + \Upsilon_i\Upsilon_j
\ee
under a change of representative connection (\ref{eq:proj_change}). Note that for $n=2$ the Weyl tensor always vanishes.

A projective structure in dimension $n$ is said to be flat if it is diffeomorphic to the real projective space $\RP^n$ with its standard flat projective structure.
\begin{defi} \label{def:RPn}
The real projective space $\RP^n$ of dimension $n$ is the space of unoriented lines through the origin in $\R^{n+1}$, thought of as $\RP^n=(\R^{n+1}\backslash\{0\})/\R^*$, where the quotient identifies points $P\in\R^{n+1}$ under the equivalence relation
\[
(P^0,\dots,P^n)\sim (cP^0,\dots,cP^n)\ \forall\ c\in\R^*.
\]
The geodesics on $\RP^n$ are given by planes through the origin in $\R^{n+1}$ under the projection $\pi_\PP:\R^{n+1}\rightarrow\RP^n$.
\end{defi}

\begin{rmk}
Let $P$ denote a non--zero point in $\R^{n+1}$ with coordinates $(P^0,\dots,P^n)^T$, and let $[P]$ denote the corresponding point in $\RP^n$, labelled by homogeneous coordinates. In a patch $\mathcal{U}_0$ where $P^0\neq 0$, we can write $[P]=[1,P^1/P^0,\dots,P^n/P^0]^T$ and define inhomogeneous coordinates on $\RP^n$ by
\[
(x^1,\dots,x^n) = (P^1/P^0,\dots,P^n/P^0).
\]
If we combine this with coordinate patches $\mathcal{U}_i$ where $P^i\neq 0,\ i=1,\dots,n$, we can build an atlas for $\RP^n$.
\end{rmk}

\begin{rmk}
The flat projective structure on $\RP^n$ has a special duality property which we now discuss. Consider the set of planes through the origin in $\R^{n+1}$. These can be specified by their normal vector, which is defined only up to multiplication by $\R^*$. Let us denote such a plane by a non--zero row vector $L\in\R_{n+1}$. A point $P\in\R^{n+1}$ lies in the plane defined by $L$ if and only if $L\cdot P=0$.

When we projectivise the $\R^{n+1}$, any $P\neq 0$ descends to a point $[P]\in\RP^n$, and any plane descends to a line $[L]\subset\RP^n$. The incidence relation $L\cdot P=0$ is now equivalent to the point $[P]$ lying in the line $[L]$. The homogeneous coordinates $[L]$ parametrise a second projective surface which we think of as the dual to the $\RP^n$ parametrised by $[P]$, and denote $\RP_n$.
\end{rmk}

\begin{rmk}
Real projective space can be viewed as homogeneous space as follows. 
The group $SL(n+1,\mathbb{R})$ acts from the left via the fundamental representation on coordinates $(P^0,\dots,P^n)^T$ in $\R^{n+1}$, and this descends to a transitive action on $\RP^n$. By the orbit stabiliser theorem, $\RP^n=SL(n+1,\R)/S$, where $S$ is a subgroup stabilising a point. If we choose the point $[1,0,\dots,0]^T$, the elements of $S$ are matrices of the general form
\[
\begin{pmatrix}\mathrm{det}a^{-1} & b\\
0 & a
\end{pmatrix}
\]
for some $a\in GL(n,\mathbb{R})$ and $b\in\mathbb{R}_{n}$.
\end{rmk} 

\begin{rmk}
It can be shown that a projective structure in dimension $n>2$ is flat if and only if its Weyl projective curvature tensor vanishes. The necessary and sufficient condition in dimension $n=2$ is the vanishing of the Cotton tensor $\nabla_{[i}\Rho_{j]k}$ for any choice of representative connection.
\end{rmk}

\subsection{The Cartan bundle}

One way of understanding the construction in \cite{DM}
is via the Cartan bundle \cite{Cartan} of the projective structure $(N,[\nabla])$ (see also \cite{KobNag,Sharpe}). Cartan geometries generalise Klein's Erlangen programme \cite{Klein}, a study of homogeneous spaces $G/S$, to the curved case, in which the total space $G$ is replaced by a principal right $S$-bundle over a manifold $N$ such that the tangent space to $N$ at every point is isomorphic to the Lie algebra quotient $\mathfrak{g}/\mathfrak{s}$. Since projective structures are modelled on $\mathbb{RP}^{n}$, which can be viewed as a homogeneous space, they constitute a type of Cartan geometry.

In the Riemannian case, the model space is $\mathbb{R}^{n}\cong\mathrm{Euc}(n)/SO(n)$. The corresponding Cartan geometry is a general, curved Riemannian manifold. One has an obvious subclass of frames which are ``adapted'' to the metric, i.e. those which are orthonormal. We can thus think of a curved Riemannian manifold as a principal $SO(n)$ bundle whose tangent spaces are modelled on $\mathbb{R}^{n}\cong\mathfrak{Euc}(n)/\mathfrak{so}(n)$. We say that Riemannian manifolds are Cartan geometries of type $(\mathrm{Euc}(n),SO(n))$.

The theory of Cartan geometries was developed as part of Cartan's
\textit{method of moving frames}. The idea is to pick out some adapted frames for manifolds equipped with some non-metric structure. The bundle of such frames over a manifold is then a principal bundle $\pi_\mathcal{G}:\mathcal{G}\rightarrow N$ with structure group $S$. In the projective case, the notion of a \textit{second order frame} must be introduced to obtain an object which is correctly adapted to the projective structure.

The bundle $P$ is equipped with a $\mathfrak{g}$-valued one-form
$\theta$ called the Cartan connection. It defines an isomorphism $\theta:T_{u}\mathcal{G}\rightarrow\mathfrak{g}$ at every point $u\in \mathcal{G}$ such that the vertical subspace $\mathcal{V}_{u}\mathcal{G}\subset T_{u}\mathcal{G}$ is mapped to $\mathfrak{s}$ and the horizontal subpace $\mathcal{H}_{u}\mathcal{G}\subset T_{u}\mathcal{G}$ is defined as the inverse image of $\mathfrak{g}/\mathfrak{s}$. Note that it is not a connection in the usual sense of a principal bundle connection, since it takes value in a Lie algebra larger than that of the structure group. Further details can be found in \cite{Sharpe}.

%The importance of the Cartan connection is that it satisfies a number of properties, in particular equivariance, i.e. $R_{h}^{*}\theta=\mathrm{Ad}(h^{-1})\theta$ for all $h\in H$. 

In the projective case, if we choose the point which is stabilised by $S$ to be $[1,0,\dots,0]$, the Cartan connection can be written as a matrix
\be \label{eq:cartan_connection}
\theta=\begin{pmatrix}-\mathrm{tr}\phi & \eta\\
\omega & \phi
\end{pmatrix},
\ee
where $\omega$, $\eta$ and $\phi$ are one-forms valued in $\mathbb{R}^{n}$, $\mathbb{R}_{n}$ and $\mathfrak{gl}(n,\mathbb{R})$ respectively.
We will refer to the components of $\omega$ and $\eta$ with respect
to the natural basis of $\mathfrak{sl}(n+1,\mathbb{R})$ as $\{\omega^{(i)}\}$ and $\{\eta_{(i)}\}$, so that $\omega^{(i)}$ and $\eta_{(i)}$ are both one-forms taking values in $\R$.

\begin{defi}
The Cartan geometry of a projective structure $(N,[\nabla])$ consists of a principal right $S$--bundle $\pi_\mathcal{G}:\mathcal{G}\rightarrow N$, where the right--action of some $s\in S$ on $\mathcal{G}$ is denoted by $R_s$, and a one--form $\theta$ on $\mathcal{G}$ called the Cartan connection, which takes values in $\mathfrak{sl}(n+1,\mathbb{R})$. The Cartan connection can be written in the form (\ref{eq:cartan_connection}) and has the following properties:
\begin{enumerate}
\item $\theta_u:T_u\mathcal{G}\rightarrow\mathfrak{sl}(n+1,\R)$ is an isomorphism for all $u\in \mathcal{G}$;
\item $\theta(\xi_\mathfrak{v})=\mathfrak{v}$ for all fundamental vector fields $\xi_\mathfrak{v}$ on $\mathcal{G}$;
\item $R^*_s\theta = \mathrm{Ad}(s^{-1})\theta=s^{-1}\theta s$ for all $s\in S$.
\item If $\xi$ is a vector field on $\mathcal{G}$ with the property that $\eta(\xi)=\phi(\xi)=0$ and $\omega(\xi)\in\R^n\backslash\{0\}$, then the integral curve of $\xi$ projects down to a geodesic on $N$ and conversely every geodesic of $[\nabla]$ arises in this way.
\item The $\mathfrak{sl}(n+1,\mathbb{R})$-valued
curvature two-form $\Theta$ satisfies
\be \label{eq:curvature_2-form}
\Theta=d\theta+\theta\wedge\theta=\begin{pmatrix}0 & L(\omega\wedge\omega)\\
0 & W(\omega\wedge\omega)
\end{pmatrix},
\ee
where $L$ and $W$ are smooth curvature functions valued in $\mathrm{Hom}(\R^n\wedge\R^n,\R_n)$ and $\mathrm{Hom}(\R^n\wedge\R^n,\R_n\otimes\R^n)$ respectively. The function $W$ represents the Weyl projective curvature tensor appearing in (\ref{eq:projcurvdecomp}).
\end{enumerate}
\end{defi}

\begin{rmk}The Cartan geometry of a projective structure is unique in the sense that for any two Cartan geometries $(\widehat{\pi}_\mathcal{G}:\widehat{\mathcal{G}}\rightarrow N,\widehat{\theta})$ and $(\pi_\mathcal{G}:\mathcal{G}\rightarrow N,\theta)$ of type $(SL(n+1,\R),S)$ satisfying the above properties there is a $S$--bundle isomorphism $\nu:\mathcal{G}\rightarrow\widehat{\mathcal{G}}$ such that $\nu^*\widehat{\theta}=\theta$. %This means that although we cannot choose a unique connection on the tangent bundle to $N$, we can choose a unique connection on the Cartan bundle.
\end{rmk}

\begin{rmk} \label{rmk:theta_symmetry}
For every open set $\mathcal{U}\subset N$, projective vector fields on $\mathcal{U}$ are in one-to-one correspondence with vector fields on $\pi_\mathcal{G}^{-1}(\mathcal{U})$
which preserve $\theta$ and are equivariant under the principal $S$--action.
\end{rmk}






\subsection{Tractor bundles}
The Cartan connection also gives us a unique connection on any bundle associated to $\mathcal{G}$ via some $S$--module. In particular, let $\mathcal{B}$ be a vector space and $\rho_\mathcal{B}:S\rightarrow GL(\mathcal{B})$ a representation of $S$ acting on $\mathcal{B}$. We can construct an \textit{associated bundle}
\[\pi_\mathcal{B}:\mathcal{G}\times_{\rho_\mathcal{B}} \mathcal{B}\rightarrow N \]
where points in $\mathcal{G}\times_{\rho_\mathcal{B}} \mathcal{B}$ are equivalence classes of pairs $[u,v]$, where $u\in \mathcal{G}$ and $v\in \mathcal{B}$, up to the equivalence relation
\[
(u_1,v_1)\sim (u_2,v_2) \quad \Leftrightarrow \quad \exists\ s\ \mbox{such that}\  u_2=u_1 s,\  v_2 = \rho_\mathcal{B}(s^{-1}) v_1.
\]

We thus obtain a vector bundle over $N$ whose fibres are diffeomorphic to $\mathcal{B}$. A section $\tilde{\sigma}:N\rightarrow \mathcal{G}\times_{\rho_\mathcal{B}} \mathcal{B}$ is represented by a map ${\sigma}:\mathcal{G}\rightarrow \mathcal{B}$ which is equivariant in the sense that ${\sigma}(us)=\rho_\mathcal{B}(s^{-1}){\sigma}(u)$ for all $s\in S$. Importantly, any such bundle inherits a connection from the Cartan connection $\theta$ on $\mathcal{G}$. The concept of an associated bundle applies to any principal bundle, but we call vector bundles which are associated to a Cartan bundle \textit{tractor} bundles, and the connections that they inherit from the Cartan connection are called tractor connections.

A particularly important example of a vector bundle associated to $\mathcal{G}$ is the \textit{cotractor bundle}, which defined by the canonical right action of $S$ on $\R_{n+1}$ given by $(s,L)\mapsto Ls^{-1}$. We call this bundle $\pi_\mathcal{T}:\cT^*\rightarrow N$. In order to describe its connection, we consider a section represented by $\sigma:\mathcal{G}\rightarrow\R_{n+1}$ and define the one--form
\be \label{eq:df-ftheta}
d\sigma - \sigma\theta.
\ee
This turns out to be a \textit{semi--basic}\footnote{Recall that a semi--basic form on a fibre bundle $\mathcal{G}\rightarrow N$ is a form which is a linear combination, with coefficients parametrised by the fibres, of basic forms on $\mathcal{G}$ (i.e. forms which are the pull-backs of forms on $N$).} one--form satisfying
\[
R_s^*(d\sigma-\sigma\theta) = (d\sigma - \sigma\theta)s,
\]
making $\sigma\mapsto d\sigma-\sigma\theta$ an equivariant connection on $\cT^*$.

Although this construction of $\mathcal{T}^*$ relies on the Cartan bundle, it is possible to construct it independently. In order to do so we need the notion of a \textit{projective density}.%To do so, we consider the transformation of the derivative of a volume form under a projective change (\ref{eq:proj_change}). This approach can be motivated by the fact that the special linear group $SL(n,\R)$, which we have already seen to be important in projective geometry, is the group of volume preserving transformations of $\R^n$.
\subsubsection{Projective densities}
From the projective change of connection (\ref{eq:proj_change}) we can derive the the corresponding change in $\nabla\chi$ for some $m$--form $\chi$ on $N$:
\be \label{eq:p-form_change}
\ov{\nabla}_i\chi_{jk\dots l} = \nabla_i\chi_{jk\dots l} - (m+1)\Upsilon_i\chi_{jk\dots l} - (m+1)\Upsilon_{[i}\chi_{jk\dots l]}.
\ee
In particular, for a volume form ($m=n$) we find
\[
\ov{\nabla}_i\chi_{jk\dots l} = \nabla_i\chi_{jk\dots l} - (n+1)\Upsilon_i\chi_{jk\dots l},
\]
where the final term in (\ref{eq:p-form_change}) has vanished because it contains a symmetrisation over $n+1$ indices. We can write this in a more compact way as
\[
\ov{\nabla}_i\chi = \nabla_i\chi - (n+1)\Upsilon_i\chi.
\]

Note that for sections $\tau$ of the bundle $\mathcal{E}(w):=(\Lambda^n)^{-w/(n+1)}$ we have
\be \label{eq:ddensity_change}
\ov{\nabla}_i\tau = \nabla_i\tau + w\Upsilon_i\tau.
\ee
We called such sections \textit{projective densities of weight $w$}, and for any vector bundle $\mathcal{B}\rightarrow N$ we write $\mathcal{B}(w)$ for the tensor product of $\mathcal{B}$ with $\mathcal{E}(w)$. For example, $T^*N(w)$ is the bundle of one--forms with projective weight $w$, and for sections $\mu_i$ of $T^*N(w)$ we have
\be \label{eq:dweighted_form_change}
\ov{\nabla}_i\mu_j = \nabla_i\mu_j + (w-1)\Upsilon_i\mu_j - \Upsilon_j\mu_i.
\ee

\subsubsection{The cotractor bundle}
We can now define the cotractor bundle $\pi_\cT:\mathcal{T}^*\rightarrow N$. For a choice of connection in the projective class we identify
\be \label{eq:T*splitting}
\mathcal{T}^* = \mathcal{E}(1)\oplus T^*N(1),
\ee
so that a section can be represented by a pair
\be \label{eq:T*coords}
\begin{pmatrix}
{\tau} \\ {\mu}_i
\end{pmatrix}.
\ee
Under a change of projective connection (\ref{eq:proj_change}), this splitting changes according to
\be \label{eq:chi_mu_change}
\ov{\begin{pmatrix}
{\tau} \\ {\mu}_i
\end{pmatrix}} =
\begin{pmatrix}
\tau \\ \mu_i + \Upsilon_i\tau
\end{pmatrix}.
\ee
Note the exact sequence
\be \label{eq:T*sequence}
0\longrightarrow T^*N(1)\longrightarrow \mathcal{T}^* \overset{V}\longrightarrow \mathcal{E}(1)\longrightarrow 0,
\ee
where we call the map $V$ the projective \textit{canonical tractor}\footnote{In fact $V$ is a section of a bundle $\mathcal{T}(1)$, where $\mathcal{T}$ can be identified with a direct sum $\mathcal{E}(1)\oplus TN(1)$ given a choice of connection in the projective class. The natural pairing between $\mathcal{T}$ and $\mathcal{T}^*$ defines the map $V:\mathcal{T}^*\rightarrow\mathcal{E}(1)$.}. A choice of connection in the projective class defines a splitting (\ref{eq:T*splitting}) of (\ref{eq:T*sequence}).

The bundle $\mathcal{T}^*$ admits a projectively invariant \textit{tractor connection} given by
\be \label{eq:tractor_connection}
\nabla^\mathcal{T}_i \begin{pmatrix}
\tau \\ \mu_j
\end{pmatrix}
= \begin{pmatrix}
\nabla_i \tau - \mu_i \\
\nabla_i\mu_j + \Rho_{ij}\tau
\end{pmatrix},
\ee
where $\nabla$ is the choice of projective connection and $\Rho_{ij}$ is its Schouten tensor. This turns out to agree with (\ref{eq:df-ftheta}). Under a change of projective connection (\ref{eq:proj_change}), we find
\begin{align*}
\ov{\nabla}^\mathcal{T}_i\ov{\begin{pmatrix}
\tau \\ \mu_j
\end{pmatrix}}
&= \ov{\nabla}^\mathcal{T}_i\begin{pmatrix}
\tau \\ \mu_j + \Upsilon_j\tau \end{pmatrix} \\
&= \begin{pmatrix}
\ov{\nabla}_i\tau - (\mu_i + \Upsilon_i\tau) \\
\ov{\nabla}_i(\mu_j+\Upsilon_j\tau) + \ov{\Rho}_{ij}\tau
\end{pmatrix} \\
&= \begin{pmatrix}
\nabla_i\tau + \Upsilon_i\tau - (\mu_i + \Upsilon_i\tau) \\
\nabla_i(\mu_j + \Upsilon_j\tau) - \Upsilon_j(\mu_i + \Upsilon_i\tau)
+ (\Rho_{ij} - \nabla_i\Upsilon_j + \Upsilon_i\Upsilon_j) \tau
\end{pmatrix},
\end{align*}
where we have used (\ref{eq:chi_mu_change}) in the first line, (\ref{eq:tractor_connection}) in the second and (\ref{eq:ddensity_change}), (\ref{eq:dweighted_form_change}) and (\ref{eq:schout_change}) in the third. After some cancellation, we identify
\begin{align*}
\ov{\nabla}^\mathcal{T}_i\ov{\begin{pmatrix}
\tau \\ \mu_j
\end{pmatrix}}
&= \begin{pmatrix}
\nabla_i\tau - \mu_i \\
\nabla_i\mu_j + \Upsilon_j\nabla_i\tau - \Upsilon_j\mu_i + \Rho_{ij}\tau
\end{pmatrix} \\
&= \ov{\begin{pmatrix}
\nabla_i\tau - \mu_i \\
\nabla_i\mu_j + \Rho_{ij}\tau
\end{pmatrix}}
= \ov{\nabla^\mathcal{T}_i \begin{pmatrix}
\tau \\ \mu_j
\end{pmatrix}}
\end{align*}
using (\ref{eq:tractor_connection}) and the change of splitting (\ref{eq:chi_mu_change}) adapted to the tensor product of $T^*N$ and $ \cT^*$ of which the derivative is a section.

Any tensor product of $\cT,\cT^*$ and $\cE(1)$ is equipped with a tractor connection which is inherited from the connection on the standard cotractor bundle. Equivalently, any such tensor product can be thought of as an associated vector bundle to the Cartan bundle $\mathcal{G}$, with its connection inherited from the Cartan connection via the corresponding representation of $S$. It is this connection, with its special equivariance property, that allows us to construct an Einstein metric as an invariant of the projective structure. This construction is the subject of the following section.

\section{The projective to Einstein correspondence}
Consider a quotient of the total space $\mathcal{G}$ of the
Cartan bundle by $GL(n,\mathbb{R})$, which is embedded in $S$
in the obvious way:
\be \label{eq:GL(n)_embedding}
GL(n,\mathbb{R})\ni a\longmapsto\begin{pmatrix}\mathrm{det}a^{-1} & 0\\
0 & a
\end{pmatrix}\in S.
\ee
It is easily verified that 
\[
\begin{pmatrix}\mathrm{det}a^{-1} & 0\\
0 & a
\end{pmatrix}\begin{pmatrix}0 & \eta\\
\omega & 0
\end{pmatrix}\begin{pmatrix}\mathrm{det}a^{-1} & 0\\
0 & a
\end{pmatrix}^{-1}=\begin{pmatrix}0 & \eta a^{-1}\mathrm{det}a^{-1}\\
(\mathrm{det}a)a\omega & 0
\end{pmatrix},
\]
for any $a\in GL(n,\mathbb{R})$, meaning that due to the equivariance
property of the Cartan connection, the natural contraction $\eta\omega:=\sum_{i}\eta_{(i)}\otimes\omega^{(i)}$
defined by $\theta$ is preserved by the adjoint action of this $GL(n,\mathbb{R})$
subgroup. It thus descends to a naturally defined object on the quotient
$M=\mathcal{G}/GL(n,\mathbb{R})$.

\begin{theo}{\cite{DM}}\label{thm:DM}
There exist a metric and two--form
$(g,\Omega)$ on $\mbox{\ensuremath{M=\mathcal{G}/GL(n,\mathbb{R})}}$ such
that the quotient map $\kappa_q:P\rightarrow M$ gives
\begin{eqnarray} 
\kappa_q^{*}g & = & \mathrm{Sym}(\eta\omega) \label{eq:g_cartan} \\
\kappa_q^{*}\Omega & = & \mathrm{Ant}(\eta\omega), \label{eq:Omega_cartan}
\end{eqnarray}
where $\mathrm{Sym}$ and $\mathrm{Ant}$ denote the symmetric and
anti-symmetric parts of the $(0,2)$ tensor $\eta\omega$. Moreover,
$\Omega$ is closed as a consequence of the Bianchi identity satisfied by the curvature two--form (\ref{eq:curvature_2-form}), $g$
is Einstein with non-zero scalar curvature, and the two are related
by an endomorphism $J$ satisfying $J^{2}=Id$. Hence $(g,\Omega)$
is an almost para--K\"ahler structure on $M$.
\end{theo}


\begin{rmk}
The full proof of Theorem \ref{thm:DM} only appears explicitly in \cite{DM} in the case $n=2$, although it can be generalised to $n>2$. This generalisation is discussed in their appendix. They show that the Ricci scalar of $g$ is $24$ in the case $n=2$. In Chapter \ref{chap:KK_lift} we will need the Ricci scalar for general $n$. We will calculate this under the assumption (stated without proof in \cite{DM}) that $g$ is Einstein.
\end{rmk}

\begin{rmk}
The quotient $M$ turns out to be an affine bundle over $N$ with
structure group $S$, i.e. $S$ acts affinely on the fibres of $\pi_M:M\rightarrow N$,
and sections of this bundle are in one-to-one correspondence with
representative connections $\nabla\in[\nabla]$. This means that given
some choice of connection $\nabla\in[\nabla]$ we have a diffeomorphism
$\kappa_A:T^{*}N\rightarrow M$ with which we can pull back the pair
$(g,\Omega)$. In canonical local coordinates $(x^{i},\zeta_{i})$ on
the cotangent bundle, we find
\begin{eqnarray}
\kappa_A^{*}g & = &  d\zeta_{i}\odot dx^{i}-(\Gamma_{ij}^{k}\zeta_{k}-\zeta_{i}\zeta_{j}-\Rho_{ij}) dx^{i}\odot dx^{j},\label{eq:coord_form}\\
\kappa_A^{*}\Omega & = &  d\zeta_{i}\wedge dx^{i}+\Rho_{ij} dx^{i}\wedge dx^{j},\qquad i,j=1,\dots ,n.\nonumber 
\end{eqnarray}
Here $\Gamma_{jk}^{i}$ are the connection components of the representative
connection $\nabla$ that we chose, and its Schouten tensor is denoted $\Rho_{ij}$. This can be shown to be projectively invariant
in the sense that a different choice of $\nabla\in[\nabla]$ corresponds
to shifting the fibre coordinates $\zeta_{i}$, i.e. metrics on $T^{*}N$
resulting from pulling back $g$ using different representative connections
are isometric. Explicitly, a projective transformation (\ref{eq:proj_change})
corresponds to a change
\begin{equation}
\zeta_{i}\longrightarrow \zeta_{i}+\Upsilon_{i}.\label{eq:zeta_change}
\end{equation}
\end{rmk}


\begin{rmk}
In fact, the metric and symplectic form (\ref{eq:coord_form}) turn
out to belong to a one-parameter family $\{(g_\Lambda,\Omega_\Lambda)\,;\,\Lambda\neq 0\}$, which
can be written in local coordinates as 
\begin{eqnarray}
g_{\Lambda} & = &  d\zeta_{i}\odot dx^{i}-(\Gamma_{ij}^{k}\zeta_{k}-\Lambda \zeta_{i}\zeta_{j}-\Lambda^{-1}\Rho_{ij}) dx^{i}\odot dx^{j}\label{eq:general_g}\\
\Omega_{\Lambda} & = &  d\zeta_{i}\wedge dx^{i}+\frac{1}{\Lambda}\Rho_{ij} dx^{i}\wedge dx^{j},\qquad i,j=1,\dots ,n.\label{eq:general_omega}
\end{eqnarray}
Metrics of the form (\ref{eq:general_g}) are a subclass of so-called
Osserman metrics. More details can be found in \cite{osserman}. They are all Einstein with non--zero scalar curvature $24\Lambda$, but for $\Lambda\neq1$ the relation to projective geometry is lost. For the remainder of the thesis we will write $g$ for $g_{\Lambda=1}$ unless stated otherwise. Note that $\{g_\Lambda\}$ will be the subject of Chapter \ref{chap:KK_lift}, whilst in Chapters  \ref{chap:EW_and_toda} and \ref{chap:c-proj} we will restrict our attention to $g$ because the projective geometry is a key aspect of the content of these chapters.
\end{rmk}



\begin{rmk}
One could also consider taking a quotient of $\mathcal{G}$ by a different subgroup of $S$. The $\R^*$ bundle over $M$ which we will discuss in Chapter \ref{chap:KK_lift} will turn out to be a quotient of $\mathcal{G}$ by $SL(n,\R)$.
\end{rmk}

\begin{rmk}
As mentioned above, in the special case where the $(N,[\nabla])$ is a projective surface, $M$ has dimension four, and so anti--self--duality is defined. It turns out that both the symplectic form $\Omega$ and the conformal curvature of $g$ are ASD. Both of these facts will play an important role in Chapter \ref{chap:EW_and_toda}.
\end{rmk}

\begin{rmk}
Note that an endomorphism $J$ which squares to the identity defines two $n$--dimensional sub--bundles of the tangent bundle $TM$ defined at each $m\in M$ as the vector subspaces of $T_mM$ which have eigenvalues $\pm 1$ with respect to $J$. These sub--bundles form a pair of smooth distributions $D_\pm$ in $TM$. Further discussion of the endomorphism $J$ will appear in Chapter \ref{chap:c-proj}.
\end{rmk}

\subsection{Symmetries of $M$}

Recall that a projective vector field on any manifold with a connection
generates a one--parameter family of transformations which preserve the
geodesics of that connection up to parametrisation. Projective vectors
fields thus naturally arise as the symmetries of a projective structure.
Explicitly, a vector field $\widehat{K}$ is projective if it satisfies
\begin{equation}
\mathcal{L}_{\widehat{K}}\Gamma_{ij}^{k}=\delta_{i}^{k}\Upsilon_{j}+\delta_{j}^{k}\Upsilon_{i}\label{eq:proj_transf}
\end{equation}
for some 1-form $\Upsilon$, where $\Gamma_{ij}^{k}$ are the connection
components, and their Lie derivative is defined (see \cite{yano})
by
\begin{equation}
\mathcal{L}_{\widehat{K}}\Gamma_{ij}^{k}\equiv\frac{\partial^{2}\widehat{K}^{k}}{\partial x^{i}\partial x^{j}}+\widehat{K}^{m}\frac{\partial\Gamma_{ij}^{k}}{\partial x^{m}}-\Gamma_{ij}^{m}\frac{\partial \widehat{K}^{k}}{\partial x^{m}}+\Gamma_{im}^{k}\frac{\partial \widehat{K}^{m}}{\partial x^{j}}+\Gamma_{mj}^{k}\frac{\partial \widehat{K}^{m}}{\partial x^{i}}.\label{eq:liederivGamma}
\end{equation}


One consequence of the symmetry property of the Cartan connection discussed in remark \ref{rmk:theta_symmetry} is that for every open set $\mathcal{U}\subset N$ we have an isomorphism between the Lie algebra of projective vector fields on $\mathcal{U}$ and the Lie algebra of vector fields on $\pi_\mathcal{G}^{-1}(\mathcal{U})$ preserving the natural contraction $\eta\omega$. Such vector fields must descend to vector fields on $\pi_M^{-1}(\mathcal{U})$ preserving $(g,\Omega)$. In fact, it can be shown that every Killing vector field of $(M,g_\Lambda)$ is also symplectic with respect to $\Omega_\Lambda$ and is therefore the lift of a projective vector field on $(N,[\nabla])$.

Explicitly, for every projective vector field $\widehat{K}$ of $(N,[\nabla])$
there is a corresponding symmetry $K$ of $(M,g_{\Lambda},\Omega_\Lambda)$
given in local coordinates by 
\begin{equation}
{K}=\widehat{K}-\zeta_{i}\frac{\partial \widehat{K}^{j}}{\partial x^{i}}\frac{\partial}{\partial \zeta_{j}}+\frac{1}{\Lambda}\Upsilon_{i}\frac{\partial}{\partial \zeta_{i}},\label{eq:kvf_from_pvf}
\end{equation}
where $\Upsilon_{i}$ is defined by (\ref{eq:proj_transf}).

\subsection{Tractor perspective} \label{sec:trac_construction}

From the tractor perspective, the space $M$ will turn out to be the projectivised cotractor bundle of $N$ with an $\RP^{n-1}$ sub--bundle removed from each fibre. We can understand what this $\RP^{n-1}$ sub--bundle is as follows.

On the total space of $\cT^*$ we pull back $\pi_\mathcal{T}:\cT^*\to N$ along $\pi_\mathcal{T}$ to get $\pi_\mathcal{T}^*(\cT^*)\to \cT^*$ as a vector bundle over the total space $\cT^*$. By construction this bundle has a tautological section $W\in \Gamma (\pi_\cT^*(\cT^*))$.  We also have $\pi_\cT^*(\cT(w))$ for any weight $w$, and we shall write simply $V\in \Gamma(\pi_\cT^*(\cT(1)))$ for the pull back to $\cT^*$ of the canonical tractor $V$ on $N$.

Now define
\be
\label{projection_map}
\kappa_\PP: \cT^*\longrightarrow \mathcal{M}:=\mathbb{P}(\cT^*)
\ee
by the fibre--wise projectivisation, and use $\pi_\cM$ for the map
$$
\pi_\cM:\mathcal{M}\to N.
$$
We denote by $\cE_{\cT^*}(w')$, for $w'\in \mathbb{R}$, the line
bundle on $\mathbb{P}(\cT^*)$ whose sections correspond to functions
$f: \pi_\cT^*\cT^* \to\mathbb{R} $ that are homogeneous of degree $w^\prime$ in
the fibres of $\pi_\cT^*\cT^*\to \mathbb{P}(\cT^*)$. For any weight $w$ we also have $\cE(w)$ on $N$ and its pull back to the bundle $\pi_\cM^*\cE(w)\to \mathbb{P}(\cT^*)$.
We define the product of these two density bundles on $\cM$ as
$$
\ce(w,w'):= \pi_\cT^*\cE(w) \otimes \cE_{\cT^*}(w').
$$

On $\cT^*$ there is  a canonical density $\tau\in \Gamma(\pi_\cT^*\cE(1))$ given by
$$
\tau:= V\hook W .
$$
%% and this is homogeneous of degree 1 up the fibres of $\pi$ in the total space $\cT^*$.
Note that $\tau$ is homogeneous of degree 1 up the fibres of the
map $\kappa_\PP:\cT^*\to \mathcal{M}$. Thus $\tau$ determines, and is equivalent
to, a section (that we also denote) $\tau$ of the density bundle $\cE(1,1)$. So $\mathcal{M}$ is stratified according to
whether or not $\tau$ is vanishing, and we write $\mathcal{Z}(\tau)$
to denote, in particular, the zero locus of $\tau$. We will show in Chapter \ref{chap:c-proj} that $M$ can be identified with $\mathcal{M}\setminus \mathcal{Z}(\tau)$.

%\rs{Note the
%  second density bundle is not necessarily orientable.}


\subsection{The model case} \label{sec:intro_model}
When $(N,[\nabla])$ is the flat projective structure $\RP^n$,the metric and symplectic form (\ref{eq:coord_form}) reduce to
\be \label{eq:intro_model_g}
g = d\zeta_i\odot dx^i + \zeta_i\zeta_j\,dx^i\odot dx^j, \quad \Omega = d\zeta_i\wedge dx^i,\quad i,j=1,\dots,n.
\ee
In this case the Cartan bundle of $N$ is just $SL(n+1,\R)$, and $M$ is simply the Lie group quotient $SL(n+1,\R)/GL(n,\R)$. The cotractor bundle has zero curvature, and although it is not trivial, the restriction of $\PP(\cT^*)$ to the set $\mathcal{Z}(\tau)\neq 0$ is. %in fact if we instead take $N$ to be the sphere $S^n$ with its standard flat projective structure where the geodesics are great circles, the tractor bundle is trivial. Note that $S^n$ is a double cover of $\RP^n$ which is orientable in all dimensions, and consists of the set of \textit{oriented} lines in $\R^{n+1}$. We obtain it by taking an analogous quotient of $\R^{n+1}$ where points are considered equivalent only up to multiplication by a \textit{positive} number. Replacing $\RP^n$ with $S^n$ allows us to write $cT=S^n\times\R^3$, so
We can thus write $M$ as
\[
M=\{([P],[L])\in\RP^n\times\RP_n\ |\ P\cdot L\neq 0\}.
\]

As discussed in Section \ref{sec:projgeom}, a point $[L]\in\RP_n$ represents a line $[L]\subset\RP^n$ which passes through $P\in\RP^n$ if and only if $L\cdot P=0$. In Chapter \ref{chap:EW_and_toda} we will show that the conformal structure on $M$ can be obtained by demanding that two pairs $([P], [L])$ and $([\tP], [\tL])$ are null--separated if there exists a line which contains the three points $([P], [\tP], [L]\cap [\tL])$.

We also find in the model case that the symplectic form $\Omega$ is parallel with respect to the Levi--Civita connection $^{\bf g}\nabla$ of $g$, meaning that the endomorphism $J$ of $TM$ which relates $g$ and $\Omega$ is also parallel. As a result of this, the distributions $D_\pm$ defined by the two $n$--dimensional eigen--bundles of $J$ are parallel in the sense that $^{\bf g}\nabla_{\xi_1}\xi_2\in\Gamma(D_\pm)$ for all $\xi_1\in \Gamma(TM),\ \xi_2\in\Gamma(D_\pm)$. This makes the distributions \textit{Frobenius integrable}, meaning that the Lie bracket of any two sections of $D_\pm$ is also a section of $D_\pm$, or equivalently (as shown by Frobenius) that each of the two distributions is tangent to a foliation by sub--manifolds of dimension $n$ at every point.

To see that a parallel distribution is necessarily Frobenius integrable, note that the Lie bracket can be written
\[
[\xi_1,\xi_2]= \nabla_{\xi_1}\xi_2- \nabla_{\xi_2}\xi_1\in\Gamma(D)\quad\mbox{for all}\quad \xi_1,\xi_2\in\Gamma(D),
\]
where $D$ is a distribution which is parallel with respect to a connection $\nabla$. The Frobenius integrability of $D_\pm$ makes $(M,g,\Omega)$ not only almost para--K\"ahler but also para--K\"ahler. Further discussion about this distinction can be found in Chapter \ref{chap:c-proj}.