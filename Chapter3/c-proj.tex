%!TEX root = ../thesis.tex
%*******************************************************************************
%****************************** Chapter 3: c-proj *********************************
%*******************************************************************************




\chapter{Para--$c$--projective compactification of $(M,g,\Omega)$} \label{chap:c-proj}
In \cite{CG} the concept of $c$--projective compactification was
defined. It is based on almost $c$--projective geometry \cite{c_proj},
an analogue of projective geometry defined for almost complex
manifolds, i.e., even--dimensional manifolds $M$ carrying a smooth endomorphism $J$ of $TM$ which satisfies $J^2=-Id$. In $c$--projective geometry, the equivalence class of torsion--free connections is replaced by an equivalence class of connections which are adapted to the almost complex structure $J$ in a natural way. In this chapter we discuss a notion of compactification which is modified to the {\it{``para''}} case, i.e. where the endomorphism $J$ squares to $Id$ rather than $-Id$. We show that the natural almost para--complex structure $J$ on any manifold $M$ arising in the projective to Einstein correspondence admits a type of compactification which we call \textit{para}--$c$--projective. The content of this chapter is based on material appearing in \cite{DGW}.

\section{Background and definitions}

\subsection{Almost (para--)complex geometry}

The purpose of this section is to introduce the definitions which are required to state the main results of \cite{CG}.

\begin{defi}
The Nijenhuis tensor of an endomorphism $J$ of $TM$ is defined by
\be \label{eq:Nijenhuis_def}
\mathcal{N}(X,Y):=[X,Y] - [JX,JY] + J([JX,Y] + [X,JY]),
\ee
where $X,Y$ are vector fields on $M$ and $[\cdot\,,\cdot]$ denotes the Lie bracket of vector fields. This is equivalent to
\be \label{eq:Nijenhuis_index_def}
\mathcal{N}^a_{bc}=J^d_{\ [b}\p_{|d|}J^a_{\ c]}-J^d_{\ [b}\p_{c]}J^a_{\ d}.
\ee
\end{defi}

Let $M$ be a complex manifold of (complex) dimension $n$, in the sense of having complex coordinates and complex transition functions. Then multiplication of the coordinates by $i$ defines an endomorphism $J$ of $TM$ which squares to $-Id$, so complex manifolds are a subset of almost complex manifolds. In this case, $J$ has eigenvalues $\pm i$, and the corresponding splitting of $TM$ into eigen--bundles is Frobenius integrable. The Newlander--Nirenberg theorem describes complex manifolds in terms of the Nijenhuis tensor (\ref{eq:Nijenhuis_def}).

\begin{theo}[\cite{CG}]
An almost complex manifold $(M,J)$ is a complex manifold if and only if the Nijenhuis tensor of $J$ vanishes. In this case, we call the almost complex structure $J$ integrable.
\end{theo}


As discussed in chapter \ref{chap:intro}, an endomorphism $J$ which squares to $Id$ defines an analogous splitting of the tangent bundle into sub--bundles with eigenvalues $\pm 1$, and this splitting is also Frobenius integrable if and only if the Nijenhuis tensor of $J$ vanishes. We thus call an almost para--complex structure $J$ with vanishing Nijenhuis tensor a para--complex structure, and say that in this case $J$ is integrable. In all the definitions below, the word \textit{almost} can be removed if the (para--)complex structure $J$ is integrable.

\begin{defi}
A (para--)Hermitian metric on an almost (para--)complex manifold $(M,J)$ is a metric $g$ satisfying
\[
g(J\cdot\,,J\cdot) = \pm g(\cdot\,,\cdot),
\]
where the minus sign corresponds to the ``para'' case. The triple $(M,J,g)$ then defines an almost (para--)Hermitian manifold.
\end{defi}

Note that every (para--)Hermitian manifold has a naturally defined two--form $\Omega(\cdot\,,\cdot)=g(\cdot\,,J\cdot)$ which is (para--)Hermitian in the sense that
\[
\Omega(J\cdot\,,J\cdot) = \pm \Omega(\cdot\,,\cdot),
\]
and can alternatively be specified as $(M,J,\Omega)$ or $(M,g,\Omega)$. An almost (para--)K\"ahler manifold $(M,J,g)$ is a (para--)Hermitian manifold whose associated two--form is closed, meaning $M$ carries compatible complex, pseudo--Riemannian and symplectic structures. The manifolds $M$ arising in the projective to Einstein correspondence are almost para--K\"ahler, and para--K\"ahler when the underlying projective structure is flat \cite{DM}.

\subsection{Almost (para--)CR structures and contact distributions}

\mynote{I think I should probably specialise to the para case at this point.}

\begin{defi}
An almost (para--)CR structure $(\mathcal{Z},H ,J)$ on a manifold $\mathcal{Z}$ is a sub--bundle $H \subset T\mathcal{Z}$ of the tangent bundle together with a fiber--preserving endomorphism $J:H \rightarrow H $ which satisfies $J^2=Id$ or $J^2=-Id$ depending on whether or not we are talking about the ``para'' case.
\end{defi}

We will be interested in the case where $ H $ is a hyperplane distribution on $\mathcal{Z}$; then $(\mathcal{Z}, H ,J)$ is called an almost (para--)CR structure of \textit{hypersurface type}. An almost (para--)complex structure $(M,J)$ of dimension $2n$ defines an almost (para--)CR structure of hypersurface type on any hypersurface $\mathcal{Z}\subset M$ given by the restriction of $J$ to the hyperplane distribution $ H :=T\mathcal{Z}\cap J(T\mathcal{Z})$ on $\mathcal{Z}$. Note that this distribution must have dimension $2n-2$. An almost (para--)CR structure is a (para--)CR structure if and only if the splitting of $ H $ into eigen--bundles induced by $J$ is Frobenius integrable.

We can define the notion of non--degeneracy for an almost (para--)CR structure as follows. The Lie bracket of vector fields induces an antisymmetric $\R$--bilinear operator $\Gamma( H )\times\Gamma( H )\rightarrow\Gamma(T\mathcal{Z}/ H )$ which in fact is also bilinear over smooth functions on $\mathcal{Z}$. This means it is induced by a bundle map $\mathcal{L}: H \times H \rightarrow T\mathcal{Z}/ H $ which is called the \textit{Levi bracket}. Since it takes values in a line bundle it can be thought of as an antisymmetric bilinear form called the \textit{Levi form}. Degeneracy (or not) of the almost (para--)CR structure is defined as degeneracy (or not) of the Levi form. Note that the Levi form also defines a \textit{symmetric} bilinear form $h(\cdot\,,\cdot)=\mathcal{L}(\cdot\,,J\cdot)$ as long as $\mathcal{L}$ is (para--)Hermitian with respect to $J$, and that this symmetric bilinear form is non--degenerate if and only if $\mathcal{L}$ is.

\begin{defi}
A contact structure on a manifold $\mathcal{Z}$ of dimension $2n-1$ is a hyperplane distribution $ H \subset T\mathcal{Z}$ specified as the kernel of a one--form $\theta$ on $\mathcal{Z}$ which satisfies the complete non--integrability condition
\be \label{eq:non_integrability}
\theta\wedge \underbrace{(d\theta \wedge \dots\wedge d\theta)}_{n-1\ \mathrm{times}} \neq 0.
\ee
\end{defi}

\mynote{I also use $\theta$ for the Cartan connection.}

The complete non--integrability condition can be thought of as the opposite of Frobenius integrability of the hyperplane distribution, see for example \cite{arnold}.


\subsection{Connections and (para--)$c$--projective equivalence}

\begin{defi}
A connection on an almost (para--)complex manifold $(M,J)$ is called complex if it preserves $J$.
\end{defi}

Note that, in contrast to a metric connection, it is not always possible to define a complex connection which is torsion--free. In fact, this is possible if and only if the Nijenhuis tensor (\ref{eq:Nijenhuis_def}) of $J$ vanishes. However, one can always define a complex connection whose torsion is equal to the Nijenhuis tensor of $J$ up to a constant multiplicative factor \cite{c_proj}. Such connections are called \textit{minimal}.

\begin{defi}
Two affine connections $\nabla$ and $\ol{\nabla}$ on an almost (para--)complex manifold $(M,J)$ are called (para--)$c$--projectively equivalent if there is a one--form $\Upsilon_a$ on $M$ such that their components $\Gamma^a_{bc}$ and $\ol{\Gamma}^a_{bc}$ are related by
\be \label{eq:c-proj_change}
\ol{\Gamma}^a_{bc} - \Gamma^a_{bc} = \delta_{b}^{a}\Upsilon_{c}+\delta_{c}^{a}\Upsilon_{b} \pm (\Upsilon_d J^d_{\ b} J^a_{\ c} + \Upsilon_d J^d_{\ c} J^a_{\ b}),
\ee
where the $+$ corresponds to the case $J^2=Id$ and the $-$ corresponds to the case $J^2=-Id$.
\end{defi}

Note that the para--$c$--projective change of connection differs from the $c$--projective case in the signs of some of the terms, to account for the fact that $J$ squares to the $Id$ rather than $-Id$. It is easy to show that if $\nabla$ is complex then so is $\ol{\nabla}$, and the index symmetry of the right hand side of (\ref{eq:c-proj_change}) means that if $\nabla$ is minimal then so is $\ol{\nabla}$. An almost (para--)$c$--projective structure on a manifold $M$ comprises an almost (para--)complex structure $J$ and a (para--)$c$--projective equivalence class $[\nabla]$ of complex minimal connections.


\subsection{Para--$c$--projective compactification}

We now specialise to the ``para'' case, where $J^2=Id$. Note that all the corresponding results for $J^2=-Id$ can be found in the original paper \cite{CG}.

\begin{defi}
\label{defi_1}  Let $(M,J)$ be an almost para--complex manifold, and let $\nabla$ be a complex minimal connection. The structure $(M,J)$ admits a para--$c$--projective compactification to a manifold with boundary $\ol{M}=M\cup\p M$
if there exists a function $T:\ol{M}\rightarrow \R$ such that $\mathcal{Z}(T)$ is the boundary
$\p M\subset \ol{M}$, the differential $dT$ does not vanish on $\p M$, and the connection $\ol{\nabla}$ related to $\nabla$ by (\ref{eq:c-proj_change}) with $\Upsilon = dT/(2T)$ extends to $\ol{M}$.
\end{defi}



It follows easily from this definition that the endomorphism $J$ on $M$ naturally extends to all of $\ol{M}$ by parallel transport with respect to $\ol{\nabla}$. It thus defines an almost para--CR structure on the hyperplane distribution $ H $ defined by $ H _x:=T_x\p M \cap J(T_x \p M)$ for all $x\in\p M$. It can be shown (see lemma 5 of \cite{CG} and modify to the case $J^2=Id$) that this almost para--CR structure is non--degenerate if and only if for any local defining function $T$ the one--form $\theta=dT\circ J$, whose restriction to $\p M$ has kernel $ H $, satisfies the complete non--integrability condition (\ref{eq:non_integrability}) making $ H $ a contact distribution on $\p M$.

To see this, first note that $\theta(X)=0\ \forall\ X\in\Gamma( H )$ implies $d\theta(\cdot\,,\cdot)=-\theta([\cdot\,,\cdot])$, so the restriction of $d\theta$ to $ H \times H $ represents the Levi form $\mathcal{L}$. This means that the almost para--CR structure on $\p M$ is non--degenerate if and only if the restriction of $d\theta(X,\cdot)$ to $ H $ is non--zero for all non--zero $X\in\Gamma( H )$. But this is equivalent to the non--integrability condition (\ref{eq:non_integrability}).

%Note also the distinction between the integrability (or not) of the ($2n-2$)--dimensional distribution $ H $ and the integrability (or not) of the two ($n-1$)--dimensional distributions given by the eigenvalue decomposition of $ H $ under the action of $J$.

Another result of lemma 5 of \cite{CG} is that $d\theta$ is Hermitian on $\p M$ if and only if the Nijenhuis tensor (\ref{eq:Nijenhuis_def}) of $J$ takes so--called \textit{asymptotically tangential values}. This is equivalent to the following statement in index notation:
\be
\label{Nijenhuis_condition}
\Big({\mathcal{N}^{a}}_{bc}\nabla_a T\Big)\Big|_{T=0}=0.  \ee
Note in particular that Hermiticity of $d\theta$ on $\p M$ implies Hermiticity of $d\theta$ on $ H $, and hence the existence of a non--degenerate metric $h(\cdot\,,\cdot)=d\theta(\cdot\,,J\cdot)|_ H $ on $ H $. Both of these facts also apply in the ``para'' case.

\mynote{Could check this and maybe put the proof in. It is only a couple of paragraphs.}

Although $c$--projective compactification is defined for any almost complex manifold, the definition can be applied to pseudo--Riemannian metrics $g$ which are Hermitian with respect to the almost complex structure so long as there exists a connection which preserves both $g$ and $J$ and has minimal torsion. Such Hermitian metrics are said to be \textit{admissible}.  Note that such a connection, if it exists, is uniquely defined, since the conditions that it be complex and minimal determine its torsion. It is thus given by the Levi--Civita connection of $g$ plus a constant multiple of the Nijenhuis tensor (\ref{eq:Nijenhuis_def}) of $J$.

The first main result of \cite{CG} is  Theorem 8 in this reference, which gives a local form for an admissible Hermitian metric which is sufficient for the corresponding $c$--projective structure to be $c$--projectively compact. The theorem is stated below, adapted to the para--$c$--projective case. The proof can be obtained by a trivial adaptation of the arguments in
\cite{CG}, and so further details may be obtained from that source.
\begin{theo}[\cite{CG}] \label{CGthm}
Let $\ol{M}$ be a smooth manifold with boundary $\p M$ and interior $M$. Let $J$ be an almost para--complex structure on $\ol{M}$, such that $\p M$ is non--degenerate and the Nijenhuis tensor $\mathcal{N}$ of $J$ has asymptotically tangential values. Let $g$ be an admissible pseudo--Riemannian Hermitian metric on $M$. For a local defining function $T$ for the boundary defined on an open 
subset ${\mathcal U}\subset \ol{M}$, put $\theta=dT\circ J$ and, given a non--zero real 
constant $C$, define a Hermitian tensor field $h_{T,C}$ on 
${\mathcal U}\cap M$ by
\[
h_{T,C}:=Tg+\frac{C}{T}(dT^2-\theta^2).
\]
Suppose that for each $x\in\p M$ there is an open neighbourhood 
${\mathcal{U}}$ of $x$ in $\ol{M}$, a local defining function $T$ defined on 
${\mathcal{U}}$, and a non--zero constant $C$ such that
\begin{itemize}
\item $h_{T,C}$ admits a smooth extension to all of $\mathcal{U}$
\item for all vector fields $X,Y$ on $U$ with $dT(Y)=\theta(Y)=0$, the function $h_{T,C}(X,JY)$ approaches $Cd\theta(X,Y)$ at the boundary.
\end{itemize}
Then $g$ is $c$--projectively compact.
\end{theo}
Note that the statement in  Theorem \ref{CGthm} does not depend on the choice of $T$. Different choices of $T$ result in rescalings of the contact form $\theta$ on the boundary by a nowhere vanishing function.


\section{Compactifying the Dunajski--Mettler Class} 

In section \ref{sec:trac_construction} it was shown that the manifolds $M$ arising in the projective to Einstein correspondence can be identified with the projectivised cotractor bundle of $N$ where the zero locus of the canonical density $\tau$ has been removed. Note that by construction in section \ref{sec:trac_construction} it is easily verified that the zero locus of $\tau$ is a smoothly embedded hypersurface in $\mathcal{M}$,  and from (\ref{eq:T*N_is_M}) it follows at once that this may be identified with the total space of the fibrewise projectivisation $\mathbb{P}(T^*N)$ (which is well known to have an almost para--CR structure). \mynote{Is there a way to justify this "well known" fact?}

As noted in chapter \ref{chap:intro}, in the special case where $N=\RP^n$ and $[\nabla]$ is projectively flat, the manifold $M=SL(n+1, \R)/GL(n, \R)$ can be identified with the projectivisation of $\R^{n+1}\times \R_{n+1}\setminus {\mathcal Z}$, where ${\mathcal Z}$ denotes the set of incident pairs (point, hyperplane). The compactification procedure described in the theorem \ref{our_thm} below will, for the model, attach these incident pairs back to $M$, and more generally (in case of a curved projective structure $(N,[\nabla])$) will attach the zero locus of $\tau$ back into $\mathbb{P}(\cT^*)$. The boundary $\p M \cong \mathcal{Z}(\tau)$ from definition \ref{defi_1} will play the role of a submanifold
separating two open sets in $\mathbb{P}(\cT^*)$ which have $\tau>0$ and $\tau<0$ respectively. The method of the proof will be to show that near
the boundary ${\mathcal{Z}}(\tau)=0$ of $\ol{M}$ the metric 
(\ref{eq:coord_form}) can be put in the local normal form of theorem 
\ref{CGthm}.

%Before we state the theorem, we note that the Libermann connection $\nabla^{L}$ \cite{Lieb} associated to $(M,g,\Omega)$ is given by
%\be
%\label{lib}
%{\nabla^L}_a X_b={\nabla^{\bf g}}_a X_b-{G^c}_{ab} X_c,\quad \mbox{where}\quad
%{{G^c}_{ab}}=-{{\Omega}^{cd}}{\nabla^{\bf g}}_d {\Omega_{ab}}
%\ee
%and $\nabla^{\bf g}$ is the Levi--Civita connection of $g$.

%\mynote{Where do we use this? And is $G^c_{ab}$ Nijenhuis?}

%This connection is metric, has minimal torsion, and preserves the almost para--complex structure $J$. It thus belongs to a para--$c$--projective equivalence class which we will show to be compactifiable in the sense of definition \ref{defi_1}.

\begin{theo}
\label{our_thm}
The Einstein almost para--K\"ahler metric $(M, g, \Omega)$ given by 
(\ref{eq:coord_form}) admits a para--$c$--projective compactification
$\ol{M}$. The structure on the
$(2n-1)$--dimensional boundary $\p M\cong \mathbb{P}(T^* N)$ of $\ol{M}$ includes a contact structure together with a conformal structure 
and a para--CR structure
defined on the contact distribution.
\end{theo}
\mynote{Here we stated that $\p M\cong \mathbb{P}(T^* N)$ but unless I can justify why this carries a para--CR structure it might be better to leave this out or put $\p M\cong \mathcal{Z}(\tau)$.}
\noindent
{\bf Proof.}
In the proof below we shall explicitly construct the boundary $\p M$ together with the contact structure and the associated conformal structure on the contact distribution. We shall
first deal with the model $M=SL(n+1)/GL(n)$, and then explain how the curvature
of $(N, [\nabla])$ modifies the compactification.

In the model case we can define coordinates $x^i$ on $N=\mathbb{RP}^n$ by taking $X=(1,x^1,\dots,x^n)$, where $(X^0,\dots,X^n)$ are homogeneous coordinates and we are working in an open set where $X^0\neq 0$. The ${x^i}$ are flat coordinates, so the connection components (and hence the Schouten tensor) vanish and (\ref{eq:coord_form}) reduces to
\be
\label{model_metric}
g=d\zeta_i\odot dx^i + \zeta_i\zeta_jdx^i\odot dx^j, \qquad
\Omega=d \zeta_i\wedge d x^i \quad
\mbox{where}\quad
i, j =1, \dots, n.
\ee
As noted in section \ref{sec:trac_construction}, we can use (\ref{eq:T*N_is_M}) relate the affine coordinates $\zeta_i$ on the fibres of $T^*N$ to the tractor coordinates (\ref{eq:T*coords}) by setting $\zeta_i=\mu_i/\tau$ on the complement of the zero locus ${\mathcal Z}(\tau)$ of $\tau$.

Now consider an open set  ${\mathcal U}\subset M$ given
by  $\zeta_ix^i>0$, and define the function $T$ on ${\mathcal U}$ by
\be
\label{formula_for_T}
T=\frac{1}{\zeta_i x^i}.
\ee
We shall attach a boundary  $\p \mathcal{U}$ to the open set $\mathcal{U}$ 
such that $T$ extends to a function $\ol{T}$ on $\mathcal{U}\cup \p \mathcal{U}$, and
$\ol{T}$ is  the defining function for this boundary.
We then investigate the geometry on $M$ in the limit $T\rightarrow 0$.
It is clear from above that
the zero locus of $\ol{T}$ will be contained in the zero locus $\mathcal{Z}(\tau)$ of $\tau$, and
therefore belongs to the boundary of $\ol{M}$. We will 
use $\ol{T}$ as a defining function for $\ol{M}$ in an open set $\ol{\mathcal{U}}\subset\ol{M}$.
The strategy of the proof is to extend $T$ to a coordinate system on 
$\mathcal{U}$, such that near the boundary the metric $g$ takes a form
as in Theorem \ref{CGthm}.


First define $\theta\in \Lambda^1(\ol{M})$ $\ov{M}$
by 
\be
\label{def_theta}
V\hook \theta=J(V)\hook d T, \quad\mbox{or equivalently}\quad 
\theta_a=\Omega_{ac}g^{bc}{{\nabla}^{\bf g}}_b T, \quad a, b, c=1, \dots, 2n
\ee
where $J$ is the para--complex structure of $(g,  \Omega)$. Using (\ref{model_metric}) this  gives
\[
\theta=2T(1-T)\zeta_id x^i-dT.
\]
We need $n$  open sets $U_1, \dots, U_n$ such that $\zeta_k\neq 0$ on $U_k$
to cover the zero locus of $T$. Here we chose $k=n$, and use
a coordinate system given by
\[
(T, Z_1, \dots, Z_{n-1}, X^1, \dots,
 X^{n-1}, Y),
\] 
where $T$ is
given by (\ref{formula_for_T}) and
\[
Z_A=\frac {\zeta_A}{\zeta_n}, \quad X^A=x^A, \quad Y=x^{n}, \quad\mbox{where}\quad
A=1, \dots, n-1.
\]
We compute
\[
\theta=2(1-T)\frac{dY+Z_AdX^A}{K}-dT, \quad
\zeta_n=\frac{1}{KT}, \quad \mbox{where}\quad K\equiv Y+Z_AX^A,
\]
and substitute
\[
\zeta_i dx^i=\frac{1}{KT}(dY+Z_AdX^A)
\]
into (\ref{model_metric}). This gives
\be
\label{CG_Form}
g=\frac{\theta^2-dT^2}{4T^2}+\frac{1}{T}h,
\ee
where 
\[
h=\frac{1}{4(1-T)}(\theta^2-dT^2)+\frac{1}{K}\Big(dZ_A\odot dX^A-\frac{1}{2(1-T)}X^A dZ_A\odot(\theta+dT)\Big)
\]
is regular at the boundary $T=0$. This is in agreement with the 
asymptotic form in Theorem \ref{CGthm} (see \cite{CG} for further details).

%Note that (\ref{def_theta}) defines the one form $\theta$ on the boundary $T=0$  only up to a overall multiple of a positive function.
The restriction $h$ to $\p M$ gives a metric on a distribution ${ H }=\mbox{Ker} (\theta|_{T=0})$
\begin{gather}
\theta|_{T=0}=2\frac{dY+Z_A dX^A}{Y+Z_AX^A}, \nonumber \\
h_0=\frac{1}{4}{(\theta|_{T=0})}^2+\frac{1}{2(Y+Z_AX^A)}(2dZ_A\odot dX^A-X^AdZ_A\odot(\theta|_{T=0})). \label{h000}
\end{gather}
 Note that $T$ is only defined up to multiplication by a positive function. Changing the defining function in this way results in a conformal rescalling of $\theta|_{T=0}$, thus the metric on the contact distribution is also defined up to an overall conformal scale. We shall choose the scale so that
the contact form is given  by $\theta_0\equiv K\theta|_{T=0}$ on $T(\p M)$,
with the metric on ${ H }$ given by
%\footnote{The singular denominator $K^{-1}$ may be avoided by adopting the Pfaff %coordinates
%\[
%y=\ln{(Y+Z_iX^i)}, \quad z_i=Z_i, \quad x_i=-\frac{X^i}{Y+Z_iX^i},
%\]
%which yields
%\[
%\theta_0=2(dy+x^idz_i), \quad h_{ H }=-(dx^i+x^idy)\odot dz_i.
%\]
%}
\be
\label{on_distri}
h_{ H }=dZ_A\odot dX^A.
\ee

We now move on to deal with the
curved case where the metric on $M$ is given by 
(\ref{eq:coord_form}).
%g=\left(d\zeta_a-\left(\Gamma_{ab}^c \zeta_c- \zeta_a\zeta_b- \Rho_{ab}\right)\d %x^b\right)\odot \d x^a%
The coordinate system $(T, Z_A, X^A, Y)$ is as above, and
the one--form $\theta$ in (\ref{def_theta}) is given by
\[
\theta=2T(1-T)\zeta_idx^i-dT+2T^2(\Rho_{ij}-\Gamma_{ij}^k\zeta_k)x^idx^j,
\]
or in the $(T, Z_A, X^A, Y)$ coordinates,
\[
\begin{split}
\theta=\ 2&(1-T)\frac{Z_AdX^A+dY}{K} - dT \\
+& 2T^2\Bigg[\bigg(\Rho_{AB}-\frac{\Gamma^C_{AB}Z_C+\Gamma^n_{AB}}{TK}\bigg)X^AdX^B 
+\bigg(\Rho_{nB}-\frac{\Gamma^C_{nB}Z_C+\Gamma^n_{nB}}{TK}\bigg)YdX^B \\
+& \bigg(\Rho_{An}-\frac{\Gamma^C_{An}Z_C+\Gamma^n_{An}}{TK}\bigg)X^AdY 
+\bigg(\Rho_{nn}-\frac{\Gamma^C_{nn}Z_C+\Gamma^n_{nn}}{TK}\bigg)YdY\Bigg].
\end{split}
\]
%We find that $g$ in these coordinates is given by
%%\begin{split}
%g=
%\end{split}
%\]

Guided by the formula (\ref{CG_Form}) we define
\[
h=Tg-\frac{1}{4T}(\theta^2-dT^2),
\]
which we find to be
\begin{equation*}
\begin{split}
h=&
\frac{1}{4(1-T)}(\theta^2-dT^2)+\frac{1}{K}\Big(dZ_A\odot dX^A-\frac{1}{2(1-T)}
X^A dZ_A\odot (\theta+dT)\Big)\\
&-\frac{1}{K}\Big(
(\Gamma_{AB}^CZ_C+\Gamma_{AB}^n)dX^A\odot dX^B+
(\Gamma_{nn}^CZ_C+\Gamma_{nn}^n)dY\odot dY \\
&\qquad\quad + 2(\Gamma_{An}^CZ_C+\Gamma_{An}^n)dX^A\odot dY\Big)\\
&+T(\Rho_{AB}dX^A\odot dX^B+2\Rho_{An}dX^A\odot dY+\Rho_{nn}dY\odot dY).
\end{split}
\end{equation*}
This is  smooth as $T\rightarrow 0$.

Restricting $h$ to $T=0$ yields a metric which differs from
(\ref{h000}) by the curved contribution given by the components of the  connection, but not the Schouten tensor. Substituting $dY=K\theta|_{T=0}/2-Z_AdX^A$, disregarding the terms involving $\theta|_{T=0}$ in $h$, and conformally rescalling by 
$K$ yields the metric
\begin{eqnarray}
\label{met_th}
h_{ H }&=&(dZ_A-\Theta_{AB}dX^B)\odot dX^A,\quad
\mbox{where}\\
\Theta_{AB}&=&\Gamma_{AB}^CZ_C+\Gamma_{AB}^n+
(\Gamma_{nn}^CZ_C+\Gamma_{nn}^n)Z_AZ_B-
2(\Gamma_{An}^CZ_C+\Gamma_{An}^n)Z_B\nonumber
\end{eqnarray}
defined on the contact distribution ${ H }=\mbox{Ker}(\theta_0)$, 
where $\theta_0=2(dY+Z_AdX^A)$.

We now invoke theorem \ref{CGthm},  verifying
by explicit computation that the remaining two conditions are satisfied. The first of these conditions is that the metric $h_{ H }$ is compatible with the Levi--form of the almost para--CR structure induced on $\p M$ by $J$, i.e.
\be
\label{boundary_compatibility}
h_H(X, Y)=Cd\theta_0(JX, Y), \quad\mbox{for}\quad X\in H .
\ee
\mynote{I think this is slightly different but I guess equivalent to the form given in the theorem. Might need a comment?}
\mynote{Okay I'm not sure if $C$ should be negative, i.e. if the statement in the theorem is the better one because $d\theta$ is like $-\mathcal{L}$ but the statement in the theorem is the same as in \cite{CG} i.e. in the case $J^2=-Id$. Would be good to check this.}
\mynote{In fact in the theorem we are not talking about $h$ and $d\theta$ restricted to $ H $, we are talking about $h$ and $d\theta$ on $\p M$, so maybe what I showed here is not the full condition?}
The second is that the Nijenhuis tensor takes asymptotically tangential values, i.e. that (\ref{Nijenhuis_condition}) is satisfied.

Both of these can be checked by computing the almost para--complex structure $J$ in the $(T, Z_A, X^A, Y)$ coordinates. We find
\be
\begin{split}
\label{J_T=0}
J|_{T=0}=\ &-\frac{\p}{\p X^A}\otimes dX^A +\frac{\p}{\p Y}\otimes dY + \frac{\p}{\p Z_A} \otimes dZ_A + \frac{\p}{\p T}\otimes dT \\
&-\frac{Z_B}{K}\frac{\p}{\p T}\otimes dX^B - \frac{1}{K}\frac{\p}{\p T}\otimes dY \\
&-\big(\Gamma^D_{AB}Z_D+\Gamma^n_{AB}\big)\frac{\p}{\p Z_A} \otimes dX^B + \big(\Gamma^D_{nB}Z_D+\Gamma^n_{nB}\big)Z_C\frac{\p}{\p Z_C} \otimes dX^B \\
&-\big(\Gamma^D_{An}Z_D+\Gamma^n_{An}\big)\frac{\p}{\p Z_A} \otimes dY
+\big(\Gamma^D_{nn}Z_D+\Gamma^n_{nn}\big)Z_C\frac{\p}{\p Z_C} \otimes dY.
\end{split}
\ee
Restricting to vectors in $ H $ amounts to substituting $dY=\theta_0/2-Z_AdX^A$ and disregarding the terms involving $\theta_0$ as above, so that
\[
\begin{split}
J|_{ H }= &-\frac{\p}{\p X^A}\otimes dX^A +Z_A\frac{\p}{\p Y}\otimes dX^A + \frac{\p}{\p Z_A} \otimes dZ_A + \frac{\p}{\p T}\otimes dT \\
&-\frac{2Z_B}{K}\frac{\p}{\p T}\otimes dX^B -\Theta_{AB}\frac{\p}{\p Z_A}\otimes dX^B
\end{split}
\]
and (\ref{boundary_compatibility}) is satisfied.

For the Nijenhuis condition, we use the formula (\ref{eq:Nijenhuis_index_def}). Note that we need only consider components of this with $a=T$, and thus only need to work with the $\p/\p T$ components of $J$ to find the terms which look like $\p J$. This is a one--form which we shall call $J^{(T)}$ and find to be
\[
\begin{split}
J^{(T)}=&\bigg(-\frac{Z_B}{K} + \frac{T[2Z_B + (\Gamma^D_{AB}Z_D+\Gamma^n_{AB})X^A + (\Gamma^D_{nB}Z_D+\Gamma^n_{nB})Y]}{K} \\
&\quad - T^2[\Rho_{AB}X^A+\Rho_{nB}Y]\bigg)dX^B \\
&+\bigg(-\frac{1}{K} + \frac{T[2 + (\Gamma^D_{An}Z_D+\Gamma^n_{An})X^A + (\Gamma^D_{nn}Z_D+\Gamma^n_{nn})Y]}{K} \\
&\qquad - T^2[\Rho_{An}X^A+\Rho_{nn}Y]\bigg)dY.
\end{split}
\]
Note that this agrees with (\ref{J_T=0}) when $T=0$. We use it to calculate ${\mathcal{N}^{a}}_{bc}\nabla_a T$, dropping terms which vanish when $T=0$ to verify (\ref{Nijenhuis_condition}).

%The only non--vanising components of the torsion of the Libermann connection (\ref{lib}) on
%the boundary $\p M$ are tangential to $\p M$,
% i. e.
%\[
%\Big({G^{a}}_{bc}{\nabla^{\bf g}}_a T\Big)|_{T=0}=0.
%\]
%The statement now follows as for the Liberman connection  (\ref{lib}) the Nijenhuis tensor is a constant multiple of $G_{ab}^c$.
\koniec

\subsection{Two--dimensional projective structures}
In the case if $n=2$ the coordinates on $\p M$ are $(X, Y, Z)$,  and (\ref{met_th}) yields
\[
h_{ H }=dZ\odot dX-[\Gamma_{11}^2+(\Gamma_{11}^1-2\Gamma_{12}^2)Z+(\Gamma_{22}^2-2\Gamma_{12}^2)Z^2+
\Gamma_{22}^1Z^3]dX\odot dX,
\]
which is transparently invariant under the projective changes 
\[
\Gamma_{ij}^k\longrightarrow \Gamma_{ij}^k+\delta^k_i\Upsilon_j+\delta^k_j\Upsilon_i
\]
of $\nabla$.
In the  two-dimensional case the 
projective
structures $(N, [\nabla])$ are equivalent to second order ODEs which are cubic in
the first derivatives (see, e.g. \cite{BDE})
\begin{equation}
\label{ODE}
\frac{d^2 Y}{d X^2}=\Gamma^1_{22}\Big(\frac{d Y}{d X}\Big)^3
+(2\Gamma^1_{12}-\Gamma^2_{22})\Big(\frac{d Y}{d X}\Big)^2
+(\Gamma^1_{11}-2\Gamma^2_{12})\Big(\frac{d Y}{d X}\Big)-
\Gamma^2_{11},
\end{equation}
where the integral curves of (\ref{ODE}) are the unparametrised geodesics of $\nabla$. 
The integral curves $C$ of $(\ref{ODE})$ are integral submanifolds
of a  differential
ideal ${\mathcal I}=<\theta_0, \theta_1>$, where
\[
\theta_0=dY+ZdX, \quad \theta_1=dZ-\Big(\Gamma_{11}^2+(\Gamma_{11}^1-2\Gamma_{12}^2)Z+(\Gamma_{22}^2-2\Gamma_{12}^1)Z^2+
\Gamma_{22}^1Z^3\Big)dX
\]
are one--forms on a three--dimensional manifold $B=\PP(T^*N)$ with local coordinates $(X, Y, Z)$. If $f:C\rightarrow B$ is an immersion, then $f^*(\theta_0)=0, f^*(\theta_1)=0$ is equivalent
to (\ref{ODE}) as long as $\theta_2\equiv dX$ does not vanish. In terms of these three one--forms
the contact structure, and the metric on the contact distribution are given by
$
\theta_0,  h_{ H }=\theta_1\odot\theta_2.
$

\section{An alternative approach to theorem \ref{our_thm}}

\begin{itemize}
\item Recall that holonomy group is...
\item Holonomy of a Cartan connection
\item A parallel section of a tractor bundle induces a holonomy reduction and orbit decomposition.
\item From a parallel metric on a projective tractor bundle you get Einstein geometries in the open orbits. Is this right? Is $\cT\oplus\cT^*$ still a projective tractor bundle? Or is our metric actually just on $\cT$?
\item In the model case, if you replace line projectivisation by ray projectivisation, the tractor bundle is trivial so $M$ is $\RP^n\times\RP^{n*}$, aka $\PP(\mathcal{T})\times\PP(\mathcal{T}^*)$. $\mathcal{T}\times\mathcal{T}^*$ has a dual pairing which induces a parallel tractor metric, which induces the Einstein metrics in the open orbits. Is it clear that this is the same metric as we get above in the general case? It is clear that both are based on a natural contraction between horizontal and vertical parts of the tangent bundle of a bundle. So probably.
\item Neat description of the conformal structure for $n=2$? (does it generalise to $n>2$?) This would lead nicely into the ASD'ty section.
\end{itemize}

It would be possible to show that the structures $(M,g,\Omega)$ arising in the projective to Einstein correspondence are para--$c$--projectively compact using a tractor approach. By our construction above it follows that $\mathcal{M}$ has a canonical para--$c$--projective geometry. In the notation of section \ref{sec:trac_construction}, $\pi^*{\cT}\oplus
\pi^*{\cT^*}$ is the corresponding para--$c$--projective tractor bundle
and this has a canonical tractor connection that trivially
extends (in fibre directions) the pull back of the projective
connection (that is available in horizontal directions). The
dual pairing between $\pi^*{\cT}$ and $\pi^*{\cT^*}$ determines a
fibre metric and compatible symplectic form on the bundle
$\pi^*{\cT}\oplus \pi^*{\cT^*}$ and this is obviously preserved by the
connection. What remains is to show that the tractor connection so
constructed satisfies properties that mean that it is {\em normal} in
the sense defined in e.g.\ \cite{CS-book}. With this established then
the main results then follow from the general holonomy theory in
\cite{CGH-duke}.


\subsection{The alternative approach applied in the model case}

In this section we describe here the flat (in the sense of parabolic
geometries) model \cite{CDT13, DM} of our construction in tractor
terms.

The flat projective structure on $N=\RP^n$ gives rise to 
the neutral signature para--K\"ahler Einstein metric on $M=SL(n+1)/GL(n)$
\be
\label{DM_metric}
g=d\zeta_i\odot dx^i+(\zeta_i dx^i)^2, \quad \Omega= d\zeta_i\wedge dx^i, \quad\mbox{where}\quad
i, j, \dots =1, \dots, n.
\ee
In \cite{DM}, \S 7.1 it was explained how this homogeneous model
corresponds to the projectivised co-tractor bundle of $\RP^n$, with
an $\RP_{n-1}$ removed from each $\RP_n$ fiber. This $\RP_{n-1}$
corresponds to incident pairs of points and hyperplanes in $\R^{n+1}\times\R_{n+1}$.


Here we shall instead take $N$ to be the sphere $S^n$ with its standard
projective structure as this is orientable in all dimensions and, more
importantly, on this (double cover of $\mathbb{R}\mathbb{P}^n$) the
tractor bundle is trivial, and this simplifies the discussion.  The
underlying space of the (compactified) model of dimension $2n$ is
$S^n\times S_n$ where both $S_n$ and $S^n$ denote spheres that are
dual as we shall explain.

Consider first two vector spaces each isomorphic to $\mathbb{R}^{n+1}$:
$$
V\cong \mathbb{R}^{n+1} \qquad \mbox W \cong \mathbb{R}^{n+1}
$$
and view each as a representation space for an $SL(n+1,\mathbb{R})$
action.  So $G:= SL(V)\times SL(W)$ acts on $V\times W$. (Note that we may
wlog consider $V$ and $W$ as respectively the $\pm 1$ eigenspaces of
the single vector space $\mathbb{V}:=V\oplus W$ equipped with a
$\mathbb{J}$ s.t. $\mathbb{J}^2=1$.)

Now the action of $SL(V)$ descends to a transitive action on the ray
projectivisation $\mathbb{P}_+(V)$ and similarly $SL(W)$ acts
transitively on $\mathbb{P}_+(W)$. Thus  $G:= SL(V)\times SL(W)$ acts transitively on the manifold
$$
{\mathcal M}:= \mathbb{P}_+(V) \times \mathbb{P}_+(W).
$$
We can represent an element of $\mathcal{M}$ in terms of pairs of homogeneous coordinates
$([Y],[Z])$ where $0\neq Y\in V$ and $0\neq Z\in W$.

Note that as a smooth manifold $\mathcal{M}=S^n\times S^n$, but as a homogeneous manifold it is
$$
G/P = \big( SL(V)/P_X \big)\times \big( SL(W)/P_U \big)
$$
where $P_X$ (resp.\ $P_U$) is the parabolic subgroup in $SL(V)$
that stabilises a point $[X]$ in $\mathbb{P}_+(V)$ (resp.\ $[U] \in \mathbb{P}_+(W)$
), and $P$ is the group product $P_X\times P_W$ which itself is a
parabolic subgroup of the semisimple group $G$.

Now introduce an additional structure which breaks the $G$
symmetry. 
Namely we fix an isomorphism
$$
I:W\to V^*
$$
where $V^*$ denotes the dual space to $V$. The subgroup $H\cong SL(n+1,\mathbb{R})$ of $G$
that fixes this may be identified with $SL(V)$ which acts on a pair
$(Y,Z)\in V\times V^*$ by the defining representation and on the first
factor and by the dual representation on the second factor.

Given this structure we may now (suppress $I$ and) view ${\mathcal{M}}$ as
consisting of pairs $([X],[U])$ where $0\neq X\in V$ and $0\neq U\in
V^*$. That is 
$$
{\mathcal{M}}= \mathbb{P}_+(V) \times \mathbb{P}_+(V^*).
$$

This is useful as follows: Each element $[U]$ in
$\mathbb{P}_+(V^*)$ determines an oriented  hyperplane in $V$ and each $[X]\in
\mathbb{P}_+(V)$ an oriented line in $V$.  So now we consider the $H$
action on $M$. This has two open orbits and a closed orbit. The last
is the incidence space 
$$
\mathcal{Z}=\{ ([X],[U])\in \mathcal{M} \mid U(X)=0 \} 
$$
which sits as smooth orientable separating hypersurface in $\mathcal{M}$. Then there are the open orbits
$$
M_+=\{ ([X],[U])\in \mathcal{M} \mid U(X)>0 \} \qquad \mbox{and} \qquad
M_-=\{ ([X],[U])\in \mathcal{M} \mid U(X)<0 \}.
$$
We may think of $\mathcal{Z}$ as the `boundary' (at infinity) for the open orbits $M_\pm$.

We now describe the geometries on the orbits. The claim is that there
are Einstein metrics in $M_\pm$, while $\mathcal{Z}$ is well known as
the model for so-called contact Langrangian (or sometimes called
para--CR) geometry, this is a real analogue of hypersurface type CR
geometry.

First observe that $N_V:=\mathbb{P}_+(V)$
is the flat model of projective
geometry. So in particular we have
$$
0\to \ce_V(-1)\stackrel{X}{\to}\cT_V\to TN_V(-1)\to 0
$$
where $\cT_V$ is the projective tractor bundle on $N_V$ and $X$ is the
tautological section of $\cT(1)$, which coincides with the canonical tractor.
Similarly there a sequence on
$N^W:= \mathbb{P}_+(V^*)$
\begin{equation}\label{useful}
0\to \ce^W(-1)\stackrel{U}{\to}\cT^W \to TN^W(-1)\to 0 .
\end{equation}

There is a natural tractor bundle $\mathcal{T}:=\cT_V\oplus \cT^W $ on $M$. 
Where $X$ and $U$ are not incident this induces a metric on $M$ as
follows. Observe that, at a point $([X],[U])$ where $X\hook U \neq 0$, the  tractor field  $U$ splits the first sequence by $\nu\in \Gamma (\ce(-1,0))$ defined by
$$
\nu:=U/\tau
$$
with $\tau:=X\hook U$ (and where we have used an obvious weight
notation). This follows as $X\hook \nu=1$. Similarly
$$
x:=X/\tau \in \Gamma (\ce(0,-1))
$$
splits the second short exact sequence because $x \hook U=1$. Thus we
obtain a neutral signature metric on $TN_V\oplus TN^W$ by these
two steps: First, using  these splittings yields a bundle
monomorphism
$$
TN_V(-1,0)\oplus TN^W(0,-1) \to \cT_V\oplus \cT^W .
$$ Second, this gives a symmetric form $\boldsymbol{g}$ and symplectic form
$\boldsymbol{\Omega}$ on $TN_V(-1,0)\oplus TN^W(0,-1)$ by then using
the canonical metric and symplectic form on $\cT_V\oplus \cT^W $ given
by the duality of $\cT_V$ and $ \cT^W$. Thus $\boldsymbol{g}\in \Gamma
(S^2T^*M(1,1))$ and $\boldsymbol{\Omega}\in \Gamma
(\Lambda^2T^*M(1,1))$. Then set
$$ g:=\frac{1}{\tau}\boldsymbol{g} \qquad \mbox{and} \qquad  \Omega:=\frac{1}{\tau}\boldsymbol{\Omega}.
$$
The metric $g$ is easily seen to have neutral signature.  It is
Einstein because the tractor metric on $\mathcal{T}$ is parallel for the
tractor connection (see \cite{CGH-duke} for the analogous c-projective
case). The tractor connection arises from the usual parallel transport on the
vector space $V\oplus V^*$ viewed as an affine manifold.

