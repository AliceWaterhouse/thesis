%!TEX root = ../thesis.tex
%*******************************************************************************
%****************************** Chapter 3: KK_lift *********************************
%*******************************************************************************

\chapter{Kaluza--Klein lift to an $S^1$--bundle over $M$}
\label{chap:KK_lift}
In this chapter, we show that there is a canonical Einstein of metric on an $\R^*$--bundle over $M$, with a connection whose curvature is the pull--back
of the symplectic structure from $M$. This metric is interesting in the context of Kaluza-Klein theory.

\subsection{Flat case}

We first note that taking the flat projective structure on $\mathbb{RP}^{n}$
results in a one--parameter family of $2n$-dimensional Einstein spaces $M$ which are Kaluza-Klein
reductions of quadrics in $\mathbb{R}^{n+1,n+1}$. The metric and symplectic form on $M$ reduce to 
\begin{eqnarray*}
g_{\Lambda} & = & {d}x^{A}\odot {d}z_{A}+\Lambda z_{A}z_B{d}x^{A}\odot dx^B\\
\Omega_{\Lambda} & = & {d}z_{A}\wedge{d}x^{A},\qquad A,B=0,\dots,n-1.
\end{eqnarray*}

\begin{prop}The Einstein spaces $M$ corresponding to $\mathbb{RP}^{n}$
are projections from the $2n+1$-dimensional quadrics $Q\subset\mathbb{R}^{n+1,n+1}$
given by $X^{\alpha}Y_{\alpha}=\frac{1}{\Lambda}$, where $X,Y\in\mathbb{R}^{n+1}$
are coordinates on $\mathbb{R}^{n+1,n+1}$ such that the metric is
given by 
\[
G={d}X^{\alpha}{d}Y_{\alpha},
\]
 under the embedding
\begin{eqnarray}
X^{\alpha} & = & \begin{cases}
x^{A}\mathrm{e}^{\tau}, & \alpha=A=0,\dots,n-1\\
\mathrm{e}^{\tau}, & \alpha=n
\end{cases}\nonumber \\
Y_{\alpha} & = & \begin{cases}
z_{A}\mathrm{e}^{-\tau}, & \alpha=A=0,\dots,n-1\\
\mathrm{e}^{-\tau}\Bigl(\frac{1}{\Lambda}-x^{C}z_{C}\Bigr), & \alpha=n
\end{cases}\label{eq:embedding-1}
\end{eqnarray}
following Kaluza-Klein reduction by the vector $\frac{\partial}{\partial\tau}.$
\end{prop}
\noindent
\textbf{Proof. }We find the basis of coordinate 1-forms $\{{d}X^{\alpha},{d}Y_{\alpha}\}$
to be
\begin{eqnarray*}
{d}X^{\alpha} & = & \begin{cases}
\mathrm{e}^{\tau}({d}x^{A}+x^{A}{d}\tau), & \alpha=A=0,\dots,n-1\\
\mathrm{e}^{\tau}{d}\tau, & \alpha=n
\end{cases}\\
{d}Y_{\alpha} & = & \begin{cases}
\mathrm{e}^{-\tau}({d}z_{A}-z_{A}{d}\tau), & \alpha=A=0,\dots,n-1\\
-\mathrm{e}^{-\tau}\biggl[\Bigl(\frac{1}{\Lambda}-x^{C}z_{C}\Bigr){d}\tau+x^{C}{d}z_{C}+z_{C}{d}x^{C}\biggr], & \alpha=n.
\end{cases}
\end{eqnarray*}
The metric is then given by
\begin{eqnarray*}
G & = & \mathrm{e}^{\tau}({d}x^{A}+x^{A}{d}\tau)\mathrm{e}^{-\tau}({d}z_{A}-z_{A}{d}\tau)\ -\ \mathrm{e}^{\tau}{d}\tau\mathrm{e}^{-\tau}\biggl[\Bigl(\frac{1}{\Lambda}-x^{C}z_{C}\Bigr){d}\tau+x^{C}{d}z_{C}+z_{C}{d}x^{C}\biggr]\\
 & = & {d}x^{A}{d}z_{A}\ +\ (x^{A}{d}z_{A}-z_{A}{d}x^{A}){d}\tau\ -\ (x^{A}z_{A}){d}\tau^{2}\ -\ {d}\tau\biggl[\Bigl(\frac{1}{\Lambda}-x^{C}z_{C}\Bigr){d}\tau+x^{C}{d}z_{C}+z_{C}{d}x^{C}\biggr]\\
 & = & {d}x^{A}{d}z_{A}\ -\ \frac{1}{\Lambda}{d}\tau^{2}\ -\ 2z_{A}{d}x^{A}{d}\tau\\
 & = & {d}x^{A}{d}z_{A}\ +\ \Lambda(z_{A}{d}x^{A})^{2}\ -\ \Lambda\Bigl(\frac{{d}\tau}{\Lambda}+z_{A}{d}x^{A}\Bigr)^{2},
\end{eqnarray*}
which is clearly going to give $g_{\Lambda}$ under Kaluza-Klein reduction
by $\frac{\partial}{\partial\tau}$.
\begin{flushright}
$\square$
\par\end{flushright}

Note that the symplectic form $\Omega$ is the exterior derivative
of the potential term $z_{A}{d}x^{A}$, implying a possible
generalisation to the curved case.


\subsection{Curved case}

We now return to a general projective structure $(N,[\nabla])$. Since the
symplectic form picks out the antisymmetric part of the Schouten tensor,
it has the fairly simple form
\[
\Omega_{\Lambda}={d}z_{A}\wedge{d}x^{A}-\frac{\partial_{[A}\Gamma_{B]C}^{C}}{\Lambda(n+1)}{d}x^{A}\wedge{d}x^{B}.
\]
By inspection, this can be written $\Omega_{\Lambda}={d}\mathcal{A}$,
where
\[
\mathcal{A}=z_{A}{d}x^{A}-\frac{\Gamma_{AC}^{C}}{\Lambda(n+1)}{d}x^{A}.
\]
This is a trivialisation of the Kaluza-Klein bundle which we are about
to construct. Note that for $\Lambda=1$, $\Omega$ and
$\mathcal{A}$ remain unchanged under a change of projective connection (\ref{proj_change}).

Motivated by the Kaluza-Klein reduction in the flat case, we consider
the following metric.
\begin{theo}
\label{theokk}
Let $g_{\Lambda}$ be the Einstein metric
(\ref{g_Lambda}) corresponding to the projective structure $(N, [\nabla])$.
The metric
\begin{equation}
\mathcal{G}_{\Lambda}=g_{\Lambda}-\Lambda\Bigl(\frac{{d}t}{\Lambda}+\mathcal{A}\Bigr)^{2}\label{eq:G}
\end{equation}
on a principal circle bundle $\sigma:\mathcal{Q}\mapsto M$ is Einstein,
with Ricci scalar $2n(2n+1)\Lambda$.
\end{theo}
\noindent
\textbf{Proof.} We prove this using the Cartan formalism. Our treatment
parallels a calculation by Kobayashi \cite{Kob}, who considered
principal circle bundles over K\"ahler manifolds in order to study the
topology of the base. Note that we temporarily surpress the constant
$\Lambda$, writing $\mathcal{G}\equiv\mathcal{G}_{\Lambda}$ and
$g\equiv g_{\Lambda}$, since the proof applies to any choice $\Lambda\neq0$
within this family. Consider a frame
\begin{equation}
e^{a}=\begin{cases}
{d}x^{A}, & a=A=0,\dots,n-1\\
{d}z_{A}-(\Gamma_{AB}^{C}z_{C}-\Lambda z_{A}z_{B}-\Lambda^{-1}P_{AB}){d}x^{B}, & a=A+n=n,\dots,2n-1,
\end{cases}\label{eq:basis}
\end{equation}
and in this basis the metric takes the form
\begin{equation}
g=e^{0}\odot e^{n}+\dots+e^{n-1}\odot e^{2n-1}.\label{eq:g_cov_const}
\end{equation}
We are interested in the metric
\[
\mathcal{G}=g-e^{t}\odot e^{t},
\]
where
\[
e^{t}=\sqrt{\Lambda}\biggl(\frac{{d}t}{\Lambda}+{\mathcal A}\biggr).
\]
 We reserve Roman indices $a,b,\dots$ for the $2n$-metric components
$0,\dots,2n-1$ and allow greek indices $\mu,\nu,\dots$ to take values
$0,1,\dots,2n$. The dual basis to $\{e^{\mu}\}$ will be denoted
$\{E_{\mu}\}$ and will act on functions as vector fields in the usual
way. We wish to find the new connection 1-forms $\hat{\psi}_{\ \nu}^{\mu}$
(defined by ${d}e^{\mu}=-\hat{\psi}_{\ \nu}^{\mu}\wedge e^{\nu}$)
in terms of the old ones $\psi_{\ b}^{a}$ (defined by ${d}e^{a}=-\psi_{\ b}^{a}\wedge e^{b}$).
Hence we examine%
\footnote{Note that our conventions are $({d}\omega)_{ab\dots c}=\partial_{[a}\omega_{b\dots c]},$
$(\eta\wedge\omega)_{a\dots d}=\eta_{[a\dots b}\omega_{c\dots d]},$
$\omega=\omega_{a\dots b}{d}x^{a}\wedge\dots\wedge{d}x^{b},$
and $F_{ab}{d}x^{a}\wedge{d}x^{b}=F_{[ab]}{d}x^{a}\otimes{d}x^{b}$
implying ${d}x^{a}\wedge{d}x^{b}=\frac{1}{2}({d}x^{a}\otimes{d}x^{b}-{d}x^{b}\otimes{d}x^{a})$.%
} ${d}e^{t}$ to find $\hat{\psi}_{\ a}^{t}.$
\[
{d}e^{t}=\sqrt{\Lambda}{d}A=\sqrt{\Lambda}\Omega_{ab}e^{a}\wedge e^{b}=-\hat{\psi}_{\ a}^{t}\wedge e^{a}\quad\implies\quad\hat{\psi}_{\ a}^{t}=\sqrt{\Lambda}\Omega_{[ab]}e^{b}=\sqrt{\Lambda}\Omega_{ab}e^{b},\qquad\hat{\psi}_{\ t}^{a}=\sqrt{\Lambda}\Omega_{\ b}^{a}e^{b}.
\]
Since ${d}e^{a}$ is unchanged, we have that
\[
\hat{\psi}_{\ t}^{a}\wedge e^{t}+\hat{\psi}_{\ b}^{a}\wedge e^{b}=\psi_{\ b}^{a}\wedge e^{b},
\]
thus
\[
\hat{\psi}_{\ b}^{a}\wedge e^{b}=\psi_{\ b}^{a}\wedge e^{b}-\sqrt{\Lambda}\Omega_{\ b}^{a}e^{b}\wedge e^{t}\qquad\implies\qquad\hat{\psi}_{\ b}^{a}=\psi_{\ b}^{a}+\sqrt{\Lambda}\Omega_{\ b}^{a}e^{t}.
\]


We now calculate the curvature 2-forms $\hat{\Psi}_{\ \nu}^{\mu}={d}\hat{\psi}_{\ \nu}^{\mu}+\hat{\psi}_{\ \rho}^{\mu}\wedge\hat{\psi}_{\ \nu}^{\rho}=\frac{1}{2}\mathcal{R}_{\rho\sigma\nu}^{\ \ \ \ \mu}e^{\rho}\wedge e^{\sigma}$
in terms of $\Psi_{\ b}^{a}={d}\psi_{\ b}^{a}+\psi_{\ c}^{a}\wedge\psi_{\ b}^{c}$,
where $\mathcal{R}_{\rho\sigma\nu}^{\ \ \ \ \mu}$ is the Riemann
tensor of $\mathcal{Q}$. Note that we use the notation $\psi_{\ b}^{a}=\psi_{\ bc}^{a}e^{c}$
\begin{eqnarray*}
\hat{\Psi}_{\ b}^{a} & = & {d}\hat{\psi}_{\ b}^{a}+\hat{\psi}_{\ c}^{a}\wedge\hat{\psi}_{\ b}^{c}+\hat{\psi}_{\ t}^{a}\wedge\hat{\psi}_{\ b}^{t}\\
 & = & {d}\psi_{\ b}^{a}+{\sqrt{\Lambda}{d}}(\Omega_{\ b}^{a}e^{t})+\psi_{\ c}^{a}\wedge\psi_{\ b}^{c}+\sqrt{\Lambda}\Omega_{\ c}^{a}e^{t}\wedge\psi_{\ b}^{c}+\sqrt{\Lambda}\Omega_{\ b}^{c}\psi_{\ c}^{a}\wedge e^{t}+\Lambda\Omega_{\ [c}^{a}\Omega_{|b|d]}e^{c}\wedge e^{d}\\
 & = & \Psi_{\ b}^{a}+\sqrt{\Lambda}E_{c}(\Omega_{\ b}^{a})e^{c}\wedge e^{t}+\Lambda(\Omega_{\ b}^{a}\Omega_{cd}+\Omega_{\ [c}^{a}\Omega_{|b|d]})e^{c}\wedge e^{d}+\sqrt{\Lambda}(\Omega_{\ b}^{c}\psi_{\ cd}^{a}-\Omega_{\ c}^{a}\psi_{\ bd}^{c})e^{d}\wedge e^{t}\\
 & = & \Psi_{\ b}^{a}+\sqrt{\Lambda}\nabla_{c}\Omega_{\ b}^{a}e^{c}\wedge e^{t}+\Lambda(\Omega_{\ b}^{a}\Omega_{cd}+\Omega_{\ [c}^{a}\Omega_{|b|d]})e^{c}\wedge e^{d}.\\
\hat{\Psi}_{\ a}^{t} & = & {d}\hat{\psi}_{\ a}^{t}+\hat{\psi}_{\ b}^{t}\wedge\hat{\psi}_{\ a}^{b}\\
 & = & \sqrt{\Lambda}E_{[c}(\Omega_{|a|b]})\theta^{c}\wedge\theta^{b}-\sqrt{\Lambda}\Omega_{ab}\psi_{\ c}^{b}\wedge e^{c}+\sqrt{\Lambda}\Omega_{bc}e^{c}\wedge(\psi_{\ a}^{b}+\sqrt{\Lambda}\Omega_{\ a}^{b}e^{t})\\
 & = & \sqrt{\Lambda}(E_{[d}(\Omega_{|a|b]})-\Omega_{ac}\psi_{\ [bd]}^{c}+\Omega_{c[d}\psi_{\ |a|b]}^{c})e^{d}\wedge e^{b}+\Lambda\Omega_{bc}\Omega_{\ a}^{b}e^{c}\wedge e^{t}\\
 & = & \sqrt{\Lambda}\nabla_{[c}\Omega_{|a|d]}e^{c}\wedge e^{d}+\Lambda\Omega_{bc}\Omega_{\ a}^{b}e^{c}\wedge e^{t}.
\end{eqnarray*}
Hence we have that
\begin{eqnarray*}
\mathcal{R}_{cdb}^{\ \ \ a} & = & R_{cdb}^{\ \ \ a}+2\Lambda(\Omega_{\ b}^{a}\Omega_{cd}+\Omega_{\ [c}^{a}\Omega_{|b|d]})\\
\mathcal{R}_{ctb}^{\ \ \ a} & = & \sqrt{\Lambda}\nabla_{c}\Omega_{\ b}^{a}\\
\mathcal{R}_{cda}^{\ \ \ t} & = & 2\sqrt{\Lambda}\nabla_{[c}\Omega_{|a|d]}\\
\mathcal{R}_{cta}^{\ \ \ t} & = & \Lambda\Omega_{bc}\Omega_{\ a}^{b},
\end{eqnarray*}
and thus, using $\mathcal{R}_{\mu\nu}=\mathcal{R}_{\rho\mu\nu}^{\ \ \ \ \rho}$,
\begin{eqnarray*}
\mathcal{R}_{tt} & = & \Lambda\Omega_{bc}\Omega^{bc}=-2n\Lambda=2n\Lambda\mathcal{G}_{tt}\\
\mathcal{R}_{bt} & = & \sqrt{\Lambda}\nabla_{c}\Omega_{\ b}^{c}=0\\
\mathcal{R}_{db} & = & R_{db}+2\Lambda(\Omega_{\ b}^{c}\Omega_{cd}+\frac{1}{2}\Omega_{\ c}^{c}\Omega_{bd}-\frac{1}{2}\Omega_{\ d}^{c}\Omega_{bc})-\Lambda\Omega_{cd}\Omega_{\ b}^{c}\\
 & = & R_{db}+2\Lambda\Omega_{b}^{\ c}\Omega_{dc}\\
 & = & 2(n+1)\Lambda g_{db}-2\Lambda g_{db}=2n\Lambda g_{db}=2n\Lambda\mathcal{G}_{db}.
\end{eqnarray*}


Note that we have used the facts that $g$ is Einstein with Ricci
scalar $4n(n+1)\Lambda$ and that the symplectic form $\Omega$ is divergence-free;
these are justified in the appendix. Since $\mathcal{G}_{at}=0$,
we conclude that
\[
\mathcal{R}_{\mu\nu}=2n\Lambda\mathcal{G}_{\mu\nu}=\frac{\mathcal{R}}{2n+1}\mathcal{G}_{\mu\nu},
\]
i.e. $\mathcal{G}$ is Einstein with Ricci scalar $2n(2n+1)\Lambda$.

\begin{flushright}
$\square$
\par\end{flushright}

Physically, this is a Kaluza-Klein reduction with constant dilation
field and where the Maxwell two-form is related to the reduced metric
by $\Omega_{a}^{\ c}\Omega_{cb}=g_{ab}$. This is what allows both
the reduced and lifted metric to be Einstein. A more general discussion
can be found in \cite{Pope}.

From the Cartan perspective, $\mathcal{G}_{\Lambda=1}$ can be thought
of as a metric on the $2n+1$-dimensional space obtained by taking
a quotient $\tilde{q}:P\mapsto P/SL(n,\mathbb{R})$ of the Cartan
bundle, where we embed $SL(n,\mathbb{R})\subset GL(n,\mathbb{R})$
in $H$ as in (\ref{eq:gl(n)_embed_H}) but with $a$ now denoting
an element of $SL(n,\mathbb{R})$ (so that $\mathrm{det}(a)^{-1}=1$).
This new subgroup acts adjointly on $\theta$ as
\[
\begin{pmatrix}1 & 0\\
0 & a
\end{pmatrix}\begin{pmatrix}-\mathrm{tr}(\phi) & \eta\\
\omega & \phi
\end{pmatrix}\begin{pmatrix}1 & 0\\
0 & a^{-1}
\end{pmatrix}=\begin{pmatrix}-\mathrm{tr}(\phi) & \eta a^{-1}\\
a\omega & \phi
\end{pmatrix},
\]
so not only is the inner product $\eta\omega$ invariant but also
the $(0,0)$-component $\theta_{\ 0}^{0}=\mathrm{-tr}\phi$, which
is a scalar one-form whose exterior derivative is constrained by (\ref{eq:curvature_2-form})
to be $\mathrm{d}\theta_{\ 0}^{0}=-\theta_{\ i}^{0}\wedge\theta_{\ 0}^{i}=-\mathrm{Ant}(\eta\wedge\omega)$.
Thus, denoting by ${A}$ the object on $\mathcal{Q}=P/SL(n,\mathbb{R})$
which is such that $\tilde{q}^{*}A=\mathrm{tr}\phi$, we have that
${d}A=\Omega$ (where we are now taking $\Omega$ and $g$
to be defined on $\mathcal{Q}$ by $\tilde{q}^{*}\Omega=\mathrm{Ant}(\eta\wedge\omega)$
and $\tilde{q}^{*}g=\mathrm{Sym}(\eta\wedge\omega)$ respectively,
or equivalently redefining $\tilde{\Omega}=\sigma^{*}\Omega$ and
$\tilde{g}=\sigma^{*}g$).

We then have a natural way of constructing a metric $\mathcal{G}$
on $\mathcal{Q}$ as a linear combination of $g$ and $e^{t}\odot e^{t}$,
where $e^{t}$ is $A$ up to addition of some exact one-form. It turns
out that the choice of linear combination such that $\mathcal{G}$
is Einstein is 
\[
\mathcal{G}=g-e^{t}\odot e^{t}.
\]
The fact that this metric is exactly (\ref{eq:G}) can be verified
by constructing the Cartan connection of $(S,[\nabla])$ explicitly
in terms of a representative connection $\nabla\in[\nabla]$.

\section{Appendix A: Ricci scalar of $(M,g)$ and divergence of $\Omega$}
We calculate these using the Cartan formalism, again using the basis
(\ref{eq:basis}). In this basis we have $g$ as above (\ref{eq:g_cov_const})
and
\[
\Omega=\sum_{A=0}^{n-1}e^{A}\wedge e^{A+n}\quad\implies\quad\Omega_{ab}=\sum_{A=0}^{n-1}\delta_{[a}^{A}\delta_{b]}^{A+n}.
\]
Note that from now on we will omit the summation sign and use the
summation convention regardless of whether $A, B$--indices are up or
down. As in \S \ref{quadric}, we look for $\psi_{\ b}^{a}$
by considering ${d}e^{a}$ (recall that $A,B=0,\dots,n-1$ and
$a,b=0,\dots,2n-1$):
\begin{eqnarray*}
{d}e^{A} & = & 0\\
{d}e^{A+n} & = & -(E_{D}(\Gamma_{AB}^{C})z_{C}-\Lambda^{-1}E_{D}(P_{AB})){d}x^{D}\wedge{d}x^{B}-(\Gamma_{AB}^{C}-2\Lambda z_{(A}\delta_{B)}^{C}){d}z_{C}\wedge{d}x^{B}\\
 & = & -(E_{D}(\Gamma_{AB}^{C})z_{C}-\Lambda^{-1}E_{D}(P_{AB}))e^{D}\wedge e^{B}\\
 &  & -(\Gamma_{AB}^{C}-2\Lambda z_{(A}\delta_{B)}^{C})(e^{C+n}+(\Gamma_{CE}^{D}z_{D}-\Lambda z_{C}z_{E}-\Lambda^{-1}P_{CE})e^{E})\wedge e^{B}\\
 & = & \bigl[\Lambda^{-1}E_{E}(P_{AB})-E_{E}(\Gamma_{AB}^{C})z_{C}+\Lambda^{-1}\Gamma_{AB}^{C}P_{CE}-\Gamma_{AB}^{C}\Gamma_{CE}^{D}z_{D}+\Lambda\Gamma_{AB}^{C}z_{E}z_{C}\\
 &  & +2\Lambda z_{(A}(\Gamma_{B)E}^{D}z_{D}-\Lambda z_{B)}z_{E}-\Lambda^{-1}P_{B)E})\bigr]e^{E}\wedge e^{B}+(2\Lambda z_{(A}\delta_{B)}^{C}-\Gamma_{AB}^{C})e^{C+n}\wedge e^{B}\\
 & = & \bigl[\Lambda^{-1}D_{E}P_{AB}-(D_{E}\Gamma_{AB}^{C})z_{C}-2z_{(A}P_{B)E}\bigr]e^{E}\wedge e^{B}+(2\Lambda z_{(A}\delta_{B)}^{C}-\Gamma_{AB}^{C})e^{C+n}\wedge e^{B}\\
\end{eqnarray*}
Note that we have used $D$ to denote the chosen connection on $S$
with components $\Gamma_{BC}^{A}$. Next we wish to read off the spin
connection $\psi_{\ b}^{a}$ such that ${d}e^{a}=-\psi_{\ b}^{a}\wedge e^{b}$
and the following index symmetries are satisfied:
\begin{eqnarray*}
\psi_{\ B}^{A} & = & \frac{1}{2}\psi_{A+n\, B}=-\frac{1}{2}\psi_{B\, A+n}=-\psi_{\ A+n}^{B+n}\\
\psi_{\ B+n}^{A} & = & \frac{1}{2}\psi_{A+n\, B+n}=-\frac{1}{2}\psi_{B+n\, A+n}=-\psi_{\ A+n}^{B}\\
\psi_{\ B}^{A+n} & = & \frac{1}{2}\psi_{A\, B}=-\frac{1}{2}\psi_{B\, A}=-\psi_{\ A}^{B+n}
\end{eqnarray*}
We find that 
\begin{eqnarray*}
\psi_{\ C+n}^{A+n} & = & (2\Lambda z_{(A}\delta_{B)}^{C}-\Gamma_{AB}^{C})e^{B}=-\psi_{\ A}^{C}\\
\psi_{\ B}^{A+n} & = & \bigl[2(D_{[A}\Gamma_{B]C}^{D})z_{D}-2\Lambda^{-1}D_{[A}P_{B]C}^{\mathrm{S}}-\Lambda^{-1}D_{C}P_{AB}^{\mathrm{A}}+2z_{(B}P_{C)A}-2z_{(A}P_{C)B}\bigr]e^{C}=:A_{ABC}e^{C}\\
\psi_{\ B+n}^{A} & = & 0.
\end{eqnarray*}
One can check that these satisfy both the index symmetries above and
are such that ${d}e^{a}=-\psi_{\ b}^{a}\wedge e^{b}$, and
we know from theory that there is a unique set of $\psi_{\ b}^{a}$
that have both of these properties. Note that we have used $P^{\mathrm{S}}$
and $P^{\mathrm{A}}$ to denote the symmetric and antisymmetric parts
of $P$ in order to avoid too much confusion from having multiple
symmetrisation brackets in the indices.

We are now ready to calculate the divergence of $\Omega$. Since it
is also covariantly constant in this basis, we obtain
\[
\nabla_{c}\Omega_{ab}=-\psi_{\ ac}^{d}\Omega_{db}-\psi_{\ bc}^{d}\Omega_{ad}=-\psi_{\ ac}^{d}\Omega_{db}+\psi_{\ bc}^{d}\Omega_{da}=2\Omega_{d[a}\psi_{\ b]c}^{d}.
\]
We can split the right hand side into
\begin{eqnarray*}
\Omega_{da}\psi_{\ bc}^{d} & = & \Omega_{Ca}\psi_{\ bc}^{C}+\Omega_{C+n\, a}\psi_{\ bc}^{C+n}\\
 & = & \delta_{[C}^{A}\delta_{a]}^{A+n}\psi_{\ bc}^{C}+\delta_{[C+n}^{A}\delta_{a]}^{A+n}\psi_{\ bc}^{C+n}\\
 & = & \frac{1}{2}\Bigl(-\delta_{a}^{C+n}\delta_{b}^{A}\delta_{c}^{B}(2\Lambda z_{(A}\delta_{B)}^{C}-\Gamma_{AB}^{C})-\delta_{a}^{C}\delta_{b}^{D+n}\delta_{c}^{B}(2\Lambda z_{(C}\delta_{B)}^{D}-\Gamma_{CB}^{D})-\delta_{a}^{C}\delta_{b}^{D}\delta_{c}^{E}A_{CDE}\Bigr).\\
\end{eqnarray*}
The first two terms are the same but with $a\leftrightarrow b$, so
are lost in the antisymmetrisation. Thus
\[
\nabla_{c}\Omega_{ab}=-\delta_{[a}^{C}\delta_{b]}^{D}\delta_{c}^{E}A_{CDE}.
\]
Tracing amounts to contracting this with $g^{ac}$:
\[
\nabla^{c}\Omega_{cb}=-\delta_{[a}^{C}\delta_{b]}^{D}g^{ac}\delta_{c}^{E}A_{CDE}=-\delta_{[a}^{C}\delta_{b]}^{D}g^{aE}A_{CDE},
\]
but $g^{aE}$ is non-zero only when $a=E+n>n$ and $\delta_{[a}^{C}\delta_{b]}^{D}$
is non-zero only when $a=C\leq n$ or $a=D\leq n$. We can therefore
conclude that the right hand side is zero and $\Omega$ is divergence-free.

Finally, we calculate the Ricci scalar of $g$ (given that it's Einstein)
via the curvature two-forms $\Psi_{\ b}^{a}={d}\psi_{\ b}^{a}+\psi_{\ c}^{a}\wedge\psi_{\ b}^{c}=\frac{1}{2}R_{cdb}^{\ \ \ a}e^{c}\wedge e^{d}$.
We are only interested in non-zero components of the Ricci tensor
such as $R_{A\, B+n}=R_{cA\, B+n}^{\ \ \ \ \ \ \ \ c}$. In fact, we will
calculate only $R_{E+n\, B}$, for which we need to consider $R_{D\, E+n\, B}^{\ \ \ \ \ \ \ \ \ A}$
and $R_{D+n\, E+n\, B}^{\ \ \ \ \ \ \ \ \ \ \ \ A+n}$, i.e. we need only calculate
$\Psi_{\ B}^{A}$ and $\Psi_{\ \ B}^{A+n}$. 
\[
\Psi_{\ B}^{A}={d}\Bigl((\Gamma_{BC}^{A}-2\Lambda z_{(B}\delta_{C)}^{A})e^{C}\Bigr)+\psi_{\ C}^{A}\wedge\psi_{\ B}^{C}+\psi_{\ C+n}^{A}\wedge\psi_{\ B}^{C+n}.
\]
The last term vanishes since $\psi_{\ B}^{C+n}=0$, and the middle
term only has components that look like $\frac{1}{2}R_{DEB}^{\ \ \ \ \ A}e^{D}\wedge e^{E}$,
so the only term we are interested in is 
\[
-2\Lambda{d}z_{(B}\delta_{C)}^{A}e^{C}=-2\Lambda\delta_{(C}^{A}(e^{B)+n}+(\Gamma_{B)E}^{D}z_{D}-\Lambda z_{B)}z_{E}-\Lambda^{-1}P_{B)E})e^{E})\wedge e^{C}.
\]
Again, discarding the $e^{E}\wedge e^{C}$ term gives
\[
-\Lambda(e^{B+n}\wedge e^{C}+\delta_{B}^{A}e^{C+n}\wedge e^{C})=\frac{1}{2}R_{D\, E+n\, B}^{\ \ \ \ \ \ \ \ \ A}e^{D}\wedge e^{E+n}+\frac{1}{2}R_{E+n\, D\, B}^{\ \ \ \ \ \ \ \ \ A}e^{E+n}\wedge e^{D},
\]
so we conclude
\[
R_{D\, E+n\, B}^{\ \ \ \ \ \ \ \ \ A}=\Lambda(\delta_{B}^{A}\delta_{D}^{E}+\delta_{D}^{A}\delta_{B}^{E}).
\]
The other Riemann tensor component we need to know to calculate $R_{E+n\, B}=R_{c\, E+n\, B}^{\ \ \ \ \ \ \ \ \ c}$
is $R_{D+n\, E+n\, B}^{\ \ \ \ \ \ \ \ \ \ \ \ A+n}$, so we examine
\[
\Psi_{\ \ B}^{A+n}={d}\psi_{\ \ B}^{A+n}+\psi_{\ \ C}^{A+n}\wedge\psi_{\ B}^{C}+\psi_{\ \ C+n}^{A+n}\wedge\psi_{\ \ B}^{C+n},
\]
but none of these terms have $e^{D+n}\wedge e^{E+n}$ components,
so $R_{D+n\, E+n\, B}^{\ \ \ \ \ \ \ \ \ \ \ \ A+n}=0$. Hence 
\[
R_{E+n\, B}=\delta_{D}^{A}R_{D\, E+n\, B}^{\ \ \ \ \ \ \ \ \  A}=\Lambda(\delta_{B}^{E}+n\delta_{B}^{E})=\Lambda(n+1)\delta_{B}^{E}.
\]
Setting this equal to $\frac{R}{2n}g_{E+n\, B}=\frac{R}{4n}\delta_{B}^{E}$
we find 
\[
R=4n(n+1)\Lambda,
\]
as required.