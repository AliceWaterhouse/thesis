%!TEX root = ../thesis.tex
%*******************************************************************************
%****************************** Chapter 7: concluding_rmks **********************************
%*******************************************************************************
\chapter{Concluding Remarks}

We have found that the modified kink is topologically and linearly stable, and investigated its asymptotic stability for the range of $a$ where exactly one discrete mode is present. It would be interesting to expand the investigation in section \ref{sec:dynamics} to the case when both discrete modes are present. This problem is much more complicated because of the extra terms in (\ref{eq:eta}) and (\ref{eq:alpha}) coming from the amplitude of the second internal mode. Similar problems have been discussed in \cite{Weinstein}, although no such analysis has been done for non--linear Klein--Gordon equation of this type with two discrete modes. The $\phi^4$ theory on the wormhole presents a useful setting to undertake such analysis because the kink has exactly two discrete modes for any $a>a_1$, and because their frequencies can be controlled by the parameter $a$.

This model shares an interesting property with its sine--Gordon counterpart in that we expect a discontinuous change in decay behaviour when $a$ moves out of the range $a_0<a<a_1$. Insight from the $\phi^4$ case may help to elucidate the character of such discontinuous changes.