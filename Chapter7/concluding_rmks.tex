%!TEX root = ../thesis.tex
%*******************************************************************************
%****************************** Chapter 7: concluding_rmks **********************************
%*******************************************************************************
\chapter{Concluding Remarks}

We use this chapter to discuss some of the open questions which remain unanswered by the work presented here.

We began by discussing a modified $\phi^4$ theory on a wormhole spacetime, finding that there is a kink solution and that it is topologically and linearly stable. We investigated its asymptotic stability for the range of $a$ where exactly one discrete mode is present. It would be interesting to expand the investigation in Section \ref{sec:dynamics} to the case when both discrete modes are present. This problem is much more complicated because of the extra terms which arise from the amplitude of the second internal mode. Similar problems have been discussed in \cite{Weinstein}, although no such analysis has been done for non--linear Klein--Gordon equation of this type with two discrete modes. The $\phi^4$ theory on the wormhole presents a useful setting to undertake such analysis because the kink has exactly two discrete modes for any $a>a_1$, and because their frequencies can be controlled by the parameter $a$.

The modified $\phi^4$ model shares an interesting property with the modified sine--Gordon theory on the same wormhole spacetime \cite{SG}. In both cases, we expect a discontinuous change in decay behaviour when $a$ moves out of the range $a_0<a<a_1$. Insight from the $\phi^4$ case may help to elucidate the character of such discontinuous changes.

In this thesis, we have considered the manifolds $M$ arising in the projective to Einstein correspondence from a number of different perspectives. However, there remain several open questions. In Chapter \ref{chap:KK_lift} we showed that a second Einstein metric can be canonically constructed on an $\R^*$--bundle over $M$, and interpreted this space in terms of the Cartan bundle of the projective structure. It would be interesting to understand what this larger manifold is from the tractor perspective.

In Chapter \ref{chap:EW_and_toda} we constructed several examples of Einstein--Weyl structures arising as Jones--Tod reductions of the Einstein manifolds $M$ in the special case $n=2$, where $M$ has ASD conformal curvature, and extracted  corresponding solutions to the $SU(\infty)$--Toda equation. Although we produced many local expressions for Einstein--Weyl structures, we were unable to provide a coordinate and scale invariant local characterisation of the Einstein--Weyl structures which are obtainable from the projective to Einstein correspondence. We know that they belong to the $SU(\infty)$--Toda class, and we showed in Section \ref{sec:monopoles} that they carry two solutions to the Abelian monopole equation, but it would be interesting to try and extend this result to a complete local characterisation.

Finally, in Chapter \ref{chap:c-proj} we showed that the manifolds $M$ are \textit{para--$c$--projectively compact} using a local form of the metric. A more fundamental understanding of this result could be obtained by realising $M$ in terms of the curved orbit decomposition of a para--$c$--projective structure, viewed as a Cartan holonomy reduction.