% ************************** Thesis Abstract *****************************
% Use `abstract' as an option in the document class to print only the titlepage and the abstract.
\begin{abstract}
The first part of this thesis is concerned with $\phi^4$ field theory on a wormhole spacetime in $3+1$ dimensions. This spacetime has two asymptotically flat ends connected by a spherical throat of radius $a$. We show that the theory possesses a kink solution which is linearly stable, and compare its discrete spectrum to that of the $\phi^4$ kink on $\R^{1,1}$. We present some results on the non--linear resonant coupling between the discrete and continuous spectra in the range of $a$ where there is exactly one discrete mode.

The second part of the thesis is based on recent work by Dunajski and Mettler. They show that a class of neutral signature Einstein manifolds $M$ can be canonically constructed as rank $n$ affine bundles over projective structures in dimension $n$. These have the same symmetry group as the underlying projective manifold, and are also endowed with a natural symplectic form, which is related to the metric by an endomorphism of the tangent bundle that squares to the identity. Consequently, they carry an almost para--K\"ahler structure.

We show that every metric within the class is a Kaluza--Klein reduction of an Einstein metric on an $\R^*$ bundle over $M$. We also show that the structures are para--$c$--projectively compact in the sense of \v Cap--Gover, and intepret the compactification in terms of the tractor bundle of the projective structure.

In dimension four, the manifolds $M$ have anti--self--dual conformal curvature, and are thus associated with a twistor space. In the presence of a symmetry, they can be reduced to Einstein--Weyl structures in dimension three via the Jones--Tod correspondence. Because $M$ is also Einstein with non--zero scalar curvature, these Einstein--Weyl structures are determined by solutions of the $SU(\infty)$--Toda equation.

We classify the Einstein--Weyl structures which can be obtained in this way in terms of the symmetry group of the underlying projective surface. Several examples are considered in detail, resulting in new, explicit solutions of the $SU(\infty)$--Toda equation. We focus in particular on the case where the projective structure is $\RP^n$, additionally describing the Jones--Tod reduction from the twistor perspective.

\end{abstract}
